\documentclass[runningheads]{llncs}
\title{A High-Order Discontinuous Galerkin Solver with Dynamic Adaptive Mesh Refinement to Simulate Cloud Formation Processes }
\titlerunning{A High-Order DG Solver to Simulate Cloud Formation Processes}
\author{Lukas Krenz\inst{1} \and{} Leonhard Rannabauer\inst{1} \and{} Michael Bader\inst{1}}
\authorrunning{L.\ Krenz, L.\ Rannabauer, M.\ Bader}
\institute{Department of Informatics, Technical University of Munich\\
  \email{lukas.krenz@in.tum.de}, \email{rannabauer@in.tum.de}, \email{bader@in.tum.de}
\vspace*{-0.2cm}
} % todo emails and stuff
\usepackage[utf8]{inputenc}
%\usepackage[T1]{fontenc}
\usepackage[american]{babel}
\usepackage[autostyle, english = american]{csquotes}
\usepackage[%
  backend=biber,
  url=false,
  doi=false,
  % style=alphabetic,
  % backref=true,
  % hyperref=true,
  maxnames=3,
  % minnames=3,
  % maxbibnames=99,
  firstinits=true,
  % uniquename=init
  ]{biblatex}
\DefineBibliographyStrings{english}{%
  backrefpage = {see p.},
  backrefpages = {see pp.}
}
\addbibresource{../bibliography.bib}

\usepackage{caption}
\usepackage{subcaption}
\usepackage{xparse} % for NewDocumentCommand
\usepackage{etoolbox} % for notblank (brackets only when argument)
\usepackage{xstring} % for \IfSubStr
\usepackage{xpatch}
\usepackage{xcolor}
\usepackage{amsmath}
\usepackage{amsfonts}
\usepackage{amssymb}
\usepackage{mathtools} % for \mathclap
\usepackage{nicefrac}
\usepackage{physics} % for derivatives
\usepackage{varioref}
\usepackage{nicefrac}
\usepackage{physics} % for derivatives
\usepackage[separate-uncertainty]{siunitx}
\usepackage{hyperref}
\usepackage[]{cleveref}
\newcommand{\creflastconjunction}{, and\nobreakspace} % use Oxford comma
\usepackage{todonotes}
\usepackage{multimedia}
\graphicspath{{../figures/}}
\let\boundary\undefined%
\crefformat{equation}{\eqA{}Eq.~\eqB #2#1#3)}
% TODO: Fix for more than two equations!
\crefmultiformat{equation}{Eqs.~\eqMultiA#2#1#3\eqMultiB}%
{ and \eqMultiA#2#1#3\eqMultiB}{, \eqMultiA#2#1#3\eqMultiB}{ and~\eqMultiA#2#1#3\eqMultiB}
\newcommand{\eqA}{}
\newcommand{\eqB}{(}
\newcommand{\eqMultiA}{(}
\newcommand{\eqMultiB}{)}
\DeclareRobustCommand{\pcrefSingle}[1]{%
\begingroup%
  \renewcommand{\eqA}{(}\renewcommand{\eqB}{}%
\cref{#1}%
\endgroup%
}
\DeclareRobustCommand{\pcrefMulti}[1]{%
\begingroup%
    \renewcommand{\eqMultiA}{}\renewcommand{\eqMultiB}{}%
    (\cref{#1})%
\endgroup%
}
\DeclareRobustCommand{\pcref}[1]{%
\IfSubStr{#1}{,}{\pcrefMulti{#1}}{\pcrefSingle{#1}}%
}

\usepackage{bm}


% Commands/Macros
% Names of methods or so
\newcommand{\muscl}{\textsc{muscl}-Hancock}
\newcommand{\dg}{\textsc{dg}}
\newcommand{\ader}{\textsc{ader}}
\newcommand{\aderdg}{\textsc{ader-dg}}
\newcommand{\amr}{\textsc{amr}}
\newcommand{\pde}{\textsc{pde}}
\newcommand{\cpp}{C\texttt{++}}
\newcommand{\tbb}{\textsc{tbb}}
\newcommand{\mpi}{\textsc{mpi}}

\newcommand{\softwareName}[1]{\textit{#1}}
\newcommand{\exahype}{\softwareName{ExaHyPE}}
\newcommand{\exahypeengine}{\softwareName{ExaHyPE-Engine}}
\newcommand{\peano}{\softwareName{Peano}}
\newcommand{\className}[1]{\texttt{#1}}
\newcommand{\funName}[1]{\textsf{#1}}
\newcommand{\fileName}[1]{\textbf{#1}}
\newcommand{\varName}[1]{\enquote{#1}}



% Variables and other equation stuff
\newcommand{\Q}{\bm{Q}}
\newcommand{\gradQ}{\gradient{\Q}}
\newcommand{\Qrho}{\rho}
\newcommand{\Qj}{\rho \bm{v}}
\newcommand{\Qv}{\bm{v}}
\newcommand{\QE}{\rho E}
\newcommand{\QZZ}{Z} % TODO: Name?
\newcommand{\QZ}{\rho \QZZ}
\newcommand{\potT}{\theta}
\newcommand{\backgroundPotT}{\overline{\theta}}
\newcommand{\pertubationPotT}{\theta'}
\newcommand{\stressT}{\bm{\sigma}}
\newcommand{\pressure}{p}
\newcommand{\maxConvEigen}[1][]{
  \vert%
  \lambda_c^{\text{max}}
  \notblank{#1}{\left(#1\right)}{}
  \vert%
}
\newcommand{\maxViscEigen}[1][]{
  \vert%
  \lambda_v^{\text{max}}
  \notblank{#1}{\left(#1\right)}{}
  \vert%
}
\newcommand{\Riemann}{\operatorname{Riemann}}

% Domain
\newcommand{\domain}{\Omega}
\newcommand{\boundary}{\partial \domain}
\newcommand{\broken}{\domain}

% Cells
\newcommand{\cell}[1][]{C_{#1}}
\newcommand{\refCell}[1][]{\hat{C}_{#1}}
\newcommand{\cellb}{\partial{} \cell}
\newcommand{\refCellb}{\partial{} \refCell}

% Basis functions
\newcommand{\lagrange}[1][i]{\varphi_{#1}}
\newcommand{\lagrangeRef}[1][i]{\hat{\varphi}_{#1}}
\NewDocumentCommand{\sbasis}{ O{} O{} }{\phi^{#1}_{#2}}
\NewDocumentCommand{\sbasisRef}{ O{} O{} }{\hat{\phi}^{#1}_{#2}} %{\hat{\sbasis[#1][#2]}}
\NewDocumentCommand{\stbasis}{ O{} O{} }{ \psi^{#1}_{#2}}
\NewDocumentCommand{\stbasisRef}{ O{} O{} }{\hat{\psi}^{#1}_{#2}} %{\hat{\stbasis[#1][#2]}}
%\newcommand{\stestfunction}[1]{\sbasis[#1]}
\NewDocumentCommand{\stestfunction}{ O{} O{} }{\sbasis[#1][#2]}
\NewDocumentCommand{\sttestfunction}{ O{} O{} }{\stbasis[#1][#2]}

% Bold if no index is supplied.
\newcommand{\bmempty}[2]{
\notblank{#2}{#1}{\bm{#1}}
}

% Space-time dofs
\NewDocumentCommand{\stpredictor}{ O{} O{} }{\overline{\bmempty{q}{#2}}^{#1}_{#2}}
\NewDocumentCommand{\stpredictorCoeff}{ O{} O{} }{\underline{\overline{\bmempty{q}{#2}}}^{#1}_{#2}}

\NewDocumentCommand{\stflux}{ O{} O{} }{\overline{\bmempty{F}{#2}}^{#1}_{#2}}
\NewDocumentCommand{\stfluxCoeff}{ O{} O{} }{\underline{\overline{\bmempty{F}{#2}}}^{#1}_{#2}}

% \NewDocumentCommand{\stsol}{ O{} O{} }{\overline{u}^{#1}_{#2}}
% \NewDocumentCommand{\stsol}{ O{} O{} }{\underline{\overline{u}}^{#1}_{#2}}

\NewDocumentCommand{\stsource}{ O{} O{} }{\overline{\bmempty{S}{#2}}^{#1}_{#2}}
\NewDocumentCommand{\stsourceCoeff}{ O{} O{} }{\underline{\overline{\bmempty{S}{#2}}}^{#1}_{#2}}

% Space dofs
\NewDocumentCommand{\spredictor}{ O{} O{} }{\bmempty{q}{#2}^{#1}_{#2}}
\NewDocumentCommand{\spredictorCoeff}{ O{} O{} }{\underline{q}^{#1}_{#2}}

\NewDocumentCommand{\ssource}{ O{} O{} }{\bmempty{S}{#2}^{#1}_{#2}}
\NewDocumentCommand{\ssourceCoeff}{ O{} O{} }{\underline{\bmempty{S}{#2}}^{#1}_{#2}}

\NewDocumentCommand{\ssol}{ O{} O{} }{\bmempty{u}{#2}^{#1}_{#2}}
\NewDocumentCommand{\ssolCoeff}{ O{} O{} }{\underline{\bmempty{u}{#2}}^{#1}_{#2}}

\NewDocumentCommand{\sflux}{ O{} O{} }{\bmempty{F}{#2}^{#1}_{#2}}
\NewDocumentCommand{\sfluxCoeff}{ O{} O{} }{\underline{\bmempty{F}{#2}}^{#1}_{#2}}

% Equation parts
\newcommand{\flux}{\bm{F}}
\newcommand{\viscFlux}{\flux^{v}}
\newcommand{\hyperFlux}{\flux^{h}}
%\newcommand{\source}{\bm{S}}
\newcommand{\source}[1][]{
  \notblank{#1}{
S_{#1}
}{
\bm{S}
}
}

% Integrals
\newcommand{\curTime}{t^k}
\newcommand{\nextTime}{t^{k+1}}
\newcommand{\intdt}[1]{\int_{\curTime}^{\nextTime} #1 \dd{t}}
\newcommand{\intdcell}[1]{\int_{\cell} #1 \dd{\bm{x}}}
\newcommand{\intdrefcell}[1]{\int_{\refCell} #1 \dd{\hat{\bm{x}}}}
\newcommand{\intdcellb}[1]{\int_{\cellb} #1 \dd{S}} % TODO: define boundary of cell
\newcommand{\intdrefcellb}[1]{\int_{\refCellb} #1 \hat{\dd{S}}} % TODO: define boundary of cell
\newcommand{\intddomain}[1]{\int_{\domain} #1 \dd{\bm{x}}}

% Yet to be sorted
\newcommand{\quadWeight}[1][i]{w_{#1}}
\newcommand{\quadNode}[1][i]{\hat{x}_{#1}}
\newcommand{\quadNodeNd}[1][i]{\hat{\bm{x}}_{#1}}
\newcommand{\sbasisSize}{(N+1)^\dimensions}
\newcommand{\stbasisSize}{(N+1)^{\dimensions+1}}
\newcommand{\normal}{\bm{n}}

\newcommand{\pleft}{\bm{L}}
\newcommand{\prightsol}{\bm{r}}
\newcommand{\prightpred}{\bm{w}}

\newcommand{\reactionTimescale}{\tau}
\newcommand{\reactionTemperature}{T_{\text{ign}}}

\newcommand{\tv}{\operatorname{TV}}

\newcommand{\NVar}{N^{\text{var}}}
\newcommand{\dimensions}{d}
\newcommand{\mapping}{\mathcal{M}}
\newcommand{\volume}{V}
\newcommand{\cellCenter}{\operatorname{cell-center}}

% MUSCL
\newcommand{\cellAvg}[1][i,j]{\bm{U}_{#1}}
\newcommand{\extrapolatedCellAvg}[3][i,j]{\cellAvg[#1]^{#3 #2} = \cellAvg #3 \frac{1}{2} \slope{#2}}%
\newcommand{\evolvedCellAvg}[2][i,j]{\hat{\bm{U}}_{#1}^{#2}}
\newcommand{\sign}{\operatorname{sign}}
\newcommand{\minmod}{\operatorname{minmod}}
\newcommand{\slope}[2][i,j]{\bm{s}^{#2}_{#1}}
\newcommand{\gradCellAvg}[1][i,j]{\gradient{\cellAvg[#1]}}
\newcommand{\fluxX}{\flux_x}
\newcommand{\fluxY}{\flux_y}%


\begin{document}
\maketitle 
\begin{abstract}
  We present a high-order discontinuous Galerkin (DG) solver of the compressible Navier Stokes equations for cloud formation processes.
  The scheme exploits an underlying parallelized implementation of the \aderdg{} method with dynamic adaptive mesh refinement. 
  We improve our method by a \pde-independent general refinement criterion, based on the local total variation of the numerical solution.
  Our generic scheme shows competitive results for both classical \textsc{cfd} and stratified scenarios.
  While established methods use tailored numerics towards the specific simulation, our scheme works scenario independent.
  All together our method can be seen as a perfect candidate for large scale cloud simulation runs on current and future super computers.
% \begin{itemize}
% \item Implementation of compressible Navier Stokes with a high-order \aderdg{}-method and a low order finite volume \muscl{}-scheme.
% \item Implementation in \exahypeengine{}, a scalable \pde{} framework.
% \item Outlier and total variation based adaptive mesh refinement that reduced number of grid cells drastically and preserves solution quality.
% \item Simulation of challenging flows over hydrostatic background atmosphere.
% \item Competitive results for both classical cfd scenarios and hydrostatic scenarios in two and three dimensions.
% \end{itemize}

  
%The abstract should briefly summarize the contents of the paper in
%150--250 words.

\keywords{\aderdg{}  \and Navier-Stokes \and Adaptive Mesh Refinement.}
\end{abstract}
\section{Introduction}
In this paper we address the resolution of basic cloud formation processes on modern super computer systems.
%%of convective processes in the context of numerical weather prediction (\textsc{nwp}) on modern super computer systems.
The simulation of cloud formations, as part of convective processes, is expected to play an important role in future numerical weather prediction~\cite{bauer2015quiet}.
%% Numerical weather prediction (\textsc{nwp}) contains processes of different spatio-temporal scales.
%% An important part of future \textsc{nwp} models will be the the resolution of convective processes~\cite{bauer2015quiet}.
This requires both suitable physical models and effective computational realisations. 
Here we focus on the simulation of simple benchmark scenarios~\cite{giraldo2008study}.
They contain relatively small scale effects which are well approximated with the compressible Navier-Stokes equations.
We use the \aderdg{} method of~\cite{dumbser2008unified}, which allows us to simulate the Naview-Stokes equations with a space-time-discretization of arbitrary high order.
In contrast to Runge-Kutta time integrators or semi-implicit methods, an increase of the order of \aderdg{} only results in larger computational kernels and does not affect the complexity of the scheme.
Additionally, \aderdg{} is a communication avoiding scheme and reduces the overhead on larger scale.
We see our scheme in the regime of already established methods for cloud simulations, as seen for example in~\cite{giraldo2008study,muller2010adaptive,muller2018strong}.
%%The general approach is to use a space-discretized by higher order Galerkin method, as seen for example in~\cite{giraldo2008study,muller2010adaptive,muller2018strong}.
%%They are often space-discretized by higher order Galerkin methods, as seen for example in~\cite{giraldo2008study,muller2010adaptive,muller2018strong}.
%%Even though all of these implementations use a discretization with a (theoretical) unlimited space-order, they use Runge-Kutta time integrators or semi-implicit methods.
%%It is well known that Runge-Kutta methods become more ineffecient for very high orders.
%%Furthermore, they and the implicit methods are more communication-intensive and thus require a larger overhead and are more complicated to scale.
%%We follow a different approach in this paper and use the \aderdg{} method of~\cite{dumbser2008unified} to simulate various computational fluid dynamics (\textsc{cfd}) scenarios with a focus on flows with stratified flows.

Due to the viscous components of the Navier-Stokes equations, it is not straightforward to apply the \aderdg{} formalism of~\cite{dumbser2008unified}, which addresses hyperbolic systems of partial differentials equations (\pde{}s) in first-order formulation.
To include viscosity, we use the numerical flux for the compressible Navier-Stokes equations of \citeauthor{gassner2008discontinuous}~\cite{gassner2008discontinuous}.
This flux has already been applied to the \aderdg{} method in~\cite{dumbser2010arbitrary}.
In contrast to this paper, we focus on the simulation of complex flows with a gravitational source term and a realistic background atmosphere.
Additionally, we use adaptive mesh refinement (\textsc{amr}) to increase the spatial resolution in areas of interest.
This has been shown to work well for the simulation of cloud dynamics~\cite{muller2010adaptive}.
Regarding the issue of limiting in high-order \dg{} methods, we note that viscosity not only models the correct physics of the problem but also smoothes oscillations and discontinuities, thus stabilizing the simulation.

We base our work on the \exahypeengine{} (\url{www.exahype.eu}), which is a framework that can solve arbitrary hyperbolic \pde\ systems.
A user of the engine is provided with a simple code interface which mirrors the parts required to formulate a well-posed Cauchy problem for a system of hyperbolic \pde{}s of first order.
The underlying \aderdg{} method\todo{\muscl{}?}, parallelization techniques and dynamic adaptive mesh refinement are available for simulations while the implementations are left as a black box to the user.
An introduction to the communication-avoiding implementation of the whole numerical scheme can be found in~\cite{charrier2018stop}.

To summarize, we make the following contributions in this paper:%
\begin{itemize}%
\item We extend the \exahypeengine{} to allow viscous terms following the aforementioned numerical scheme.
\item We thus provide an implementation of the compressible Navier-Stokes equations.
  In addition, we tailor the equation set to stratified flows with gravitational source term.
  We emphasize that we use a standard formulation of the Navier-Stokes equations as seen in the field of computational fluid mechanics and only use small modifcations of the governing equations, in contrast to a equation set that is tailored exactly to the application area.
\item We present a general \textsc{amr}-criterion that is based on the detection of outlier cells w.r.t.\ their total variation.
  Furthermore, we show how to utilize this criterion for stratified flows.
\item We evaluate our implementation with standard \textsc{cfd} scenarios and atmospheric flows and inspect the effectiveness of our proposed \amr{}-criterion.
  We thus inspect, whether our proposed general implementation can achieve results that are competitive with the state-of-the-art models that rely on heavily specified equations and numerics.
\end{itemize}
%
\section{Equation Set}
\newcommand{\diffCoeff}{\varepsilon}%
\newcommand{\hyperFluxDef}{
  \begin{pmatrix}
    \Qj \\
    \Qv  \otimes \Qj + \bm{I} \pressure  \\
    \Qv \cdot (\bm{I} \QE + \bm{I} \pressure)
  \end{pmatrix}
}%
\newcommand{\viscFluxDef}{
  \begin{pmatrix}
    0\\
     \stressT (\Q, \gradQ)  \\
     \Qv \cdot \stressT (\Q, \gradQ) - \kappa \gradient{T}
   \end{pmatrix}
}%
% \begin{equation}
%   \label{eq:ns-short}
%  \pdv{}{t} \bm{Q} + \divergence{\bm{F}(\bm{Q}, \gradient{\bm{Q}})} = \bm{S}(\bm{Q}) 
% \end{equation}
The compressible Navier-Stokes equations in the conservative form are given as% \pcref{eq:conservation-law-gradient}
\begin{equation}
 \label{eq:equation-set} 
%  \begin{array}{l}
%  \text{mass cons.} \\
%  \text{momentum cons.} \\
%  \text{energy cons.} \\
%  \text{cont.\ gas} 
% \end{array}
% :
\quad
  \pdv{}{t}
  \underbrace{
  \begin{pmatrix}
    \Qrho\\
    \Qj\\
    \QE
    \end{pmatrix}}_{\Q}
  +
  \divergence{
  \underbrace{
  \left(
   \underbrace{\hyperFluxDef}_{\hyperFlux(\Q)}
+
\underbrace{\viscFluxDef}_{\viscFlux(\Q, \gradQ)}
  \right)}_{\flux(\Q, \gradQ)}}
 =
  \underbrace{
  \begin{pmatrix}
    \source[\Qrho\phantom{\Qrho}]\\
    \source[\Qj]\\
    \source[\QE]
    \end{pmatrix}}_{\source(\Q, \bm{x}, t)}
\end{equation}
with the vector of conserved quantities $\Q$, flux $\flux(\Q, \gradQ)$ and source $\source(\Q)$.
Note that the flux can be split into a hyperbolic part $\hyperFlux(\Q)$,
which is identical to the flux of the Euler equations,
and a viscous part $\viscFlux(\Q, \gradQ)$.
The conserved quantities
\(\Q\)
% \begin{equation} 
%   \label{eq:conserved-variables}
%  \Q = \left( \Qrho, \Qj, \QE, \QZ \right),
% \end{equation}
are the density $\Qrho$, the two or three-dimensional momentum $\Qj$ and the energy density $\QE$.
The rows of \cref{eq:equation-set} are the conservation of mass, the conservation of momentum and the conservation of energy.

The pressure $\pressure$ is given by the equation of state of an ideal gas
\begin{equation}
  \label{eq:eos}
  \pressure = (\gamma - 1) \left(\QE - \frac{1}{2} \left(\Qv \cdot \Qj \right) - gz \right).
\end{equation}
The term $gz$ is the geopotential height with the gravity of Earth $g$~\cite{giraldo2008study}.
The temperature $T$ relates to the pressure by the thermal equation of state
\begin{equation}
  \label{eq:temperature}
  \pressure = \Qrho R T,
\end{equation}
where $R$ is the specific gas constant of a fluid.

We model the diffusivity by the stress tensor
\begin{equation}
  \label{eq:stress-tensor}
  \stressT(\Q, \gradQ) =
  \mu
  \bigl(
  \left(\nicefrac{2}{3} \divergence{\Qv} \right) -
  \left( \gradient{\Qv} + \gradient{\Qv}^\intercal \right)
  \bigr),
\end{equation}
with constant viscosity $\mu$.
The heat diffusion is governed by the coefficient
\begin{equation}
  \label{eq:heat-conduction-coeff}
  \kappa = \frac{\mu \gamma}{\Pr} \frac{1}{\gamma - 1} R = \frac{\mu c_p}{\Pr},
\end{equation}
where the ratio of specific heats $\gamma$, the heat capacity at constant pressure $c_p$ and the Prandtl number $\Pr$ depend on the fluid.

Many realistic atmospheric flows can be described by a perturbation over a background state that is in hydrostatic equilibrium
\newcommand{\backgroundPressure}{\overline{\pressure}}
\newcommand{\backgroundRho}{\overline{\Qrho}}
\begin{equation}
  \label{eq:hydrostatic-balance}
  \pdv{}{z} \backgroundPressure{\left (z \right )} = -g \backgroundRho(z),
\end{equation}
i.e.\ a state, where the pressure gradient is exactly in balance with the gravitational source term $\source[\Qj] = - \bm{k} \Qrho g$.
The vector $\bm{k}$ is the unit vector pointing in $z$-direction.
The momentum equation is dominated by the background flow in this case.
Because this can lead to numerical instabilities, problems of this kind are challenging and require some care.
To lessen the impact of this, we split the pressure $\pressure = \backgroundPressure + \pressure'$ into a sum of the background pressure $\backgroundPressure(z)$ and perturbation $\pressure'(\bm{x}, t)$.
We split the density $\Qrho = \backgroundRho + \Qrho'$ in the same manner and arrive at
\begin{equation}
  \label{eq:momentum-equation-split}
  \pdv{\Qj}{t}+ \divergence{ \left(
    \Qv \otimes \Qj + \bm{I} \pressure'
    \right)
  } + \viscFlux_\Qj
  =
  -g \bm{k} \Qrho'.
\end{equation}
Note that a similar and more complex splitting is performed in~\cite{muller2010adaptive,giraldo2008study}.
In contrast to this, we use the true compressible Navier-Stokes equations with minimal modifications.

\section{Numerics}
The \exahypeengine{} implements an \aderdg{}-scheme and a \muscl{} finite volume method.
Both can be considered as instances of the more general \textsc{PnPm} schemes of~\cite{dumbser2008unified}.
We use a Rusanov-style flux that is adapted to \pde{}s with viscous terms~\cite{gassner2008discontinuous,fambri2017space}.
The finite volume scheme is stabilized with the van Albada limiter~\cite{van1997comparative}.
The user can state dynamic \amr{} rules by supplying custom criteria that are evaluated point-wise.
Our criterion uses an element-local error estimate based on the total variation of the numerical solution.
We exploit the fact that the total variation of a numerical solution is a perfect indicator for edges of a wavefront.
Let $\bm{f}(\bm{x}): \mathbb{R}^{N_\text{vars}} \to \mathbb{R}$ be a sufficiently smooth function that maps the discrete solution at a point $\bm{x}$ to an arbitrary indicator variable.
The total variation (\textsc{tv}) of this function is defined by
\begin{equation}
  \label{eq:tv}
  \tv \left[ f(\bm{x}) \right] =
  \left\Vert
\intdcell{ \vert \gradient{f \left( \bm{x} \right)} \vert }
\right\Vert_1
\end{equation}
for each cell.
The operator $\Vert \cdot \Vert_1$ denotes the discrete $L_1$ norm in this equation.
We compute the integral efficiently with Gaussian quadrature over the collocated quadrature points.
\newcommand{\mean}{\mu}%
\newcommand{\std}{\sigma}%
\newcommand{\variance}{\std^2}%
\newcommand{\Trefine}{T_\text{refine}}%
\newcommand{\Tdelete}{T_\text{coarsen}}%
How can we decide whether a cell is important or not?
To resolve this conundrum, we compute the mean and the population standard deviation of the total variation of all cells.
It is important that we use the method of~\cite{chan1982updating} to compute the modes in a parallel and numerical stable manner.
A cell is then considered to contain significant information if its deviates from the mean more than a given threshold.
This criterion can be described formally by
\begin{equation}
  \label{eq:refinement-criterion}
  \operatorname{evaluate-refinement}(\Q, \mu, \sigma) =
  \begin{cases}
    \text{refine} & \text{if } \tv(\Q) \geq \mu + \Trefine \sigma, \\
    \text{coarsen} & \text{if } \tv(\Q) < \mu + \Tdelete \sigma, \\
    \text{keep} & \text{otherwise}.
    \end{cases}
\end{equation}
The parameters $\Trefine > \Tdelete$ can be chosen freely.
Chebyshev's inequality
\begin{equation}
  \label{eq:chebychev}
  \mathbb{P}\bigl(\vert X - \mu \vert \geq c \sigma \bigr) \leq \frac{1}{c^2},
\end{equation}
with probability $\mathbb{P}$ guarantees that we neither mark all cells for refinement nor for coarsening.
This inequality holds for arbitrary distributions under the weak assumption that they have a finite mean $\mu$ and a finite standard deviation $\sigma$~\cite{wasserman2004all}.
Note that subcells are coarsened only if all subcells belonging to the coarse cell are marked for coarsening.
In contrast to already published criteria which are either designed solely for the simulation of clouds~\cite{muller2010adaptive} or computationally expensive~\cite{fambri2017space}, our criterion works for arbitrary \pde{}s and yet, is easy to compute and intuitive.

\section{Results}
We use a mix of various benchmarking scenarios that consist of standard \textsc{cfd} scenarios and atmospheric flows.
%
For the standard \textsc{cfd} testing scenarios, we use the following constants of fluids:
\begin{equation}
  \gamma = 1.4, \quad \Pr = 0.7, \quad c_v = 1.0.
\end{equation}
A simple scenario is the Taylor-Green vortex.
It has an analytical solution in the incompressible limit which is given by
\begin{align}
  \label{eq:taylor-green}
  \begin{split}
  \Qrho(\bm{x}, t) &= 1,\\
  \Qv(\bm{x}, t) &= \exp(-2 \mu t)
  \begin{pmatrix}
    \phantom{-}\sin(x) \cos(y) \\
- \cos(x) \sin(y) 
    \end{pmatrix}, \\
  \pressure(\bm{x}, t) &= \exp(-4 \mu t) \, \nicefrac{1}{4} \left( \cos(2x) + \cos(2y) \right) + C.
  \end{split}
\end{align}
The constant $C = \nicefrac{100}{\gamma}$ governs the speed of sound and thus the Mach number $\text{Ma} = 0.1$~\cite{dumbser2016high}. The viscosity is set to $\mu = 0.1$. 

We simulate on the domain $[0,2\pi]^2$ 
%We use a domain of size 
%$(\SI{2 \pi}{\m} \times \SI{2 \pi}{\m})$ 
%% keine Meter-Angaben bei analytischen Testcases
and impose the analytical solution at the boundary.
A comparison at time $t = 10.0$ of the analytical solution for the pressure with our approximation (\cref{fig:taylor-green}) shows excellent agreement.
We used an \aderdg{}-scheme of order $5$ with a grid of $25^2$ cells.

\begin{figure}[tb]
  \centering
  \begin{subfigure}[t]{.473\textwidth}
    \centering
    \includegraphics{paper_lid_driven_cavity}
    \caption{\label{fig:cavity-flow}Our approximation (solid lines) of the lid-driven cavity flow vs.\ reference solution (crosses) of~\cite{ghia1982high}.
    The respective other coordinate is held constant at a value of 0.}
  \end{subfigure}\quad%
  \begin{subfigure}[t]{0.473\textwidth}
    \centering
    \includegraphics{paper_taylor_green_vel}
    \caption{\label{fig:taylor-green}Our result (markers) of the Taylor-Green vortex vs.\ the analytical solution (lines) \cref{eq:taylor-green}.
    The plot shows two velocity slices, the respective other coordinate is held constant at a value of $\pi$.}

    %\includegraphics{paper_abc_flow_velocity}
    %\caption{\label{fig:abc-flow}\textsc{abc}-flow. We show the velocity in $x$-direction for slices through the domain, where all but the mentioned coordinates are held constant at a value of zero.
    %The analytical solution is represented by lines, the numerical results by markers.}
  \end{subfigure}
  \caption{Two dimensional \textsc{cfd} scenarios}
  \label{fig:cdf-results}
\end{figure}

The Arnold-Beltrami-Childress (\textsc{abc}) flow is similar to the Taylor-Green vortex but possesses an analytical solution for three-dimensions as well~\cite{tavelli2016staggered}.
It is defined in the domain \( \left[ -\pi, \pi \right]^3 \) as
\begin{align}
  \label{eq:abc-flow}
  \begin{split}
  \Qrho(\bm{x}, t) &= 1,\\
  \Qv(\bm{x}, t) &= \phantom{-} \exp(-1\mu t)
  \begin{pmatrix}
    \sin(z) + \cos(y)\\
    \sin(x) + \cos(z)\\
    \sin(y) + \cos(x)
  \end{pmatrix}, \\
  \pressure(\bm{x}, t) &= -\exp(-2 \mu t) \, \left(\cos(x)\sin(y) + \sin(x)\cos(z) + \sin(z)\cos(y)\right)
  + C.
  \end{split}
\end{align}
again under the assumption of incompressibility.
The constant $C = \nicefrac{100}{\gamma}$ is chosen as before.
We use a viscosity of $\mu = 0.01$ and analytical boundary conditions.
Our results (\cref{fig:abc-flow}) show a good agreement between the analytical solution and our approximation with an \aderdg{}-scheme of order $3$ with a mesh consisting of $27^3$ cells at time $t = \SI{0.1}{\s}$.

\begin{figure}[tb]
\centering
% or .473 size each, and \\quad inbetween
\begin{minipage}[t]{.473\textwidth}
  \centering
    \includegraphics{paper_abc_flow_velocity}
    \captionof{figure}{\label{fig:abc-flow}Our approximation (markers) of the \textsc{abc}-flow vs.\ analytical solution (lines, \cref{eq:abc-flow}).
    All other axes are held constant at a value of 0. We show every 6th value.}
\end{minipage}\qquad%
\begin{minipage}[t]{.473\textwidth}
  \centering
    \includegraphics[width=\textwidth]{paper_two_bubbles_fv}
    \captionof{figure}{\label{fig:two-bubbles-fv}%
    Colliding bubbles with \muscl{}}
\end{minipage}
\end{figure}

As a final example of standard flow scenarios, we consider the lid-driven cavity flow where the fluid is initially at rest, with $\Qrho = 1$ and $ \pressure(\bm{x}) = \nicefrac{100}{\gamma}$.
We consider a domain of size $\SI{1}{m} \times \SI{1}{\m}$ which is surrounded by no-slip walls.
The flow is driven entirely by the upper wall which has a velocity of $v_x = \SI{1}{\m/\s}$.
The simulation runs for $\SI{10}{\s}$
Again, our results (\cref{fig:cavity-flow}) have an excellent agreement with the reference solution of~\cite{ghia1982high}.
We used an \aderdg{}-method of order $3$ with a mesh of size $27^2$.

With  the constants
\begin{equation}\label{eq:atmosphere-constants}
    \gamma = 1.4 ,\quad \Pr =  0.71 ,\quad R = 287.058 ,\quad p_0 = \SI{10e5}{\Pa}, \quad g = \SI{9.8}{m/s^2},
\end{equation}
all following stratified flow scenarios are described in terms of the potential temperature
\begin{equation}
  \potT = T \left( \frac{p_o}{p} \right)^{R/c_p},
\end{equation}
with reference pressure $p_o$~\cite{muller2010adaptive,giraldo2008study}.
%
We compute the initial background density and pressure by inserting the assumption of a constant background energy in \cref{eq:hydrostatic-balance}.
The background atmosphere is then perturbed.
We set the density and energy at the boundary such that it corresponds to the background atmosphere.
Furthermore, to ensure that the atmosphere stays in hydrostatic balance, we need to impose the viscous heat flux
\begin{equation}
  \label{eq:atmosphere-bc}
  \viscFlux_{\QE} = \kappa \pdv{\overline{T}}{z}.% =
\end{equation}
at the boundary.
In this equation, $\overline{T}(z)$ is the background temperature at position $z$, which can be computed from \cref{eq:hydrostatic-balance,eq:eos}~\cite{giraldo2008study}.

Our first scenario is the colliding bubbles scenario~\cite{muller2010adaptive}.
We use perturbations of the form
\begin{equation}
  \label{eq:bubbles-pertubation}
  \pertubationPotT =
  \begin{cases}
    A & r \leq a, \\
    A \exp \left( - \frac{(r-a)^2}{s^2} \right) & r > a,
    \end{cases}
\end{equation}
where $s$ is the decay rate and $r$  is the radius to the center~\pcref{eq:radius}:
\begin{equation}
  \label{eq:radius}
  r^2 = \Vert \bm{x} - \bm{x_c} \Vert_2
\end{equation}
i.e., $r$ denotes the Euclidian distance between the spatial positions $\bm{x} = (x, z)$ and the center of a bubble $\bm{x_c} = (x_c, z_c)$ -- for three-dimensional scenarios $\bm{x}$ and $\bm{x_c}$ also contain a $y$ coordinate.

We have two bubbles, with constants
\begin{equation}
  \label{eq:bubbles-values}
\begin{alignedat}{6}
  & \text{warm:} \qquad && A = \SI{0.5}{\K}, \quad&& a = \SI{150}{\m}, \quad&& s = \SI{50}{\m}, \quad&& x_c = \SI{500}{\m,} \quad&& z_c = \SI{300}{\m},\\
  & \text{cold:} \qquad && A = \SI{-0.15}{\K}, \quad&& a = \SI{0}{\m}, \quad&& s = \SI{50}{\m}, \quad&& x_c = \SI{560}{\m}, \quad&& z_c = \SI{640}{\m}.
  \end{alignedat}
\end{equation}
Similar to~\cite{muller2010adaptive}, we use a constant viscosity of $\mu = 0.01$ to regularize the solution.
Note that we use a different implementation of viscosity than~\cite{muller2010adaptive}.
Hence, it is difficult to compare the parametrization directly.
We ran this scenario twice: once without \amr{} and a mesh of size $\SI{1000/81}{\m} = \SI{12.35}{\m}$ and 
once with \amr{} with two adaptive refinement levels and parameters $\Trefine = 2.5$ and $\Tdelete = -0.5$.
We specialize the \amr{}-criterion~\pcref{eq:refinement-criterion} to our stratified flows by using the potential temperature.
This resulted in a mesh with cell-size lengths of approx.\ $\SI{111.1}{\m}, \SI{37.04}{\m}, \SI{12.34}{\m}$.
The resulting mesh can be observed in \cref{fig:two-bubbles-ader}.
We observe that the $L_2$ difference between the potential temperature of the \amr{} run, which uses 1953 cells, and the one of the fully refined run with 6561 cells, is only $1.87$.
The relative error is \num{6.69e-6}.
We further emphasize that our \amr{}-criterion accurately tracks the position of the edges of cloud instead of only its position.
This is the main advantage of our gradient-based method in contrast to methods working directly with the value of the solution, as for example~\cite{muller2010adaptive}.
Overall, our result for this benchmark shows an excellent agreement to the previous solutions of~\cite{muller2010adaptive}.

In addition, we simulated the same scenario with our \muscl{} method, using $7^2$ patches with $90^2$ finite volume cells each.
As we use limiting, we do not need any viscosity.
The results of this method (\cref{fig:two-bubbles-fv}) also agree with the reference but contain fewer details.
Note that the numerical dissipativity of the finite volume scheme has a smoothing effect that is similar to the smoothing caused by viscosity.

For our second scenario, the cosine bubble, we use a perturbation of the form
\begin{align}
  \label{eq:cos-pertubation}
  \pertubationPotT &= \begin{cases}
    \nicefrac{A}{2} \left[ 1 + \cos(\pi r) \right] & r \leq a, \\
    0 & r > a,
    \end{cases}
\end{align}
where $A$ denotes the maximal perturbation and $a$ is the size of the bubble.
We use the constants
\begin{equation}\label{eq:cosine-bubble}
  A = \SI{0.5}{\K} \quad a = \SI{250}{\m} \quad x_c = \SI{500}{\m} \quad z_c = \SI{350}{\m}.
\end{equation}
For the three-dimensional bubble, we set $y_c = x_c = \SI{500}{\m}$ and lower the bubble to $z_c = \SI{260}{\m}$.
This corresponds to the parameters used in~\cite{kelly2012continuous}\footnote{%
We found that the parameters presented in the manuscript of~\cite{kelly2012continuous} only agree with the results, if we use the same parameters as for 2D simulations.}.
For the 2D case, we use a constant viscosity of $\mu = 0.01$ and an \aderdg{}-method of order 5 with two levels of dynamic \amr{}, resulting again in cell sizes of roughly $\SI{111.1}{\m}, \SI{37.04}{\m}, \SI{12.34}{\m}$.
We use slightly different \amr{} parameters of $\Trefine = 1.5$ and $\Tdelete = -0.5$ and let the simulation run for \SI{600}{\s}.
Note that, as seen in \cref{fig:cosine-2d}, our \amr{}-criterion tracks the wavefront of the cloud accurately.
This result shows an excellent agreement to the ones achieved in~\cite{giraldo2008study,muller2010adaptive}.

For the 3D case, we use an \aderdg{}-scheme of order 3 with a static mesh with cell sizes of \SI{40}{\m}. and a shorter simulation duration of \SI{400}{\s}.
Due to the relatively coarse resolution and the hence increased aliasing errors, we need to increase the viscosity to $\mu = 0.02$.
This corresponds to a larger amount of smoothing.
Our results (\cref{fig:cosine-3d}) capture the overall dynamics of the scenario well, similar to the reference solution of~\cite{kelly2012continuous}.

\begin{figure}[tb]
  \centering
  \includegraphics[width=1.0\textwidth]{paper_two_bubbles}
  \caption{\label{fig:two-bubbles-ader}Colliding Bubbles \aderdg{}}
\end{figure}

\begin{figure}[tb]
  \centering
  \begin{subfigure}[t]{0.5\textwidth}
    \centering
    \includegraphics[width=\textwidth]{paper_cosine_bubble}
    \caption{\label{fig:cosine-2d}2D Cosine Bubble}
  \end{subfigure}~%
  \begin{subfigure}[t]{0.5\textwidth}
    \centering
    \includegraphics[width=\textwidth]{paper_cosine_bubble_3d}
    \caption{3D Cosine Bubble}
  \end{subfigure}
  \caption{\label{fig:cosine-3d}Cosine Bubble Scenarios}
  \label{fig:cosine-bubbles-results}
\end{figure}


\section{Conclusion}
We presented an implementation of an \aderdg{}-method and a \muscl{}-scheme with \amr{} for the Navier-Stokes equations, based on the \exahypeengine.
Our implementation is capable of simulating different scenarios: \todo{LK: Flesh this out a tiny bit, mention properties of \amr{} criterion}
We evaluated our method for standard \textsc{cfd} scenarios and achieved competitive results for all used scenarios.
This is true for both two-dimensional scenarios (Taylor-Green vortex and lid-driven cavity) and for the three-dimensional \textsc{abc}-flow.

Furthermore, our method allows us to simulate flows in hydrostatic equilibrium correctly, as our results for the cosine and colliding bubble scenarios showed.
We showed that our \amr{}-criterion is able to vastly reduce the number of grid cells while preserving the quality of the results.

\subsection*{Acknowledgements}
This work was funded by the European Union’s Horizon 2020 Research and Innovation Programme under grant agreements 
No~671698 (project ExaHyPE, \url{www.exahype.eu}) and 
No~823844 (ChEESE centre of excellence, \url{www.cheese-coe.eu}).
Computing resources were provided by the Leibniz Supercomputing Centre (project pr83no).
Special thanks go to Dominic E.\ Charrier for his support with the implementation in the \exahypeengine{}.

\printbibliography{}
\end{document}

\documentclass[runningheads]{llncs}
\title{Cloud Supercomputing}
\author{Lukas Krenz\inst{1} \and{} Leonhard Rannabauer\inst{1} \and{} Michael Bader\inst{1}}
\institute{Technical University of Munich} % todo emails and stuff
\usepackage[utf8]{inputenc}
%\usepackage[T1]{fontenc}
\usepackage[american]{babel}
\usepackage[autostyle, english = american]{csquotes}
\usepackage[%
  backend=biber,
  url=false,
  doi=false,
  % style=alphabetic,
  % backref=true,
  % hyperref=true,
  maxnames=3,
  % minnames=3,
  % maxbibnames=99,
  firstinits=true,
  % uniquename=init
  ]{biblatex}
\DefineBibliographyStrings{english}{%
  backrefpage = {see p.},
  backrefpages = {see pp.}
}
\addbibresource{../bibliography.bib}

\usepackage{caption}
\usepackage{xparse} % for NewDocumentCommand
\usepackage{etoolbox} % for notblank (brackets only when argument)
\usepackage{xstring} % for \IfSubStr
\usepackage{xpatch}
\usepackage{xcolor}
\usepackage{amsmath}
\usepackage{amsfonts}
\usepackage{amssymb}
\usepackage{mathtools} % for \mathclap
\usepackage{nicefrac}
\usepackage{physics} % for derivatives
\usepackage{varioref}
\usepackage{nicefrac}
\usepackage{physics} % for derivatives
\usepackage[separate-uncertainty]{siunitx}
\usepackage{hyperref}
\usepackage[noabbrev]{cleveref}
\newcommand{\creflastconjunction}{, and\nobreakspace} % use Oxford comma
\usepackage{todonotes}
\usepackage{multimedia}
\graphicspath{{../figures/}}
\let\boundary\undefined%
\crefformat{equation}{\eqA{}eq.~\eqB #2#1#3)}
% TODO: Fix for more than two equations!
\crefmultiformat{equation}{eqs.~\eqMultiA#2#1#3\eqMultiB}%
{ and \eqMultiA#2#1#3\eqMultiB}{, \eqMultiA#2#1#3\eqMultiB}{ and~\eqMultiA#2#1#3\eqMultiB}
\newcommand{\eqA}{}
\newcommand{\eqB}{(}
\newcommand{\eqMultiA}{(}
\newcommand{\eqMultiB}{)}
\DeclareRobustCommand{\pcrefSingle}[1]{%
\begingroup%
  \renewcommand{\eqA}{(}\renewcommand{\eqB}{}%
\cref{#1}%
\endgroup%
}
\DeclareRobustCommand{\pcrefMulti}[1]{%
\begingroup%
    \renewcommand{\eqMultiA}{}\renewcommand{\eqMultiB}{}%
    (\cref{#1})%
\endgroup%
}
\DeclareRobustCommand{\pcref}[1]{%
\IfSubStr{#1}{,}{\pcrefMulti{#1}}{\pcrefSingle{#1}}%
}

\usepackage{bm}


% Commands/Macros
% Names of methods or so
\newcommand{\muscl}{\textsc{muscl}-Hancock}
\newcommand{\dg}{\textsc{dg}}
\newcommand{\ader}{\textsc{ader}}
\newcommand{\aderdg}{\textsc{ader-dg}}
\newcommand{\amr}{\textsc{amr}}
\newcommand{\pde}{\textsc{pde}}
\newcommand{\cpp}{C\texttt{++}}
\newcommand{\tbb}{\textsc{tbb}}
\newcommand{\mpi}{\textsc{mpi}}

\newcommand{\softwareName}[1]{\textit{#1}}
\newcommand{\exahype}{\softwareName{ExaHyPE}}
\newcommand{\peano}{\softwareName{Peano}}
\newcommand{\className}[1]{\texttt{#1}}
\newcommand{\funName}[1]{\textsf{#1}}
\newcommand{\fileName}[1]{\textbf{#1}}
\newcommand{\varName}[1]{\enquote{#1}}



% Variables and other equation stuff
\newcommand{\Q}{\bm{Q}}
\newcommand{\gradQ}{\gradient{\Q}}
\newcommand{\Qrho}{\rho}
\newcommand{\Qj}{\rho \bm{v}}
\newcommand{\Qv}{\bm{v}}
\newcommand{\QE}{\rho E}
\newcommand{\QZZ}{Z} % TODO: Name?
\newcommand{\QZ}{\rho \QZZ}
\newcommand{\potT}{\theta}
\newcommand{\backgroundPotT}{\overline{\theta}}
\newcommand{\pertubationPotT}{\theta'}
\newcommand{\stressT}{\bm{\sigma}}
\newcommand{\pressure}{p}
\newcommand{\maxConvEigen}[1][]{
  \vert%
  \lambda_c^{\text{max}}
  \notblank{#1}{\left(#1\right)}{}
  \vert%
}
\newcommand{\maxViscEigen}[1][]{
  \vert%
  \lambda_v^{\text{max}}
  \notblank{#1}{\left(#1\right)}{}
  \vert%
}
\newcommand{\Riemann}{\operatorname{Riemann}}

% Domain
\newcommand{\domain}{\Omega}
\newcommand{\boundary}{\partial \domain}
\newcommand{\broken}{\domain}

% Cells
\newcommand{\cell}[1][]{C_{#1}}
\newcommand{\refCell}[1][]{\hat{C}_{#1}}
\newcommand{\cellb}{\partial{} \cell}
\newcommand{\refCellb}{\partial{} \refCell}

% Basis functions
\newcommand{\lagrange}[1][i]{\varphi_{#1}}
\newcommand{\lagrangeRef}[1][i]{\hat{\varphi}_{#1}}
\NewDocumentCommand{\sbasis}{ O{} O{} }{\phi^{#1}_{#2}}
\NewDocumentCommand{\sbasisRef}{ O{} O{} }{\hat{\phi}^{#1}_{#2}} %{\hat{\sbasis[#1][#2]}}
\NewDocumentCommand{\stbasis}{ O{} O{} }{ \psi^{#1}_{#2}}
\NewDocumentCommand{\stbasisRef}{ O{} O{} }{\hat{\psi}^{#1}_{#2}} %{\hat{\stbasis[#1][#2]}}
%\newcommand{\stestfunction}[1]{\sbasis[#1]}
\NewDocumentCommand{\stestfunction}{ O{} O{} }{\sbasis[#1][#2]}
\NewDocumentCommand{\sttestfunction}{ O{} O{} }{\stbasis[#1][#2]}

% Bold if no index is supplied.
\newcommand{\bmempty}[2]{
\notblank{#2}{#1}{\bm{#1}}
}

% Space-time dofs
\NewDocumentCommand{\stpredictor}{ O{} O{} }{\overline{\bmempty{q}{#2}}^{#1}_{#2}}
\NewDocumentCommand{\stpredictorCoeff}{ O{} O{} }{\underline{\overline{\bmempty{q}{#2}}}^{#1}_{#2}}

\NewDocumentCommand{\stflux}{ O{} O{} }{\overline{\bmempty{F}{#2}}^{#1}_{#2}}
\NewDocumentCommand{\stfluxCoeff}{ O{} O{} }{\underline{\overline{\bmempty{F}{#2}}}^{#1}_{#2}}

% \NewDocumentCommand{\stsol}{ O{} O{} }{\overline{u}^{#1}_{#2}}
% \NewDocumentCommand{\stsol}{ O{} O{} }{\underline{\overline{u}}^{#1}_{#2}}

\NewDocumentCommand{\stsource}{ O{} O{} }{\overline{\bmempty{S}{#2}}^{#1}_{#2}}
\NewDocumentCommand{\stsourceCoeff}{ O{} O{} }{\underline{\overline{\bmempty{S}{#2}}}^{#1}_{#2}}

% Space dofs
\NewDocumentCommand{\spredictor}{ O{} O{} }{\bmempty{q}{#2}^{#1}_{#2}}
\NewDocumentCommand{\spredictorCoeff}{ O{} O{} }{\underline{q}^{#1}_{#2}}

\NewDocumentCommand{\ssource}{ O{} O{} }{\bmempty{S}{#2}^{#1}_{#2}}
\NewDocumentCommand{\ssourceCoeff}{ O{} O{} }{\underline{\bmempty{S}{#2}}^{#1}_{#2}}

\NewDocumentCommand{\ssol}{ O{} O{} }{\bmempty{u}{#2}^{#1}_{#2}}
\NewDocumentCommand{\ssolCoeff}{ O{} O{} }{\underline{\bmempty{u}{#2}}^{#1}_{#2}}

\NewDocumentCommand{\sflux}{ O{} O{} }{\bmempty{F}{#2}^{#1}_{#2}}
\NewDocumentCommand{\sfluxCoeff}{ O{} O{} }{\underline{\bmempty{F}{#2}}^{#1}_{#2}}

% Equation parts
\newcommand{\flux}{\bm{F}}
\newcommand{\viscFlux}{\flux^{v}}
\newcommand{\hyperFlux}{\flux^{h}}
%\newcommand{\source}{\bm{S}}
\newcommand{\source}[1][]{
  \notblank{#1}{
S_{#1}
}{
\bm{S}
}
}

% Integrals
\newcommand{\curTime}{t^k}
\newcommand{\nextTime}{t^{k+1}}
\newcommand{\intdt}[1]{\int_{\curTime}^{\nextTime} #1 \dd{t}}
\newcommand{\intdcell}[1]{\int_{\cell} #1 \dd{\bm{x}}}
\newcommand{\intdrefcell}[1]{\int_{\refCell} #1 \dd{\hat{\bm{x}}}}
\newcommand{\intdcellb}[1]{\int_{\cellb} #1 \dd{S}} % TODO: define boundary of cell
\newcommand{\intdrefcellb}[1]{\int_{\refCellb} #1 \hat{\dd{S}}} % TODO: define boundary of cell
\newcommand{\intddomain}[1]{\int_{\domain} #1 \dd{\bm{x}}}

% Yet to be sorted
\newcommand{\quadWeight}[1][i]{w_{#1}}
\newcommand{\quadNode}[1][i]{\hat{x}_{#1}}
\newcommand{\quadNodeNd}[1][i]{\hat{\bm{x}}_{#1}}
\newcommand{\sbasisSize}{(N+1)^\dimensions}
\newcommand{\stbasisSize}{(N+1)^{\dimensions+1}}
\newcommand{\normal}{\bm{n}}

\newcommand{\pleft}{\bm{L}}
\newcommand{\prightsol}{\bm{r}}
\newcommand{\prightpred}{\bm{w}}

\newcommand{\reactionTimescale}{\tau}
\newcommand{\reactionTemperature}{T_{\text{ign}}}

\newcommand{\tv}{\operatorname{TV}}

\newcommand{\NVar}{N^{\text{var}}}
\newcommand{\dimensions}{d}
\newcommand{\mapping}{\mathcal{M}}
\newcommand{\volume}{V}
\newcommand{\cellCenter}{\operatorname{cell-center}}

% MUSCL
\newcommand{\cellAvg}[1][i,j]{\bm{U}_{#1}}
\newcommand{\extrapolatedCellAvg}[3][i,j]{\cellAvg[#1]^{#3 #2} = \cellAvg #3 \frac{1}{2} \slope{#2}}%
\newcommand{\evolvedCellAvg}[2][i,j]{\hat{\bm{U}}_{#1}^{#2}}
\newcommand{\sign}{\operatorname{sign}}
\newcommand{\minmod}{\operatorname{minmod}}
\newcommand{\slope}[2][i,j]{\bm{s}^{#2}_{#1}}
\newcommand{\gradCellAvg}[1][i,j]{\gradient{\cellAvg[#1]}}
\newcommand{\fluxX}{\flux_x}
\newcommand{\fluxY}{\flux_y}%


\begin{document}
\maketitlet 
\begin{abstract}
  In the context of simulation of convective proceess for numerical weather prediction, we present an application of the compressible Navier Stokes equations on cloud formation processes.
  While established methods have to use tailored numerics towards the specific simulation, our generic scheme shows competitive results for both classical cfd and hydrostatic scenarios.
  Our scheme exploits an underlying parallelized implementation of the \aderdg{} method with dynamic adaptive mesh refinement. 
  We improve our method by a PDE independent general refinement criterion, based on the local total variation of the numerical solution.
  All together our method can be seen as a perfect candidate for large scale cloud simulation runs on current and future super computers.
\begin{itemize}
\item Implementation of compressible Navier Stokes with a high-order \aderdg{}-method and a low order finite volume \muscl{}-scheme.
\item Implementation in \exahypeengine{}, a scalable \pde{} framework.
\item Outlier and total variation based adaptive mesh refinement that reduced number of grid cells drastically and preserves solution quality.
\item Simulation of challenging flows over hydrostatic background atmosphere.
\item Competitive results for both classical cfd scenarios and hydrostatic scenarios in two and three dimensions.
\end{itemize}

  
The abstract should briefly summarize the contents of the paper in
150--250 words.

\keywords{\aderdg{}  \and Navier-Stokes \and Another keyword.}
\end{abstract}
\section{Introduction}
In this paper we address the resolution of convective processes in the context of numerical weather prediction (\textsc{nwp}), on modern super computer systems.
The simulation of these process, with suitable physical models and on large scale clusters is expected to play an important role in future \textsc{nwp}~\cite{bauer2015quiet}.
%% Numerical weather prediction (\textsc{nwp}) contains processes of different spatio-temporal scales.
%% An important part of future \textsc{nwp} models will be the the resolution of convective processes~\cite{bauer2015quiet}.
%% This part requires both suitable physical models and effective computational realisations. 
We focus on the simulation of cloud formation processes with simple benchmark scenarios~\cite{giraldo2008study}.
These relatively small scale effects are well approximated with the compressible Navier-Stokes equations.
The general approach is to use a space-discretized by higher order Galerkin method, as seen for example in~\cite{giraldo2008study,muller2010adaptive,muller2018strong}.
%%They are often space-discretized by higher order Galerkin methods, as seen for example in~\cite{giraldo2008study,muller2010adaptive,muller2018strong}.
Even though all of these implementations use a discretization with a (theoretical) unlimited space-order, they use Runge-Kutta time integrators or semi-implicit methods.
It is well known that Runge-Kutta methods become more ineffecient for very high orders.
Furthermore, they and the implicit methods are more communication-intensive and thus require a larger overhead and are more complicated to scale.

We follow a different approach in this paper and use the \aderdg{} method of~\cite{dumbser2008unified} to simulate various computational fluid dynamics (\textsc{cfd}) scenarios with a focus on flows with stratified flows.
\todo{LK: Ist hier stratified flows korrekt?}
\todo{MB: viscosity ist unmotiviert ist.
    Lieber noch so 10--15 Zeilen drüber schreiben.}
Due to the viscous components of the Navier-Stokes equations, it is not straightforward to apply the \dg{} method.
For high-order \dg{} methods, viscosity not only models the physical reality of the problem but also helps us by smoothing out oscillations and discontinuities, thus stabilizing the simulation.
To include viscosity, we use the numerical flux for the compressible Navier-Stokes equations of \citeauthor{gassner2008discontinuous}~\cite{gassner2008discontinuous}.
This flux has already been applied to the \aderdg{} method in~\cite{dumbser2010arbitrary}.
In contrast to this paper, we focus on the simulation of complex flows with a gravitational source term and a realistic background atmosphere.
Additionally, we use adaptive mesh refinement (\textsc{amr}) to increase the spatial resolution in areas of interest.
This has been shown to work well for the simulation of cloud dynamics~\cite{muller2010adaptive}.
%\todo{Einordnung, Kontext zu giraldo, etc$\ldots$}
%Insert references to Giraldo SE/DG bubbles~\cite{giraldo2008study}, 3D bubbles~\cite{giraldo2013implicit,kelly2012continuous}, bubble amr (müller~\cite{muller2010adaptive}).

We base our work on the \exahypeengine{} (\url{www.exahype.eu}), which is a framework that can solve arbitrary hyperbolic partial differential equations.
A user of the engine is provided with a simple code interface which mirrors the parts required to formulate a well posed Cauchy problem for a system of hyperbolic \pde{}s of first order.
The underlying \aderdg{} method\todo{\muscl{}?}, parallelization techniques and dynamic adaptive mesh refinement are available for simulations while the implementations are left as a black box to the user.
An introduction to the communication-avoiding implementation of the whole numerical scheme can be found in the recent work~\cite{charrier2018stop} by \citeauthor{charrier2018stop}.

We make the following contributions\todo{Feedback von MB: Besser hervorheben, warum diese Sachen interessant/neu sind.
Dumbser macht in seinen NS-DG papern keine komplexen Source terme, keine hydrostatic flows, etc.
}:
\begin{itemize}%
\item We extend the \exahypeengine{} to allow viscous terms following the aforementioned numerical scheme.
\item We use this to provide an implementation of the compressible Navier-Stokes equation.
  In addition, we tailor the equation set to stratified flows with gravitational source term.
  We want to emphasize here that we use a standard formulation of the Navier-Stokes equations as seen in the field of computational fluid mechanics and only use small modifcations of the governing equations, in contrast to a equation set that is tailored exactly to the application area.
\item We present a general \textsc{amr}-criterion that is based on the detection of outlier cells w.r.t.\ their total variation.
  Furthermore, we show how to utilize this criterion for stratified flows.
\item We evaluate our implementation with standard \textsc{cfd} scenarios and atmospheric flows and inspect the effectiveness of our proposed \amr{}-criterion.
  We thus inspect, whether our proposed general implementation can achieve results that are competitive with the state-of-the-art models that rely on heavily specified equations and numerics.
\end{itemize}

\section{Equation Set}
\newcommand{\diffCoeff}{\varepsilon}%
\newcommand{\hyperFluxDef}{
  \begin{pmatrix}
    \Qj \\
    \Qv  \otimes \Qj + \bm{I} \pressure  \\
    \Qv \cdot (\bm{I} \QE + \bm{I} \pressure)
  \end{pmatrix}
}%
\newcommand{\viscFluxDef}{
  \begin{pmatrix}
    0\\
     \stressT (\Q, \gradQ)  \\
     \Qv \cdot \stressT (\Q, \gradQ) - \kappa \gradient{T}
   \end{pmatrix}
}%

\begin{equation}
  \label{eq:ns-short}
 \pdv{}{t} \bm{Q} + \divergence{\bm{F}(\bm{Q}, \gradient{\bm{Q}})} = \bm{S}(\bm{Q}) 
\end{equation}
The compressible Navier-Stokes equations in the conservative form are defined as% \pcref{eq:conservation-law-gradient}
\begin{equation}
 \label{eq:equation-set} 
%  \begin{array}{l}
%  \text{mass cons.} \\
%  \text{momentum cons.} \\
%  \text{energy cons.} \\
%  \text{cont.\ gas} 
% \end{array}
% :
\quad
  \pdv{}{t}
  \underbrace{
  \begin{pmatrix}
    \Qrho\\
    \Qj\\
    \QE
    \end{pmatrix}}_{\Q}
  +
  \divergence{
  \underbrace{
  \left(
   \underbrace{\hyperFluxDef}_{\hyperFlux(\Q)}
+
\underbrace{\viscFluxDef}_{\viscFlux(\Q, \gradQ)}
  \right)}_{\flux(\Q, \gradQ)}}
 =
  \underbrace{
  \begin{pmatrix}
    \source[\Qrho\phantom{\Qrho}]\\
    \source[\Qj]\\
    \source[\QE]
    \end{pmatrix}}_{\source(\Q, \bm{x}, t)}
\end{equation}
with vector of conserved quantities $\Q$, flux $\flux(\Q, \gradQ)$ and source $\source(\Q)$.
Note that the flux can be split into a hyperbolic part $\hyperFlux(\Q)$,
which is identical to the flux of the Euler equations,
and a viscous part $\viscFlux(\Q, \gradQ)$.
The conserved quantities
\(\Q\)
% \begin{equation} 
%   \label{eq:conserved-variables}
%  \Q = \left( \Qrho, \Qj, \QE, \QZ \right),
% \end{equation}
are the density $\Qrho$, the two or three-dimensional momentum $\Qj$ and the energy density $\QE$.
The rows of \cref{eq:equation-set} are the conservation of mass, the conservation of momentum and the conservation of energy.

The pressure $\pressure$ is given by the equation of state of an ideal gas
\begin{equation}
  \label{eq:eos}
  \pressure = (\gamma - 1) \left(\QE - \frac{1}{2} \left(\Qv \cdot \Qj \right) - gz \right).
\end{equation}
The term $gz$ is the geopotential height with the gravity of Earth $g$~\cite{giraldo2008study}.
The temperature $T$ relates to the pressure by the thermal equation of state
\begin{equation}
  \label{eq:temperature}
  \pressure = \Qrho R T,
\end{equation}
where $R$ is the specific gas constant of a fluid.

We model the diffusivity by the stress tensor
\begin{equation}
  \label{eq:stress-tensor}
  \stressT(\Q, \gradQ) =
  \mu
  \left(
  \left(\nicefrac{2}{3} \divergence{\Qv} \right) -
  \left( \gradient{\Qv} + \gradient{\Qv}^\intercal \right)
  \right),
\end{equation}
with constant viscosity $\mu$.
The heat diffusion is governed by the coefficient
\begin{equation}
  \label{eq:heat-conduction-coeff}
  \kappa = \frac{\mu \gamma}{\Pr} \frac{1}{\gamma - 1} R = \frac{\mu c_p}{\Pr},
\end{equation}
where the ratio of specific heats $\gamma$, the heat capacity at constant pressure $c_p$ and the Prandtl number $\Pr$ depend on the fluid.

Many realistic atmospheric flows can be described by a perturbation over a background state that is in hydrostatic equilibrium\todo{Mention source}
\newcommand{\backgroundPressure}{\overline{\pressure}}
\newcommand{\backgroundRho}{\overline{\Qrho}}
\begin{equation}
  \label{eq:hydrostatic-balance}
  \pdv{}{z} \backgroundPressure{\left (z \right )} = -g \backgroundRho(z),
\end{equation}
i.e.\ a state, where the pressure gradient is exactly in balance with the gravitational source term.
The momentum equation is dominated by the background flow in this case.
Because this can lead to numerical instabilities, problems of this kind are challenging and require some care.
To lessen the impact of this, we split the pressure $\pressure = \backgroundPressure + \pressure'$ into a sum of the background pressure $\backgroundPressure(z)$ and perturbation $\pressure'(\bm{x}, t)$.
We split the density $\Qrho = \backgroundRho + \Qrho'$ in the same manner and arrive at
\begin{equation}
  \label{eq:momentum-equation-split}
  \pdv{\Qj}{t}+ \divergence{ \left(
    \Qv \otimes \Qj + \bm{I} \pressure'
    \right)
  } + \viscFlux_\Qj
  =
  -g \bm{k} \Qrho'.
\end{equation}
Note that a similar and more complex splitting is performed in~\cite{muller2010adaptive,giraldo2008study}.
In contrast to this, we use the true compressible Navier-Stokes equations with minimal modifications.
\todo{Mention differences in splitting methods}
\section{Numerics}

\todo{LR: Hier würde ich noch ein paar einleitende Worte hinzufügen}
\todo{LK: CFL? Riemannsolver (no gradients)}
We use an \aderdg{}-scheme and a \muscl{} finite volume method.
Both can be considered as instances of the more general \textsc{PnPm} schemes of~\cite{dumbser2008unified}.
We use a Rusanov-style flux that is adapted to \pde{}s with viscous terms~\cite{gassner2008discontinuous,fambri2017space}.
The finite volume scheme is stabilized with the van Albada limiter\todo{LK: Which one is best? Cite vanalbada}.

The user can state dynamic \amr{} rules by supplying custom criteria that are evaluated point-wise.
We use an element local error estimate based on the total variation of the numerical solution. 
Let $\bm{f}(\bm{x}): \mathbb{R}^{N_\text{vars}} \to \mathbb{R}$ be a sufficiently smooth function that maps the discrete solution at a point $\bm{x}$ to an arbitrary indicator variable.
The total variation (\textsc{tv}) of this function is defined by
\begin{equation}
  \label{eq:tv}
  \tv \left[ f(\bm{x}) \right] =
  \Vert
\intdcell{ \vert \gradient{f \left( \bm{x} \right)} \vert }
\Vert_1
\end{equation}
for each cell.
The operator $\Vert \cdot \Vert_1$ denotes the discrete $l_1$ norm in this equation.
We compute the integral efficiently with Gaussian quadrature over the collocated quadrature points.
\todo{LK: If using limiting, add sentence about tv computation for fv}
\newcommand{\mean}{\mu}%
\newcommand{\std}{\sigma}%
\newcommand{\variance}{\std^2}%
\newcommand{\Trefine}{T_\text{refine}}%
\newcommand{\Tdelete}{T_\text{erase}}%
How can we decide whether a cell is important or not?
To resolve this conundrum, we compute the mean and the population standard deviation of the total variation of all cells.
It is important to note that we use the method of~\cite{chan1982updating} to compute the modes in a parallel and numerical stable manner.
A cell is then considered to contain significant information if its deviates from the mean more than a given threshold.
This criterion can be described formally by
\begin{equation}
  \label{eq:refinement-criterion}
  \operatorname{evaluate-refinement}(\Q, \mu, \sigma) =
  \begin{cases}
    \text{refine} & \text{if } \tv(\Q) \geq \mu + \Trefine \sigma, \\
    \text{erase} & \text{if } \tv(\Q) < \mu + \Tdelete \sigma, \\
    \text{keep} & \text{otherwise}.
    \end{cases}
\end{equation}
The parameters $\Trefine > \Tdelete$ can be chosen freely.
Chebyshev's inequality
\begin{equation}
  \label{eq:chebychev}
  \mathbb{P}(\vert X - \mu \vert \geq c \sigma) \leq \frac{1}{c^2},
\end{equation}
with probability $\mathbb{P}$ guarantees that we neither mark all cells for refinement nor for erasement.
\todo{LR: Was gibt es an compressible NS mit dynamic AMR, wie unterscheiden wir uns im Refinement Kriterium?}
This inequality holds for arbitrary distributions under the weak assumption that they have a finite mean $\mu$ and a finite standard deviation $\sigma$~\cite{wasserman2004all}.
Note that subcells are coarsened only if all subcells belonging to the coarse cell are marked for erasion.

We use a \muscl{} based finite volume limiting as proposed in~\cite{dumbser2016simple}.
Briefly, we first compute a candidate solution with our \aderdg{}-scheme and check whether it is valid.
A solution is considered to be invalid if it is a non-finite floating point number or if it is unphysical, i.e.\ if the pressure or density are not positive.
\todo{LK: Mention dmp if using, mention tv-indicator for limiting if using}
In case of an invalid solution, we recompute the timestep with a our \muscl{} finite volume method.\todo{Source for the fv method}

\section{Results}
We use a diverse mix of benchmarking scenarios that consist of standard \textsc{cfd} scenarios and atmospheric flows.

\todo{LK: Constants of fluids?}
A simple scenario is the Taylor-Green vortex.
It has an analytical solution in the incompressible limit which is given by
\begin{align}
  \label{eq:taylor-green}
  \begin{split}
  \Qrho(\bm{x}, t) &= 1,\\
  \Qv(\bm{x}, t) &= \exp(-2 \mu t)
  \begin{pmatrix}
    \phantom{-}\sin(x) \cos(y) \\
- \cos(x) \sin(y) 
    \end{pmatrix}, \\
  \pressure(\bm{x}, t) &= \exp(-4 \mu t) \, \nicefrac{1}{4} \left( \cos(2x) + \cos(2y) \right) + C.
  \end{split}
\end{align}
The constant $C = \nicefrac{100}{\gamma}$ governs the speed of sound and thus the Mach number~\cite{dumbser2016high}, the viscosity is set to $\mu = 0.1$.
We use a domain of size $(\SI{2 \pi}{\m} \times \SI{2 \pi}{\m})$ and impose the analytical solution at the boundary\todo{tEnd, mach=0.1}.

We use the Arnold-Beltrami-Childress (\textsc{abc}) flow as a simple three-dimensional flow~\cite{tavelli2016staggered}.
It also has an exact solution in the domain of \( \left[ -\SI{\pi}{\m}, \SI{\pi}{\m} \right]^3 \) under the assumption of incompressibility
\begin{align}
  \label{eq:abc-flow}
  \begin{split}
  \Qrho(\bm{x}, t) &= 1,\\
  \Qv(\bm{x}, t) &= \phantom{-} \exp(-1\mu t)
  \begin{pmatrix}
    \sin(z) + \cos(y)\\
    \sin(x) + \cos(z)\\
    \sin(y) + \cos(x)
  \end{pmatrix}, \\
  \pressure(\bm{x}, t) &= -\exp(-2 \mu t) \, \left(\cos(x)\sin(y) + \sin(x)\cos(z) + \sin(z)\cos(y)\right)
  + C.
  \end{split}
\end{align}
The constant $C = \nicefrac{100}{\gamma}$ is chosen as before.
We use a viscosity of $\mu = 0.01$ and analytical boundary conditions\todo{tEnd}.

As a final example of standard flow scenarios, we consider the lid-driven cavity flow where the fluid is initially at rest, with $\Qrho = 1$ and $ \pressure(\bm{x}) = \nicefrac{100}{\gamma}$.
We consider a domain of size $\SI{1}{m} \times \SI{1}{\m}$ which is surrounded by no-slip walls.
The flow is driven entirely by the upper wall which has a velocity of $v_x = \SI{1}{\m/\s}$.
The simulation runs for $\SI{10}{\s}$

Our atmospheric flow scenarios are described in terms of the potential energy
\begin{equation}
  \potT = T \left( \frac{p_0}{p} \right)^{R/c_p},
\end{equation}
with reference pressure $p_o$~\cite{muller2010adaptive,giraldo2008study}.
We use the constants
\begin{equation}\label{eq:atmosphere-constants}
    \gamma = 1.4 ,\qquad \Pr =  0.71 ,\qquad R = 287.058 ,\qquad p_0 = \SI{10e5}{\Pa}, \qquad g = \SI{9.8}{m/s^2},
\end{equation}
for all remaining scenarios.

We compute the initial background density and pressure by inserting the assumption of a constant background energy in \cref{eq:hydrostatic-balance}.
The background atmosphere is then perturbed.
\todo{LK: Mention bcs briefly?}
Let
\begin{equation}
  \label{eq:radius}
  r^2 = (x - x_0)^2 + (z - z_0)^2
\end{equation}
denote the difference between spatial positions $x,z$ and the center of a bubble $x_c, z_c$\todo{3D}.
For our first scenario, the cosine bubble, we use a pertubation of the form
% \begin{equation}
%   \label{eq:cos-pertubation-alt}
%   \pertubationPotT = 
%     \nicefrac{A}{2} \left( 1 + \cos(\pi r) \right) \left[ r \leq a \right]
% \end{equation}
\begin{align}
  \label{eq:cos-pertubation}
  \pertubationPotT &= \begin{cases}
    \nicefrac{A}{2} \left[ 1 + \cos(\pi r) \right] & r \leq a, \\
    0 & r > a,
    \end{cases}
\end{align}
where $A$ denotes the maximal perturbation and $a$ is the size of the bubble.
We use the constants\todo{3d bubble is higher}
\begin{equation}\label{eq:cosine-bubble}
  A = \SI{0.5}{\K} \quad a = \SI{250}{\m} \quad x_c = \SI{500}{\m} \quad z_c = \SI{350}{\m}.
\end{equation}
We use a constant viscosity of $\mu = 0.1$\todo{tend = 400 for 3d and 600 for 2d}.
For our second scenario, the colliding bubbles, we use perturbations of the form
\begin{equation}
  \label{eq:bubbles-pertubation}
  \pertubationPotT =
  \begin{cases}
    A & r \leq a, \\
    A \exp \left( - \frac{(r-a)^2}{s^2} \right) & r > a,
    \end{cases}
\end{equation}
where $s$ is the decay rate and $r$  is the radius to the center~\pcref{eq:radius}.
We have two bubbles, with constants
\begin{equation}
  \label{eq:bubbles-values}
\begin{alignedat}{6}
  & \text{warm:} \qquad && A = \SI{0.5}{\K}, \quad&& a = \SI{150}{\m}, \quad&& s = \SI{50}{\m}, \quad&& x_c = \SI{500}{\m,} \quad&& z_c = \SI{300}{\m},\\
  & \text{cold:} \qquad && A = \SI{-0.15}{\K}, \quad&& a = \SI{0}{\m}, \quad&& s = \SI{50}{\m}, \quad&& x_c = \SI{560}{\m}, \quad&& z_c = \SI{640}{\m}.
  \end{alignedat}
\end{equation}
Similar to~\cite{muller2010adaptive} and to the previous scenario, we use a constant viscosity of $\mu = 0.1$ to regularize the solution.
To ensure that the atmosphere stays in hydrostatic balance~\cite{giraldo2008study}, we need to impose the viscous heat flux
\begin{equation}
  \label{eq:atmosphere-bc}
  \viscFlux_{\QE} = \kappa \pdv{\overline{T}}{z}.% =
%\kappa \frac{R \overline{\theta} \left(\frac{\overline{p}{\left (z \right )}}{p_{0}}\right)^{\frac{R}{c_{p}}} \frac{d}{d z} \overline{p}{\left (z \right )}}{c_{p} \overline{p}{\left (z \right )}}.
\end{equation}
where $\overline{T}(z)$ is the background temperature at position $z$ which can be computed from \cref{eq:hydrostatic-balance,eq:eos}.
We furthermore set the density and energy at the boundary such that it corresponds to the background atmosphere.

% We start with a coarse mesh that is then subdivided up to two times in an adaptive manner.
% For this we use the criterion of \cref{eq:refinement-criterion} with parameters xxx and xxx.
% We compare this with a simulation obtained by a fully refined grid.

Show following results:
\begin{itemize}
\item Simple cfd: Taylor Green, ABC, Lid Driven Cavity
\item Two Bubbles with \amr{} (2 levels if possible) and no \amr{} (comparison)
\item Two Bubbles (Euler eqs, FV)  
\item Two Bubbles with limiting instead of visc.\ if working  
\item Cosine bubble in 2D and possibly 3D, if working
\end{itemize}

\section{Conclusion}
We presented our implementation of a \muscl{}-limited \aderdg{}-scheme with \amr{} for the Navier-Stokes equations.
Our implementation is capable of simulating different scenarios:
\begin{itemize}
  % Convergence test?
\item We evaluated our method for standard \textsc{cfd} scenarios and achieved competitive results for all used scenarios.
  This is true for both two-dimensional scenarios (Taylor-Green vortex and lid-driven cavity) and for the three-dimensional \textsc{abc}-flow.
\item Furthermore, our method allows us to simulate flows in hydrostatic equilibrium correctly as our results for the cosine and colliding bubble scenarios showed.
\item We showed that our \amr{}-criterion is able to vastly reduce the number of grid cells while preserving the quality of the results.
\end{itemize}
\todo{LK: Flesh this out a tiny bit, mention properties of \amr{} criterion}

\subsection*{Acknowledgements}
Lukas Krenz was funded by ChEESE (823844), Leonhard Rannabauer was funded by ExaHyPE (671698).
\todo{LK: Standard sentence is: This research was funded by the European Union’s Horizon 2020 Research and Innovation Programme under theproject ExaHyPE, grant no.  671698 (call FETHPC-1-2014).

Also LRZ?}


\printbibliography{}
\end{document}

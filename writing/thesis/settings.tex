\PassOptionsToPackage{table,svgnames,dvipsnames}{xcolor}
\usepackage[utf8]{inputenc}
\usepackage[T1]{fontenc}

\usepackage[sc, osf]{mathpazo}
\linespread{1.025}
\usepackage[euler-digits,small]{eulervm}
\setkomafont{disposition}{\normalfont\bfseries}
\frenchspacing
\usepackage{pifont}% http://ctan.org/pkg/pifont
\newcommand{\cmark}{\ding{51}}%
\newcommand{\xmark}{\ding{56}}%

\usepackage[american]{babel}
\usepackage[autostyle, english = american]{csquotes}
\usepackage[%
  backend=biber,
  url=false,
  style=alphabetic,
  backref=true,
  hyperref=true,
  maxnames=4,
  minnames=3,
  maxbibnames=99,
  firstinits,
  uniquename=init]{biblatex}
\DefineBibliographyStrings{english}{%
  backrefpage = {see p.},
  backrefpages = {see pp.}
}
\addbibresource{../bibliography.bib}

\usepackage{graphicx}
\graphicspath{ {../figures/} }
\usepackage[prependcaption, textsize=tiny]{todonotes}
\usepackage{rotating}
\usepackage{tikz}

\usepackage{pgfplots}
\usepackage{pgfplotstable}
\usepackage{booktabs}
\usepackage{xspace}
\usepackage{xparse} % for NewDocumentCommand
\usepackage{etoolbox} % for notblank (brackets only when argument)
\usepackage{xstring} % for \IfSubStr
\usepackage[final,tracking=smallcaps,expansion=alltext, protrusion=true]{microtype}
\SetTracking{encoding=*, shape=sc}{50} %latex & friends, page 52
\usepackage{caption}
\usepackage{subcaption}
\usepackage[section]{placeins} % float barriers!
\usepackage{floatpag} % \thisfloatpagestyle
\usepackage{bm}
%\usepackage{tocstyle}\usetocstyle{noonewithdot}
%\usepackage{array}
\usepackage{amsmath}
\usepackage{amsfonts}
\usepackage{amssymb}
\usepackage{mathtools} % for \mathclap
\usepackage{nicefrac}
\usepackage{physics} % for derivatives
\usepackage[separate-uncertainty]{siunitx}
\usepackage{calc}

% For debuggins:
%\usepackage{showframe}

% -------------- Algorithms --------------
\usepackage{algorithm,algorithmicx}
\usepackage[noend]{algpseudocode}

\newcommand*\Let[2]{\State #1 \(\gets\) #2}
\algrenewcommand\algorithmicrequire{\textbf{Input: }}
\algdef{SE}[DOWHILE]{Do}{doWhile}{\algorithmicdo}[1]{\algorithmicwhile\ #1}

\DeclareMathOperator*{\argmin}{argmin}
\DeclareMathOperator{\BigO}{O}
\newcommand*\from{\colon}

\newcommand\sidetitle[1]{\leavevmode\marginline{\small\emph{#1}}\ignorespaces}

% Settings for pgfplots
\pgfplotsset{compat=1.9} % TODO: adjust to your installed version
\pgfplotsset{
  % For available color names, see http://www.latextemplates.com/svgnames-colors
  cycle list={CornflowerBlue\\Dandelion\\ForestGreen\\BrickRed\\},
}


% -----------------TOC--------------------
% TODO!
\usepackage{titletoc}

\titlecontents{chapter}
  [0cm]{\bfseries \raggedleft\parshape 1 1cm \dimexpr\linewidth-3cm\relax}
  {\noindent\rule{4cm}{0.1pt}\\ \contentslabel{1.5em}}{}{\hspace{0.8em}\textbullet\hspace{0.8em}\makebox[1em][l]{\thecontentspage}
  }

\titlecontents{section}
  [0cm]{\scshape \raggedleft \parshape 1 1cm \dimexpr\linewidth-3cm\relax}
  {\contentslabel{2em}}{}{\hspace{0.8em}\textbullet\hspace{0.8em}\makebox[1em][l]{\thecontentspage}}

\titlecontents{subsection}
  [0cm]{\itshape \raggedleft\ \parshape 1 1cm \dimexpr\linewidth-3cm\relax}
  {\contentslabel{2.5em}}{}{\hspace{0.8em}\textbullet\hspace{0.8em}\makebox[1em][l]{\thecontentspage}}


% -----------------Headings-----------------
% TODO!
% See https://tex.stackexchange.com/questions/230635/formatting-and-styling-of-headers-using-titlesec-in-scrbook-class
\colorlet{chaptercolor}{gray!80!black}

\setkomafont{chapter}{\normalfont\color{chaptercolor}\Huge\raggedleft}
\setkomafont{chapterprefix}{\Large}
\renewcommand*{\chapterformat}{%
  \makebox[\linewidth][r]{\MakeUppercase{\chapappifchapterprefix{}}}%
  \rlap{\enskip\resizebox{!}{1.2cm}{\thechapter} \rule{15cm}{1.2cm}}%
}

\renewcommand*\chapterheadstartvskip{\vspace{20pt}}
\renewcommand*\chapterheadendvskip{\vspace{20pt}}

\usepackage{varioref}
\usepackage[%
  citecolor=Gray,
  linkcolor=Black,
  urlcolor=Black,
  colorlinks=true]{hyperref}
\usepackage{hyperref}
\usepackage[noabbrev]{cleveref}
\newcommand{\creflastconjunction}{, and\nobreakspace} % use Oxford comma

% For referencing in a form of (equation 5) instead of (equation (5))
% From https://tex.stackexchange.com/questions/263523/is-it-possible-to-remove-parenthesis-from-single-cref-references
\crefformat{equation}{\eqA{}eq.~\eqB #2#1#3)}
% TODO: Fix for more than two equations!
\crefmultiformat{equation}{eqs.~\eqMultiA#2#1#3\eqMultiB}%
{ and \eqMultiA#2#1#3\eqMultiB}{, \eqMultiA#2#1#3\eqMultiB}{ and~\eqMultiA#2#1#3\eqMultiB}
\newcommand{\eqA}{}
\newcommand{\eqB}{(}
\newcommand{\eqMultiA}{(}
\newcommand{\eqMultiB}{)}
\DeclareRobustCommand{\pcrefSingle}[1]{%
\begingroup%
  \renewcommand{\eqA}{(}\renewcommand{\eqB}{}%
\cref{#1}%
\endgroup%
}
\DeclareRobustCommand{\pcrefMulti}[1]{%
\begingroup%
    \renewcommand{\eqMultiA}{}\renewcommand{\eqMultiB}{}%
    (\cref{#1})%
\endgroup%
}
\DeclareRobustCommand{\pcref}[1]{%
\IfSubStr{#1}{,}{\pcrefMulti{#1}}{\pcrefSingle{#1}}%
}


% Commands/Macros
% Names of methods or so
\newcommand{\muscl}{\textsc{muscl}-Hancock}
\newcommand{\dg}{\textsc{dg}}
\newcommand{\ader}{\textsc{ader}}
\newcommand{\aderdg}{\textsc{ader-dg}}
\newcommand{\amr}{\textsc{amr}}
\newcommand{\pde}{\textsc{pde}}
\newcommand{\cpp}{C\texttt{++}}
\newcommand{\tbb}{\textsc{tbb}}
\newcommand{\mpi}{\textsc{mpi}}

\newcommand{\softwareName}[1]{\textit{#1}}
\newcommand{\exahype}{\softwareName{ExaHyPE}}
\newcommand{\peano}{\softwareName{Peano}}
\newcommand{\className}[1]{\texttt{#1}}
\newcommand{\funName}[1]{\textsf{#1}}
\newcommand{\fileName}[1]{\textbf{#1}}
\newcommand{\varName}[1]{\enquote{#1}}



% Variables and other equation stuff
\newcommand{\Q}{\bm{Q}}
\newcommand{\gradQ}{\gradient{\Q}}
\newcommand{\Qrho}{\rho}
\newcommand{\Qj}{\rho \bm{v}}
\newcommand{\Qv}{\bm{v}}
\newcommand{\QE}{\rho E}
\newcommand{\QZZ}{Z} % TODO: Name?
\newcommand{\QZ}{\rho \QZZ}
\newcommand{\potT}{\theta}
\newcommand{\backgroundPotT}{\overline{\theta}}
\newcommand{\pertubationPotT}{\theta'}
\newcommand{\stressT}{\bm{\sigma}}
\newcommand{\pressure}{p}
\newcommand{\maxConvEigen}[1][]{
  \vert%
  \lambda_c^{\text{max}}
  \notblank{#1}{\left(#1\right)}{}
  \vert%
}
\newcommand{\maxViscEigen}[1][]{
  \vert%
  \lambda_v^{\text{max}}
  \notblank{#1}{\left(#1\right)}{}
  \vert%
}
\newcommand{\Riemann}{\operatorname{Riemann}}

% Domain
\newcommand{\domain}{\Omega}
\newcommand{\boundary}{\partial \domain}
\newcommand{\broken}{\domain}

% Cells
\newcommand{\cell}[1][]{C_{#1}}
\newcommand{\refCell}[1][]{\hat{C}_{#1}}
\newcommand{\cellb}{\partial{} \cell}
\newcommand{\refCellb}{\partial{} \refCell}

% Basis functions
\newcommand{\lagrange}[1][i]{\varphi_{#1}}
\newcommand{\lagrangeRef}[1][i]{\hat{\varphi}_{#1}}
\NewDocumentCommand{\sbasis}{ O{} O{} }{\phi^{#1}_{#2}}
\NewDocumentCommand{\sbasisRef}{ O{} O{} }{\hat{\phi}^{#1}_{#2}} %{\hat{\sbasis[#1][#2]}}
\NewDocumentCommand{\stbasis}{ O{} O{} }{ \psi^{#1}_{#2}}
\NewDocumentCommand{\stbasisRef}{ O{} O{} }{\hat{\psi}^{#1}_{#2}} %{\hat{\stbasis[#1][#2]}}
%\newcommand{\stestfunction}[1]{\sbasis[#1]}
\NewDocumentCommand{\stestfunction}{ O{} O{} }{\sbasis[#1][#2]}
\NewDocumentCommand{\sttestfunction}{ O{} O{} }{\stbasis[#1][#2]}

% Bold if no index is supplied.
\newcommand{\bmempty}[2]{
\notblank{#2}{#1}{\bm{#1}}
}

% Space-time dofs
\NewDocumentCommand{\stpredictor}{ O{} O{} }{\overline{\bmempty{q}{#2}}^{#1}_{#2}}
\NewDocumentCommand{\stpredictorCoeff}{ O{} O{} }{\underline{\overline{\bmempty{q}{#2}}}^{#1}_{#2}}

\NewDocumentCommand{\stflux}{ O{} O{} }{\overline{\bmempty{F}{#2}}^{#1}_{#2}}
\NewDocumentCommand{\stfluxCoeff}{ O{} O{} }{\underline{\overline{\bmempty{F}{#2}}}^{#1}_{#2}}

% \NewDocumentCommand{\stsol}{ O{} O{} }{\overline{u}^{#1}_{#2}}
% \NewDocumentCommand{\stsol}{ O{} O{} }{\underline{\overline{u}}^{#1}_{#2}}

\NewDocumentCommand{\stsource}{ O{} O{} }{\overline{\bmempty{S}{#2}}^{#1}_{#2}}
\NewDocumentCommand{\stsourceCoeff}{ O{} O{} }{\underline{\overline{\bmempty{S}{#2}}}^{#1}_{#2}}

% Space dofs
\NewDocumentCommand{\spredictor}{ O{} O{} }{\bmempty{q}{#2}^{#1}_{#2}}
\NewDocumentCommand{\spredictorCoeff}{ O{} O{} }{\underline{q}^{#1}_{#2}}

\NewDocumentCommand{\ssource}{ O{} O{} }{\bmempty{S}{#2}^{#1}_{#2}}
\NewDocumentCommand{\ssourceCoeff}{ O{} O{} }{\underline{\bmempty{S}{#2}}^{#1}_{#2}}

\NewDocumentCommand{\ssol}{ O{} O{} }{\bmempty{u}{#2}^{#1}_{#2}}
\NewDocumentCommand{\ssolCoeff}{ O{} O{} }{\underline{\bmempty{u}{#2}}^{#1}_{#2}}

\NewDocumentCommand{\sflux}{ O{} O{} }{\bmempty{F}{#2}^{#1}_{#2}}
\NewDocumentCommand{\sfluxCoeff}{ O{} O{} }{\underline{\bmempty{F}{#2}}^{#1}_{#2}}

% Equation parts
\newcommand{\flux}{\bm{F}}
\newcommand{\viscFlux}{\flux^{v}}
\newcommand{\hyperFlux}{\flux^{h}}
%\newcommand{\source}{\bm{S}}
\newcommand{\source}[1][]{
  \notblank{#1}{
S_{#1}
}{
\bm{S}
}
}

% Integrals
\newcommand{\curTime}{t^k}
\newcommand{\nextTime}{t^{k+1}}
\newcommand{\intdt}[1]{\int_{\curTime}^{\nextTime} #1 \dd{t}}
\newcommand{\intdcell}[1]{\int_{\cell} #1 \dd{\bm{x}}}
\newcommand{\intdrefcell}[1]{\int_{\refCell} #1 \dd{\hat{\bm{x}}}}
\newcommand{\intdcellb}[1]{\int_{\cellb} #1 \dd{S}} % TODO: define boundary of cell
\newcommand{\intdrefcellb}[1]{\int_{\refCellb} #1 \hat{\dd{S}}} % TODO: define boundary of cell
\newcommand{\intddomain}[1]{\int_{\domain} #1 \dd{\bm{x}}}

% Yet to be sorted
\newcommand{\quadWeight}[1][i]{w_{#1}}
\newcommand{\quadNode}[1][i]{\hat{x}_{#1}}
\newcommand{\quadNodeNd}[1][i]{\hat{\bm{x}}_{#1}}
\newcommand{\sbasisSize}{(N+1)^\dimensions}
\newcommand{\stbasisSize}{(N+1)^{\dimensions+1}}
\newcommand{\normal}{\bm{n}}

\newcommand{\pleft}{\bm{L}}
\newcommand{\prightsol}{\bm{r}}
\newcommand{\prightpred}{\bm{w}}

\newcommand{\reactionTimescale}{\tau}
\newcommand{\reactionTemperature}{T_{\text{ign}}}

\newcommand{\tv}{\operatorname{TV}}

\newcommand{\NVar}{N^{\text{var}}}
\newcommand{\dimensions}{d}
\newcommand{\mapping}{\mathcal{M}}
\newcommand{\volume}{V}
\newcommand{\cellCenter}{\operatorname{cell-center}}

% MUSCL
\newcommand{\cellAvg}[1][i,j]{\bm{U}_{#1}}
\newcommand{\extrapolatedCellAvg}[3][i,j]{\cellAvg[#1]^{#3 #2} = \cellAvg #3 \frac{1}{2} \slope{#2}}%
\newcommand{\evolvedCellAvg}[2][i,j]{\hat{\bm{U}}_{#1}^{#2}}
\newcommand{\sign}{\operatorname{sign}}
\newcommand{\minmod}{\operatorname{minmod}}
\newcommand{\slope}[2][i,j]{\bm{s}^{#2}_{#1}}
\newcommand{\gradCellAvg}[1][i,j]{\gradient{\cellAvg[#1]}}
\newcommand{\fluxX}{\flux_x}
\newcommand{\fluxY}{\flux_y}%


%%% Local Variables:
%%% mode: latex
%%% TeX-master: "main"
%%% End:

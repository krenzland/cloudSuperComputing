% TODO: Extract to settings.tex

% Variables and other equation stuff
\newcommand{\Q}{\bm{Q}}
\newcommand{\gradQ}{\gradient{\Q}}
\newcommand{\Qrho}{\rho}
\newcommand{\Qj}{\rho \bm{v}}
\newcommand{\Qv}{\bm{v}}
\newcommand{\QE}{\rho E}
\newcommand{\QZZ}{Z} % TODO: Name?
\newcommand{\QZ}{\rho \QZZ}
\newcommand{\potT}{\theta}
\newcommand{\backgroundPotT}{\overline{\theta}}
\newcommand{\pertubationPotT}{\theta'}
\newcommand{\stressT}{\bm{\sigma}}
\newcommand{\pressure}{p}
\newcommand{\maxConvEigen}[1][]{
  \vert%
  \lambda_c^{\text{max}}
  \notblank{#1}{\left(#1\right)}{}
  \vert%
}
\newcommand{\maxViscEigen}[1][]{
  \vert%
  \lambda_v^{\text{max}}
  \notblank{#1}{\left(#1\right)}{}
  \vert%
}
\newcommand{\Riemann}{\operatorname{Riemann}}

% Stuff for numerics
\newcommand{\domain}{\Omega}
\newcommand{\broken}{\domain}
\newcommand{\cell}[1][i]{C_{#1}}
\newcommand{\boundary}{\partial \domain}
\newcommand{\sbasis}[1]{\Phi_{#1}}
\newcommand{\stbasis}[1]{\Phi_{#1}}
\newcommand{\testfunction}[1]{\Phi_{#1}}
\newcommand{\normal}{\bm{n}}
\newcommand{\dsol}[1][h]{\bm{u}_{#1}}
\newcommand{\stpredictor}[1][h]{\bm{q}_{#1}}

% Equation parts
\newcommand{\flux}{F}
\newcommand{\viscFlux}{\flux^{v}}
\newcommand{\hyperFlux}{\flux^{h}}
\newcommand{\source}{\bm{S}}

% Integrals
\newcommand{\intdt}[1]{\int_{t^n}^{t^{n+1}} #1 \dd{t}}
\newcommand{\intdcell}[1]{\int_{\cell} #1 \dd{\bm{x}}}
\newcommand{\intdrefcell}[1]{\int_{\cell} #1 \dd{\hat{\bm{x}}}}
\newcommand{\intdcellb}[1]{\int_{\partial{} \cell} #1 \dd{S}} % TODO: define boundary of cell
\newcommand{\intdrefcellb}[1]{\int_{\partial{} \cell} #1 \hat{\dd{S}}} % TODO: define boundary of cell

% Names of methods or so
\newcommand{\muscl}{\textsc{muscl}-Hancock}
\newcommand{\dg}{\textsc{dg}}
\newcommand{\ader}{\textsc{ader}}
\newcommand{\aderdg}{\textsc{ader-dg}}
\newcommand{\amr}{\textsc{amr}}
\newcommand{\pde}{\textsc{pde}}

\chapter{An ADER-DG scheme for the Navier-Stokes Equations}\label{chap:methods}
This chapter describes the numerical and physical background that is needed to simulate reacting fluids with good accuracy.
We start with a description of the arbitrary high order derivatives (\ader) discontinuous Galerkin (\dg) method with a focus on equations that contain both advective and diffusive terms.
We then seque into a short description of the \muscl\ scheme, a second order finite volume scheme.
This discussion of the discretization enables us to talk about the equation set in sufficient detail.

We conclude this chapter by a description of adaptive mesh refinement (\amr) and finite volume limiting, which combines both numerical methods to achieve a stable, high-order method.

\section{The ADER-DG Method}\label{sec:ader-dg}
We describe the arbitrary derivative discontinous Galerkin (\aderdg) method in this chapter.
Our description of the main method follows~\cite{dumbser2008unified,dumbser2010arbitrary,dumbser2018efficient}, our notation follows primarily~\cite{dumbser2018efficient}.
We discuss the approximation of systems of partial differential equations (\pde) of the form
\begin{equation}
  \label{eq:conservation-law-gradient}
 \frac{\partial}{\partial_t}  \Q + \div \flux(\Q, \gradQ) = \source(\bm{x}, t, \Q),
\end{equation}
which in contrast to hyperbolic conservation laws of the form
\begin{equation}
  \label{eq:conservation-law}
 \frac{\partial}{\partial_t}  \Q + \div \flux(\Q) = \source(\bm{x}, t, \Q)
\end{equation}
contain the gradient $\gradQ$ of the so called vector of conservative variables $\Q$~\cite{dumbser2010arbitrary}.
There are therefore not hyperbolic but rather parabolic or elliptic.

We discuss the solution of a hyperbolic conservation law \cref{eq:conservation-law} with domain $\domain$ and boundary $\boundary$.
Let $\domain$ and $\boundary$ denote our two- or three-dimensional computational domain and its boundary.
In our implementation of the discontinous Galerkin (\dg) framework, we approximate this solution in the space
\begin{equation}
  \label{eq:dg-space}
  \broken = \bigcup_i \cell
\end{equation}
of disjoint quadrilateral cells $\cell$.
Note that we do not distinguish between the broken approximation space and the domain, the use should be clear given its surrounding context.
In the following we make use of the Einstein summation convention where summation over repeated indexes is implied.

Inside each cell $\cell$ we represent the solution in terms of the basis function 
\begin{equation}
  \label{eq:cell-approx}
  u(\bm{x}, t^n)_{|\cell[i]} = \hat{u}^n_{i,l} \sbasis{l}(\bm{x}),
\end{equation}
where $l$ is a multi-index, containing one index per spatial dimension.
\todo{dofs $\hat{u}$, local solution, etc.}
For example, $(l = (l_1, l_2))$ for the two dimensional case.
This polynomial is interpolating, i.e.\ \ldots\todo{interpolating?}.
This choice of basis (called \textit{nodal} basis) allows us to easily compute integrals over cells using Gaussian quadrature.
In detail, we use a the Lagrange interpolation polynomials.

We now describe the derivation of the \textsc{ader-dg} method.
This scheme is a predictor-corrector method.
We first compute a local solution of the cell in the predictor step and then connect with the neighbors in the corrector step.
In the following we first describe the corrector step as it follows directly from the \pde.
The predictor is derived shortly after.

\begin{algorithm}[H]
  \begin{algorithmic}
    \Let{$\stpredictor$}{Initial guess}
    \Let{$F_h$}{$\flux(\Q, \gradQ)$}
    \Let{$\stpredictor$}{Solve \cref{eq:space-time-predictor} using $F_h, \Q, \gradQ$}
    \Let{$\left( \hat{F_h}, \hat{\Q}, \hat{\gradQ}, \hat{\stpredictor} \right)$}
        {Extrapolate to boundary using \cref{eq:boundary-extrapolation}}
    \Let{update}{Use \cref{eq:corrector} with boundary extrapolated values (fluxes of neighbors)}
  \end{algorithmic}
  \caption{The \textsc{ader-dg} algorithm for one cell.}
\end{algorithm}

\sidetitle{Reference Cell}
\newcommand{\mapping}{\mathcal{M}}
\newcommand{\volume}{V}
For reasons of computational efficiency, we describe the scheme in terms of a reference cell.
We use the quad cell $\cell = [0, 1]^D$\todo{check if $[0,1]$ detontes correct interval}.
The mapping, which takes a point on the reference cell with coordinates $\hat{\bm{x}}$ and returns a point $\bm{x}$ on a cell with center and widths $(\Delta x, \Delta y, \Delta z)$, is given by
\begin{equation}
  \hat{x} = \mapping (\bm{x}) =
  \operatorname{cell-center}(\cell) +
\begin{pmatrix}
\Delta x && \\
&\Delta y & \\
&&\Delta z&
\end{pmatrix}
 % \operatorname{diag}(\Delta x, \Delta y, \Delta z)
  \begin{pmatrix}
    x - 0.5\\
    y - 0.5\\
    z - 0.5
  \end{pmatrix}.
\end{equation}
We also define the inverse mapping $\mapping^{-1}$.
This can be used to define a function $f(\bm{x})$ acting on a point $\bm{x}$ of a regular cell in term of a function $\hat{f}(\bm{x})$ that acts on a point of the reference cell by considering
\begin{equation}
  \label{eq:function-ref-cell}
  f(\bm{x}) = f\left( \mapping^{-1}(\bm{x}) \right) = \hat{f}(\hat{\bm{x}}).
\end{equation}

The determinant of $\mapping$ is simply the volume $V$ of the cell
\begin{equation}
  \label{eq:determinant-mapping}
  \volume = \operatorname{det}(\operatorname{Jacobian}\mapping) = \Delta x \, \Delta y \,\Delta z.
\end{equation}
This is useful for evaluating derivatives by the chain rule and, similarly, integrals by substitution.
For example, we can evaluate a volume integral by
\begin{equation}
  \label{eq:integration-by-substitution}
  \intdcell{
f(\bm{x})
  }
  =
\volume \intdrefcell{
    \hat{f}(\hat{\bm{x}})
  }
\end{equation}

\todo{Maybe show this.}

\sidetitle{Predictor}
To derive the predictor, we again take an approximation of our solution in a nodal basis, but now consider polynomials that are defined both in space and time
\begin{equation}
  \label{eq:cell-approx-space-time}
  q(\bm{x}, t^n)_{|\cell[i]} = \hat{u}^n_{i,l} \stbasis{l}(\bm{x}, t).
\end{equation}
We now multiply the conservation law again by a test function (of the same function space as the basis) and arrive at the weak formulation
\begin{equation}\label{eq:weak-pde-space-time}
\intdt{\intdcell{
    \testfunction{k}(\bm{x}, t)
    \pdv{\stpredictor}{t}
}}
+
\intdt{\intdcell{
    \testfunction{k}(\bm{x}, t)
    \left(
      \divergence{\flux(\stpredictor, \gradient{\stpredictor})}
    \right)
}}
=
\intdt{\intdcell{
  \testfunction{k} \source(\stpredictor)
}}.
\end{equation}
Similar to the derivation of the corrector, we again integrate the first term by parts in time and the flux divergence in space.
This time we do not use the Riemann solver for the flux boundary term but rather use the discrete solution at time $t$.
Note that this neglects the interaction with neighbouring cells; we account for this in the corrector step.
\todo{Replace $\bm{x}$ with a macro some other letter. Is also coordinate!}

\begin{align}\label{eq:space-time-predictor}
\begin{split}
\intdcell{
  \testfunction{k} (\bm{x}, t^{n+1}) \stpredictor(\bm{x}, t^{n+1})
}
&-
\intdt{\intdcell{
    \pdv{}{t} \testfunction{k}(\bm{x}, t) \stpredictor(\bm{x}, t)
}}
-
\intdcell{
  \testfunction{k}(\bm{x},t^n) \dsol(\bm{x}, t^n)
} = \\
\intdt{\intdcell{
    \testfunction{k}(\bm{x},t) \divergence{\flux(\stpredictor, \gradient{\stpredictor})}
}}
&+
\intdt{\intdcell{
    \testfunction{k}(\bm{x}, t) \source(\stpredictor)
}}
\end{split}
\end{align}
Inserting \cref{eq:cell-approx-space-time} results in a local systems of equations that can be solved in a fixed point iteration scheme.
\todo{Inital guess correct? Maybe cite sth?}
As initial value for the space-time-predictor we use the solution of the timestep before.
\todo{Collect into matrices and show scheme?}
To show the computational efficiency of our scheme, we follow the approach of~\cite{dumbser2008unified} and collect our integrals into matrices.
\begin{equation}
  \label{eq:predictor-matrices}
  \begin{alignedat}{2}
& A = \intdt{\intdcell{
    \testfunction{k} \pdv{t} \stbasis{l}
}}
\qquad&& B \\
& A
\qquad&& B
\end{alignedat}
\end{equation}
For details and proof of convergence for the linear case, see~\cite{dumbser2008unified}.

\sidetitle{Boundary extrapolation and stuff}
\todo{Explain that stpredictor and fluxes are saved and extrapolated to boundary.
This is important because we extrapolate the gradient as well!}
Extrapolate unknowns to boundary, convert from our basis to basis on faces and so on

\sidetitle{Corrector}
First, we multiply the system \cref{eq:conservation-law} by a test function $\testfunction{i}$ and integrate over the space-time volume $(\cell \times [t^n, t^{n+1}])$.
We arrive at the so called weak formulation of the \pde\
\begin{equation}
  \label{eq:weak-pde}
\intdt{\intdcell{
\testfunction{k} \pdv{\Q}{t}
}}
+
\intdt{\intdcell{
    \testfunction{k} \left( \div{F(\Q, \gradQ} \right)
}}
=
\intdt{\intdcell{
    \testfunction{k} S(\Q, \bm{x}, t)
}}
\end{equation}
\todo{Include gradient everywhere!}
\todo{Double check the following description}
We now replace the solution $\bm{Q}$ with so called spacetime-predictor $\stpredictor (\bm{x},t)$ into the weak form and write it as a polynomial using the representation of \cref{eq:cell-approx}.
Integrate first by parts in time (note basis here not defined over time), flux divergence by parts in space.
\newcommand{\massMatrixDef}{\intdcell{
  \testfunction{k} \sbasis{l}
}}
\begin{align}
\begin{split}
\label{eq:corrector}
\left(
\massMatrixDef
\right)
(\bm{u^{n+1} - u^{n}})
&+
\left(\intdt{\intdcell{
      \testfunction{k} \Riemann(\stpredictor^-, \stpredictor^+) \cdot \normal
}}\right)
-\\
\left(\intdt{\intdcell{
    \gradient{\testfunction{k}} \cdot  \flux(\stpredictor, \gradient{\stpredictor}) % todo correct gradient?
}}\right)
&=
\left(\intdt{\intdcell{
      \testfunction{k} \source(\stpredictor)
}}\right)
\end{split}
\end{align}
The first term is our mass matrix with entries
\newcommand{\massMatrix}[1][]{\bm{M}_{#1}}
\newcommand{\quadWeight}[1][i]{w_{#1}}
\begin{equation}
  \label{eq:mass-matrix}
  \massMatrix[kl] = \massMatrixDef = \volume \underbrace{\quadWeight[k] \delta_{kl}}_{\text{diagonal}}
\end{equation}
which is diagonal and thus convenient to invert.
\newcommand{\sumbasis}{\sum_i^{(N + 1)^d - 1}}
We can then insert the definition of our basis function into our equation and arrive at
\begin{align}
  \massMatrix \left( \dsol^{t+1} - \dsol^{t} \right) &=
  \Delta t \left( \bm{a} - \bm{b} + \bm{s} \right) 
  \intertext{with}
\begin{split}
  a &= \sumbasis \\
  b &= \sumbasis \\
  s &= \sumbasis.
\end{split}
\end{align}
We can solve this for the new unknows $\bm{u}^{n+1}$
\begin{align}\label{eq:update-predictor}
\begin{split}
  \bm{u}^{n+1} &= \bm{u}^{n} + \Delta t (\operatorname{update}) \\
  \operatorname{update} &= \massMatrix^{-1} \left( \bm{a} - \bm{b} + \bm{s} \right)
\end{split}
\end{align}
which is clearly of the form of a one-step scheme.

\sidetitle{Riemann solver \textit{\&} timestep}
As a Riemann solver we use a simple Rusanov-flux that is adapted for diffusive problems
\begin{equation}
  \label{eq:rusanov-flux}
  \Riemann(q_h^-, \nabla q_h^-; g_h^+, \nabla q_h^+) \cdot \bm{n} =
  \frac{1}{2} \left(
    F(q_h^+, \nabla q_h^+) +
    F(q_h^-, \nabla q_h^-)
  \right) -
  \frac{1}{2} s_\text{max} (q_h^+ - q_h^-),
\end{equation}
with a penalty term
\begin{equation}
  \label{eq:parabolic-penalty}
  s_\text{max}  = \max \left(
\maxConvEigen[q_h^-], \, \maxConvEigen[q_h^+]
\right) +
2 \eta \max \left(
\maxViscEigen[q_h^-], \, \maxViscEigen[q_h^+]
\right)
\end{equation}
and
\begin{equation}
  \eta = \frac{2N+1}{h \sqrt{\frac{1}{2} \pi}}.
\end{equation}
In this equation, $N$ is the polynomial order and $h$ is the side-length of an element.
The penalty term depends on the maximal absolute eigenvalues of both the convective and viscous part of the equations
\begin{align}
  \begin{split}
    \maxConvEigen &= \left( \partial \bm{F}/\partial \bm{Q}\right) \cdot \normal,\\
    \maxViscEigen &= \left( \partial \bm{F}/\partial \left( \nabla \bm{Q} \cdot \normal \right)\right) \cdot \normal,
  \end{split}
\end{align}
in direction of the normal vector $\normal$ to the cell face. 

This Riemann solver was first published in~\cite{gassner2008discontinuous} and used for an \textsc{ader-dg} scheme in~\cite{dumbser2010arbitrary}.
The timestep is restriced to a so-called \textsc{cfl}-type penalty
\begin{equation}\label{eq:cfl-aderdg}
 \Delta t \leq  \text{CFL} \, \frac{\alpha(N) \, h}{\maxConvEigen + 2 \maxViscEigen \frac{2N+1}{h}}
\end{equation}
with $N$ polynomial order and $h$ characteristic length scale of elements~\cite{dumbser2010arbitrary,gassner2008discontinuous}.
The constant $\alpha(N) \leq {\left( 2N+1  \right)}^{-1}$ is obtained from von Neumann analysis on a simple model problem and depends on the approximation order~\cite{dumbser2008unified}.
We use a value of $0.7$ for the constant $\text{CFL}$.
\todo{Describe CFL-cond.\ for multiple dimensions!}

\section{MUSCL-Hancock Finite Volume Scheme}\label{sec:muscl}
\newcommand{\cellAvg}[1][i,j]{U_{#1}}
\newcommand{\sign}{\operatorname{sign}}
\newcommand{\minmod}{\operatorname{minmod}}
\newcommand{\slope}[2][i,j]{s^{#2}_{#1}}
\newcommand{\gradCellAvg}[1][i,j]{\gradient{\cellAvg[#1]}}
\newcommand{\fluxX}{\flux_x}
\newcommand{\fluxY}{\flux_y}
We discussed a modern, high-order \textsc{DG} method in the previous chapter.
While this scheme is efficient, it can become unstable, especially in case of discontinuities.
A simple alternative is a finite volume scheme.
We use the \muscl\ method, our discussion and notation follows the one in~\cite{toro2013riemann}.
Similar to~\cite{toro2013riemann} we only describe the two-dimensional case, an extension to three dimensions is straigthforward.
\todo{Source term!}

In contrast to the previous \dg\ scheme, we now only store cell averages instead of a full polynomial solution per cell.
We achieve higher order then by a reconstruction of a linear function.
Let $\cellAvg$ denote the averages per cell at spatial coordinates $i,j$.

\sidetitle{Boundary Extrapolated Values}
We first compute the slope of the linear function connecting the cells.
This is done by assuming a linear function per cell.
\todo{Explain concrete limiter!}
\todo{Cite minmod limiter. Is this correct? It's the one used}
We use the minmod slope-limiter
\begin{equation}
  \label{eq:minmod}
  \operatorname{minmod}(a, b) =
  \begin{cases}
    0.0 & \sign(a) \neq \sign(b) \\
      a & \vert a \vert < \vert b \vert \\
      b & \text{otherwise}
  \end{cases}
\end{equation}
to avoid unphyiscally steep gradients and stabilize the system.
This function then be used to compute the slopes
\begin{align}\label{eq:slopes}
  \begin{split}
   \slope{x} &=  \Delta x \minmod \left( \cellAvg[i+1,j] - \cellAvg[i,j], \, \cellAvg[i,j] - \cellAvg[i-1, j] \right),\\ 
   \slope{y} &=  \Delta y \minmod \left( \cellAvg[i,j+1] - \cellAvg[i,j], \, \cellAvg[i,j] - \cellAvg[i, j-1] \right),
   \end{split}
\end{align}
where $\Delta x$ and $\Delta y$ are the inverse cell sizes in $x$ and $y$ direction respectively.

We then use the slope to reconstruct the value of the boundary.
In the following $\cellAvg^{\pm x}$ is the average of the left ($+$) or right ($-$) cell boundary.
Similar, $\cellAvg^{\pm y}$ is the value at the top/bottom cell boundary.
These so called boundary extrapolated values are given by
\newcommand{\extrapolatedCellAvg}[3][i,j]{\cellAvg[#1]^{#3 #2} = \cellAvg #3 \frac{1}{2} \slope{#2}}
\begin{equation}
\begin{alignedat}{2}
& \extrapolatedCellAvg{x}{-} , \qquad && \extrapolatedCellAvg{x}{+}, \\
& \extrapolatedCellAvg{y}{-} , \qquad && \extrapolatedCellAvg{y}{+}.
\end{alignedat}
\end{equation}
We also use the slopes defined in \cref{eq:slopes} to estimate the gradient of $\Q$ in each cell by the block matrix\todo{Notation!}
\begin{equation}
  \label{eq:muscl-gradient}
  \gradCellAvg = \left( \slope{x} \bigg\rvert \slope{y} \right),
\end{equation}
which is of course an abuse of notation insofar as it is not the gradient of the constant cell value but rather of the linear reconstruction.
\sidetitle{Time Evolution}
To achieve second order in time, we evolve the boundary extrapolated values in time by
\todo{Correct index for gradient of cell average}
\todo{Actually compare with implementation. Currently gradient is ignored here!}
\todo{Describe splliting of $\flux$ into $\fluxX, \fluxY$}
\newcommand{\evolvedCellAvg}[2][i,j]{\hat{U}_{#1}^{#2}}
\begin{equation}\label{eq:muscl-time-evolution}
  \begin{split}
  \forall_{k \in [-x, +x, -y, +y]}:  \hat{U}_{ij}^k = \cellAvg^k &+
  \frac{\Delta t}{2 \Delta x} \left[ \fluxX(\cellAvg^{-x}, \gradCellAvg) - \fluxX(\cellAvg^{+x}, \gradCellAvg) \right] \\&+
  \frac{\Delta t}{2 \Delta y} \left[ \fluxY(\cellAvg^{-y}, \gradCellAvg) - \fluxY(\cellAvg^{+y}, \gradCellAvg) \right].
  \end{split}
\end{equation}

Finally, the update can be described by\todo{Double check update}
\begin{align}
  \begin{split}
    \cellAvg (t^{n+1}) &= \cellAvg (t^n)
    \\ &+
  \frac{\Delta t}{\Delta x}
    \Riemann \left(
      \evolvedCellAvg[i-1,j]{+x}, \gradCellAvg[i-1,j]; \ %
      \evolvedCellAvg[i,j\phantom{-1}]{-x}, \gradCellAvg[i,j\phantom{-1}]
    \right)
    \\ &-
  \frac{\Delta t}{\Delta x}
    \Riemann \left(
      \evolvedCellAvg[i,j\phantom{+1}]{+x}, \gradCellAvg[i,j\phantom{+1}]; \ %
      \evolvedCellAvg[i+1,j]{-x}, \gradCellAvg[i+1,j]
    \right)
    \\ &+
  \frac{\Delta t}{\Delta y}
    \Riemann \left(
      \evolvedCellAvg[i,j-1]{+y}, \gradCellAvg[i,j-1]; \ %
      \evolvedCellAvg[i,j\phantom{-1}]{-y}, \gradCellAvg[i,j\phantom{-1}]
    \right)
    \\ &-
  \frac{\Delta t}{\Delta y}
    \Riemann \left(
      \evolvedCellAvg[i,j\phantom{+1}]{+y}, \gradCellAvg[i,j\phantom{+1}]; \ %
      \evolvedCellAvg[i,j+1]{-y}, \gradCellAvg[i,j+1]
    \right)    
  \end{split}
\end{align}

We again use the Rusanov-type Riemann solver~\cref{eq:rusanov-flux}.
The timestep is subject to a penalty of the form
\begin{equation}\label{eq:cfl-muscl}
 \Delta t \leq  \text{CFL} \, \frac{h}{\maxConvEigen + 2 \maxViscEigen \frac{1}{h}},
\end{equation}
with a constant value of CFL that is typically close to $0.7$.

\section{Reactive Compressible Navier Stokes Equations}\label{sec:navier-stokes}
Fluid motion can be described by the compressible Navier Stokes equations.
We follow the description in~\cite{dumbser2010arbitrary} for the Navier Stokes part.
The coupling with the advection-diffusion-reaction equation follows~\cite{hidalgo2011ader} which describes this equation set for the one-dimensional case.

The vector of conserved quantities is given by
\begin{equation}
  \label{eq:conserved-variables}
 \Q = \left( \Qrho, \Qj, \QE, \QZ \right),
\end{equation}
where $\Qrho$ is the density, $\Qj$ is the momentum, $\QE$ the energy density and $\QZ$ is the mass fraction of the chemical reactant.

We split the flux into a hyperbolic part $\hyperFlux(\Q)$ and an viscous flux $\viscFlux(\gradQ)$
\begin{equation}
  \label{eq:flux}
  \flux(\Q, \gradQ) = \hyperFlux(\Q) + \viscFlux(\Q, \gradQ).
\end{equation}

\newcommand{\diffCoeff}{\varepsilon}
\newcommand{\hyperFluxDef}{
  \begin{pmatrix}
    \Qj \\
    \Qv  \otimes \Qj + \bm{I} \pressure  \\
    \Qv \cdot (\bm{I} \QE + \bm{I} \pressure) \\
    \Qj \QZZ
  \end{pmatrix}
}

\newcommand{\viscFluxDef}{
  \begin{pmatrix}
     -\diffCoeff \gradient{\Qrho}\\
     \stressT (\Q, \gradQ)  \\
     \Qv \cdot \stressT (\Q, \gradQ) - \kappa \gradient{T}\\
     -\diffCoeff \gradient{\QZ}
   \end{pmatrix}
}

The hyperbolic flux is given by
\begin{equation}
  \label{eq:hyper-flux}
  \hyperFlux(\Q) = \hyperFluxDef,
\end{equation}
which are the Euler equations coupled with an advection equation.
The viscous flux is
\begin{equation}
  \label{eq:visc-flux}
  \viscFlux(\Q, \gradQ) = \viscFluxDef.
\end{equation}
where $\pressure$, $\stressT$ and $(\kappa \nabla T)$ denote the pressure, stress tensor and heat flux respectively.
Temperature is denoted by $T$.


Putting everything together, our complete \pde\ is a conservation law of the form of \cref{eq:conservation-law-gradient}
\begin{equation}
 \begin{array}{l}
 \text{mass cons.} \\
 \text{momentum cons.} \\
 \text{energy cons.} \\
 \text{cont.\ gas} 
\end{array}
:
\quad
  \pdv{}{t}
  \underbrace{
  \begin{pmatrix}
    \Qrho\\
    \Qj\\
    \QE\\
    \QZ
    \end{pmatrix}}_{\Q}
  + 
  \divergence{
  \left(
   \underbrace{\hyperFluxDef}_{\hyperFlux(\Q)}
+
\underbrace{\viscFluxDef}_{\viscFlux(\Q, \gradQ)}
  \right)}
 =
  \underbrace{
  \begin{pmatrix}
    S_{\Qrho\phantom{\Qrho}}\\
    S_{\Qj}\\
    S_{\QE}\\
    S_{\QZ}
    \end{pmatrix}}_{\source(\Q, \bm{x}, t)}
\end{equation}

We close the system with the equation of state of an ideal reacting gas
\begin{equation}
  \label{eq:eos}
  \pressure = (\gamma - 1) \left(\QE - \frac{1}{2} \left(\Qv \cdot \Qj \right)  - q_0 \QZ \right).
\end{equation}
The term $q_0 \QZ$ corresponds to the chemical energy, where $q_0$ is the heat release.
The temperature $T$ relates to pressure and density by the ideal gas law
\begin{equation}
  \label{eq:temperature}
 \frac{\pressure}{\Qrho} = RT,
\end{equation}
where $R$ is the specific gas constant.

We further introduce the ratio of specific heats $\gamma$ and the heat fraction for constant volume $c_v$ and constant volume $c_r$.
These constants are all fluid dependent and relate to each other by
\begin{align}
  \label{eq:fluid-constants}
  \begin{split}
  c_v &= \frac{1}{\gamma - 1} R \\
  c_p &= \frac{\gamma}{\gamma - 1} R\\
  R &= c_p - c_v\\
  \gamma &= \frac{c_p}{c_v}.
  \end{split}
\end{align}
\todo{Cite sth!}
We can finally define the heat conduction coefficient $\kappa$
\begin{equation}
  \label{eq:heat-conduction-coeff}
  \kappa = \frac{\mu \gamma}{\Pr} \frac{1}{\gamma - 1} R = \frac{\mu \gamma}{\Pr} c_v
\end{equation}
where Prandtl number $\Pr$ again depends on the fluid.

% Sutherland's viscosity law:
% \begin{equation}
%   \label{eq:sutherland}
%  \mu(T)  = \mu_0 {\left(\frac{T}{T_0}  \right)}^{\beta} \frac{T_0 + C}{T + C}
% \end{equation}
% with \(\beta = 1.5\), \(C = \text{const.}\), reference temperature $T_0$ and reference viscosity $\mu_=$.
% Equal to
% \begin{align}
%   \lambda &= \frac{\mu_0 (T_0 + C)}{T_0^\beta} \\
%   \mu(T) &= \lambda \frac{T^\beta}{T + C},
% \end{align}
% where $\lambda$ is constant for a given fluid.

The viscous effects are modeled by the stress tensor
\begin{equation}
  \label{eq:stress-tensor}
  \stressT(Q, \nabla Q) =
  \mu
  \left(
  \left(\nicefrac{2}{3} \divergence{\Qv} \right) -
  \left( \gradient{\Qv} + \gradient{\Qv}^\intercal \right)
  \right).
\end{equation}

The maximal eigenvalues for convective and viscous part are
\begin{align}
  \begin{split}
  \maxConvEigen \vert  &= \Vert \Qv \Vert + c\\
  \maxViscEigen \vert &= \max \left( \frac{4}{3} \frac{\mu}{\Qrho},
                        \frac{\gamma \mu}{\Pr \Qrho},
                        \diffCoeff \right)
  \end{split}
\end{align}
with speed of sound $c = \sqrt{\gamma R T }$.

\subsection{Boundary conditions}
To close the system we need to impose boundary conditions.

For some scenarios we use Cauchy boundary conditions.
In most cases, we would like to impose periodic boundary conditions, due to the inner workings of ExaHyPE this is not possible.
Instead we use the analytical solution of our problems at the boundary, imposing both value and gradients of the conservative variables.
Note that this leads to an error when our problem does not posses an exact analytical solution.
This is the case for test cases that are analytical solutions to the incompressible Navier Stokes equations but do not satisfy the compressible equation set.

As a physical boundary condition we limit ourselves to the no-slip boundary condition, where we assume that the fluid has a velocity of zero near the wall.
\todo{Check if this is the correct physical description!}
We enfore this by setting
\begin{align}
  \label{eq:no-slip}
  \begin{split}
  \Qrho^o &= \Qrho^i, \\
  \Qj^o &= -\Qj^i, \\
  \QE^0 &= \QE^i,\\
  {(\nabla Q)}^o &= {(\nabla Q)}^i,
  \end{split}
\end{align}
where a superscript of $o$ and $i$ denotes the values outside and inside of the boundary respectively.

\section{Adaptive Mesh Refinement \textit{\&} Finite Volume Limiting}\label{sec:grid}
Grid, spacefilling curve, and so on.

\subsection{Adaptive Mesh Refinement}\label{sec:amr}
We use adaptive mesh refinement (\textsc{amr}).

We use the total variation (\textsc{tv}) over an arbitrary indicator variable given by
\newcommand{\tv}{\operatorname{TV}}
\begin{equation}
  \label{eq:tv}
  \tv \left[ f(\bm{x}] \right] =
\Vert \left(\intdcell{ \vert \gradient{f \left( \bm{x} \right)} \vert } \right) \Vert_1
\end{equation}
This integral can be evaluated efficiently using \cref{eq:integration-by-substitution}.
\todo{Gradient on reference cell, Gaussian quad., inline}
\todo{Definition for FV schemes (maybe just mapping to DG-Cells)}

\newcommand{\gobs}{\operatorname{G}}
\newcommand{\mean}{\mu}
\newcommand{\std}{\sigma}
\newcommand{\variance}{\std^2}
\newcommand{\gobsCount}{n}
We want to compute the mean and the variance over all grid cells by performing only pairwise merge operations.
This allows us to minimize communication over nodes and allows for a simple implementation.
The simplest way to compute this would be by using the textbook definitions of mean and variance of a random variable $X$
\newcommand{\expectation}{\mathbb{E}}
\begin{align}
  \begin{split}
    \mu [X] &= \expectation \left[ X \right]\\
    \variance [X] &= \expectation \left[ X^2 \right] - \expectation \left[ X \right]^2.
  \end{split}
\end{align}
The reduction for the variance is numerically unstable when the variance is orders of magnitudes smaller than the mean as this can lead to catastrophic cancellations.
We thus compute the mean and variance with the parallel algorithm of~\cite{chan1982updating}.
To do this, we need to store three variables per computation unit:
the mean, the variance and the number of processed elements.
We collect these items in the vector $\gobs = (\mean, \variance, \gobsCount)$
We can then merge two pairs of observed variables (\cref{alg:merge-variance}).

\begin{algorithm}[ht]
  \begin{algorithmic}
% TODO: Special cases 
\Function{reduce-variance}{$\gobs_0, \gobs_1$}
\If{$\gobsCount_0 = 0$}
  \State\Return{$\mean_1, \variance_1, \gobsCount_1$}
\EndIf\
\If{$\gobsCount_1 = 0$}
  \State\Return{$\mean_0, \variance_0, \gobsCount_0$}
\EndIf\
  \Let{$\Delta$}{$\mean_1 - \mean_0$}  
  \Let{$\gobsCount_\Sigma$}{$\gobsCount_0 + \gobsCount_1$}
  \Let{$m_a$}{$\variance_0 (\gobsCount_0 - 1)$}
  \Let{$m_b$}{$\variance_1 (\gobsCount_1 - 1)$}
  \Let{$m_\Sigma$}{$\nicefrac{m_a + m_b + (\Delta^2 \gobsCount_0 \gobsCount_1)}{\gobsCount_\Sigma}$}
  \Let{$\mean_\Sigma$}{$\mean_0  \gobsCount_0 + \mean_1 \gobsCount_1$}
  \State\Return{$
    \nicefrac{\mean_\Sigma}{\gobsCount_\Sigma},
    \nicefrac{m_\text{total}}{\gobsCount_\Sigma - 1},
    \gobsCount_\Sigma
    $}
\EndFunction\
  \end{algorithmic}
  \caption{\label{alg:merge-variance}
    Merging two sets of reduced mean and variance~\cite{chan1982updating}}
\end{algorithm}
\todo{Cite Chebychev, add concrete form of threshold.}
Chebychev's inequality
\begin{equation}
  \label{eq:chebychev}
  \mathbb{P}(\vert X - \mu \vert \geq c \sigma) \leq \frac{1}{c^2},
\end{equation}
with probability $\mathbb{P}$,
holds for an arbitrary distribution with mean $\mu$ and standard deviation $\sigma$ probability $\mathbb{P}$.
This motivates the following refinement criterion
\begin{equation}
  \label{eq:refinement-criterion}
  \operatorname{evaluate-refinement}(\Q, \mu, \sigma) =
  \begin{cases}
    \text{refine} & \text{if } \tv(\Q) \geq \mu + T_r \sigma \\
    \text{delete} & \text{if } \tv(\Q) \leq \mu + T_r \sigma \\
    \text{keep} & \text{otherwise}
    \end{cases}
\end{equation}
which refines when a cell has a significantly higher standard deviation.
In this equation, the constants $T_r$ and $T_d$ can be used to tailor the trade-off between the quality of approximation and computational cost.
They can be chosen freely, as long $T_d < T_r$ holds.
Chebychev's inequality then guarantees that only a subset is marked for refinement.
Note that even though this inequality provides only a loose bound.
If one would be willing to assume a distribution of the indicator variable, tighter bounds can be derived.

\subsection{Limiting}\label{sec:limiting}
Higher order \textsc{dg} methods cannot cope with discontinuous solutions.
Even worse, in the case of non-linear fluxes, discontinuities can appear even from smooth initial data.
A classical way of dealing with this problem is the usage of so called limiters.
The idea is to smooth out steep gradients and thus eliminate discontinuities.\todo{It doesnt work like this exactly..}

A relatively recent way of dealing with this problem is the finite volume subcell limiter.
Our discussion follows the one in~\cite{dumbser2016simple}.
This limiting is an a posteriori limiting, which means that we first evaluate a timestep with an unlimited \textsc{ader-dg} method and then check, whether our solution is \enquote{correct}.
In our case, we check whether the solution is a valid floating point number.
We also check whether it is physically admissible.
In our case we use the criterion
\begin{equation}
  \label{eq:limiting-physical}
  \operatorname{is-admissible}(\Q) =
  \begin{cases}
    \text{true} & \Qrho > 0 \land \pressure > 0 \land \left(0 \leq \QZ \leq 1 \right)\\
    \text{false} & \text{otherwise}.
  \end{cases}
\end{equation}
The positivity of the pressure implies positivity of energy.
\todo{Mention dmp if I end up using it.}

If a solution violates our criteria, we recompute the timestep using a limited finite volume method.
In our case we use the \muscl{}-method (\cref{sec:muscl}).
The recomputation starts by subdividing the infringing cell into $N_2 = 2N + 1$ subcells.
This number of subcells has the advantage that we do not loose spatial resolution.
We can see this easily by noticing the similar form of \cref{eq:cfl-aderdg} and \cref{eq:cfl-muscl}.

For the sake of brevity, we omit technical details.
We refer the interested reader to~\cite{dumbser2016simple}.

%%% Local Variables:
%%% mode: latex
%%% TeX-master: "../main"
%%% End:

\chapter{Methods}
Methods here.\todo{Organize into chapters!}

% TODO: Extract to settings.tex
\newcommand{\Qrho}{\ensuremath{\rho}}
\newcommand{\Qj}{\ensuremath{\rho \bm{v}}}
%\newcommand{\Qv}{\ensuremath{\Qrho^{-1} \Qj}}
\newcommand{\Qv}{\ensuremath{\bm{v}}}
\newcommand{\QE}{\ensuremath{\rho E}}
\newcommand{\stressT}{\ensuremath{\bm{\sigma}}}
\newcommand{\pressure}{\ensuremath{p}}
\newcommand{\maxConvEigen}{\ensuremath{\vert \lambda_c^{\text{max}} \vert}}
\newcommand{\maxViscEigen}{\ensuremath{\vert \lambda_v^{\text{max}} \vert}}

\section{Conservative Form}
Standard (hyperbolic) form of conservation law
\begin{equation}
  \label{eq:conservation-law}
 \frac{\partial}{\partial_t}  Q + \nabla \cdot F(Q) = S(\bm{x}, t, Q)
\end{equation}
extend to:
\begin{equation}
  \label{eq:conservation-law}
 \frac{\partial}{\partial_t}  Q + \nabla \cdot F(Q, \nabla Q) = S(\bm{x}, t, Q)
\end{equation}

\section{Compressible Euler}
Vector of conserved quantities:
\begin{equation}
  \label{eq:conserved-variables}
 Q = \left( \Qrho, \Qj, \QE \right) 
\end{equation}

\newcommand{\eulerFlux}{%
  \begin{pmatrix}
    \Qj \\
    \Qv  \otimes \Qj + \pressure \bm{I}  \\
    \Qv \cdot (\bm{I} \QE + \bm{I} \pressure)
  }
Flux:
\begin{equation}
  F(Q) = \eulerFlux
  \end{pmatrix}
\end{equation}
\begin{equation}
  \pressure = (\gamma - 1) \left(\QE - 0.5 \left(\Qv \cdot \Qj \right) \right)
\end{equation}

\section{Compressible Navier Stokes}
Following~\cite{dumbser2010arbitrary} by \citeauthor{dumbser2010arbitrary}.

Flux:
\begin{equation}
  F(Q, \nabla Q) = 
  \begin{pmatrix}
    \Qj \\
    \Qv  \otimes \Qj + \bm{I} \pressure + \stressT (Q, \nabla Q)  \\
    \Qv \cdot (\bm{I} \QE + \bm{I} \pressure + \stressT (Q, \nabla Q)) - \kappa \nabla T
  \end{pmatrix}
\end{equation}

With:
Pressure \pressure
\begin{equation}
  \pressure = (\gamma - 1) \left(\QE - 0.5 \left(\Qv \cdot \Qj \right) \right)
\end{equation}

Temperature $T$
\begin{equation}
  \label{eq:temperature}
 \frac{\pressure}{\Qrho} = RT
\end{equation}

Heat conduction coefficient $\kappa$
\begin{equation}
  \label{eq:heat-conduction-coeff}
  \kappa = \frac{\mu \gamma}{\Pr} \frac{1}{\gamma - 1} R
\end{equation}
with $R$ gas constant and $Pr$ Prandtl number

Sutherland's viscosity law:
\begin{equation}
  \label{eq:sutherland}
 \mu(T)  = \mu_0 {\left(\frac{T}{T_0}  \right)}^{\beta} \frac{T_0 + C}{T + C}
\end{equation}
with \(\beta = 1.5\), \(C = \text{const.}\), reference temperature $T_0$ and reference viscosity $\mu_=$.
Equal to
\begin{align}
  \lambda &= \frac{\mu_0 (T_0 + C)}{T_0^\beta} \\
  \mu(T) &= \lambda \frac{T^\beta}{T + C},
\end{align}
where $\lambda$ is constant for a given fluid.

stress tensor
\begin{equation}
  \label{eq:stress-tensor}
  \stressT(Q, \nabla Q) = \left(\nicefrac{2}{3} \mu \nabla \cdot \Qv \right) -
  \mu \left( \nabla (\Qv) + \nabla ( \Qv )^\intercal \right)
\end{equation}

Max eigenvalue of convective part \maxConvEigen\,i.e.\ of $\left( \partial \bm{F}/\partial \bm{Q}\right) \cdot \bm{n}$,
and viscous part \maxViscEigen\, i.e.\ of $\left( \partial \bm{F}/\partial \left( \nabla \bm{Q} \cdot \bm{n} \right)\right) \cdot \bm{n}$
is
\begin{align}
  \maxConvEigen \vert  &= \Vert \Qv \Vert + c\\
  \maxViscEigen \vert &= \max \left( \frac{4}{3} \frac{\mu}{\Qrho}
                                       , \frac{\gamma \mu}{\Pr \Qrho} \right)
\end{align}
with speed of sound $c = \sqrt{\gamma R T }$

Maximum timestep (\textsc{cfl})
\begin{equation}
 \Delta t = \frac{\text{CFL}}{2N + 1} \, \frac{h}{\maxConvEigen + 2 \maxViscEigen \frac{2N+1}{h}}
\end{equation}
with $N$ polynomial order and $h$ characteristic length scale of elements

Numerical flux:
\begin{equation}
  \label{eq:rusanov-flux}
  G(q_h^-, \nabla q_h^-; g_h^+, \nabla q_h^+) \cdot \bm{n} =
  \frac{1}{2} \left(
    F(q_h^+, \nabla q_h^+) +
    F(q_h^-, \nabla q_h^-)
  \right) -
  \frac{1}{2} s_\text{max} (q_h^+ - q_h^-)
\end{equation}
with
\begin{equation}
  \label{eq:parabolic-penalty}
  s_\text{max}  = \max \left(
\vert \lambda_c(q_h^-) \vert, \, \vert \lambda_c(q_h^+)
\right) +
2 \eta \max \left(
\vert \lambda_v(q_h^-) \vert, \, \vert \lambda_v(q_h^+)
\right)
\end{equation}
and
\begin{equation}
  \eta = \frac{N+1}{h}
\end{equation}


\section{Scenarios}
\label{sec:scenarios}

\subsection{Clouds}
Air is initially at rest and in hydrostractic balance
\begin{equation}
  \label{eq:hydrostatic-balance}
 \pdv{p (z)}{z} = -\rho(z) g.
\end{equation}
Inserting \cref{eq:temperature} leads to the \textsc{ode}
\begin{equation}
  \pdv{p(z)}{z} = - \frac{p(z)}{RT} g,
\end{equation}
which can be solved by seperation of variables.
This results in the equation
\begin{equation}
  \label{eq:bubble-pressure}
  p(z) = c \exp \left( - \frac{gz}{RT} \right),
\end{equation}
where the constant of integration $c$ can be found by considering the case
\begin{equation}
  p(0) = c.
\end{equation}

The scenario is described in terms of potential temperature $\theta$
\begin{equation}
  \theta = T \frac{p_0}{p}^{R/c_p},
\end{equation}
where $p_o = \SI{10e5}{\Pa}$ is the reference pressure.
Solving for $T$ leads to
\begin{equation}
  T = \theta (\frac{p}{p_0})^{R/c_p}.
\end{equation}

Note: Above approach leads to wrong atmosphere, rather do (with constant potential temperature):
Equation of state with potential temperature:
\begin{equation}
  \label{eq:eos}
\Qrho(z) = \frac{p_{0}^{\frac{c_{p} - c_{v}}{c_{p}}} p^{\frac{c_{v}}{c_{p}}}{\left (z \right )}}{R \theta}
\end{equation}
Hydrostatic equilibrium with potential temperature:
\begin{equation}
  \label{eq:hydrostatic-balance-potT}
  \frac{d}{d z} p{\left (z \right )} = - \frac{g p_{0}^{\frac{R}{c_{p}}} p^{\frac{1}{\gamma}}{\left (z \right )}}{R \theta}
\end{equation}
Solution for pressure with initial condition of $p(0) = p_0$:
% \begin{equation*}
% p = \left(\frac{R \gamma \theta \left(\gamma^{\frac{\gamma}{\gamma - 1}} p_{0}\right)^{\frac{\gamma - 1}{\gamma}} - R \theta \left(\gamma^{\frac{\gamma}{\gamma - 1}} p_{0}\right)^{\frac{\gamma - 1}{\gamma}} - g \gamma p_{0}^{\frac{R}{c_{p}}} z \left(\gamma - 1\right) + g p_{0}^{\frac{R}{c_{p}}} z \left(\gamma - 1\right)}{R \gamma \theta \left(\gamma - 1\right)}\right)^{\frac{\gamma}{\gamma - 1}}
% \end{equation*}
\begin{equation}
p(z) =  \left(C - \frac{C}{\gamma} - \frac{g p_{0}^{\frac{R}{c_{p}}} z}{R \theta} + \frac{g p_{0}^{\frac{R}{c_{p}}} z}{R \gamma \theta}\right)^{\frac{\gamma}{\gamma - 1}},
\end{equation}
with constant of integration
\begin{equation}
  \label{eq:hydrostatic-balance-constant}
 C = \frac{\left(\gamma^{\frac{\gamma}{\gamma - 1}} p_{0}\right)^{\frac{\gamma - 1}{\gamma}}}{\gamma - 1},
\end{equation}
which can be obtained by setting $p(0) = p_0$.

\section{Convergence}
We first define the $L_p$ norms for $p > 0$ by
% notation: https://en.wikipedia.org/wiki/Lp_space#Lp_spaces
\begin{equation}
  \label{eq:Lp-nrom}
  \Vert f(x) \Vert = \left( \int_K \vert f(x) \vert^p d\mu  \right)^{1/p}.
\end{equation}
We start by a known analytical solution $f(x)$ and compare it to our approximation $\hat{f}(x)$.
Let $Q$ and $\hat{Q}$ denote the analytical and approximate solution at node.
We start by computing the point-wise error, and compute the cell-wise error from this.
Observe that for our broken space $\Omega$
\begin{equation}
  \label{eq:lp-norm-broken}
 \Vert f(x) \Vert = \sum_{K \in \Omega} \Vert f_K(x) \Vert_p.
\end{equation}

We are now ready to integrate the error for each cell.
The output contains the position and conserved variables of each point.
Each cell consists of $(N + 1)^d$ nodes, each associated with a quadrature weight $w_i$.
The quadrature weights are associated with the reference cuboid.
We thus need to map our element to the reference element, before we can compute the integral.

Let $V$ denote the volume (or area) of each cell.
For example, in three dimensions $V = \Delta x \, \Delta y \, \Delta z$.
The final result is then
\begin{equation}
  \Vert f_K(x) \Vert = \left( V \sum_{\bm{i}} \vert f(\bm{x}_{\bm{i}}) \vert^p w_{{\bm{i}}}  \right)^{1/p},
\end{equation}
where we used the mapping to the reference triangle (integration by substitution).

Following~\cite{dumbser2010arbitrary}, the scenario can be described in primitive variables by setting
\begin{align}
  \pressure &= \pressure_0 \cos( \bm{k} \bm{x} - \omega t ) + \pressure_b, \\
  \Qrho &= \Qrho_0 \sin (\bm{k} \bm{x} - \omega t) + \Qrho_b, \\
  \Qv &= \bm{v_0} \sin(\bm{k} \bm{x} - \omega t).
\end{align}
We set the constants to \( \left(  \pressure_0 = 0.1, \pressure_b = \gamma^{-1}, \Qrho_0 = 0.5, \Qrho_b = 1, 
\bm{v_0} = 0.25 {(1,1)}^\intercal, \bm{k} = \pi/5 {(1,1)}^\intercal \right) \).

We derive a source term by inserting this solution into the \textsc{pde} using the symbolic math toolkit \textit{SymPy}.

%%% Local Variables:
%%% mode: latex
%%% TeX-master: "../main"
%%% End:

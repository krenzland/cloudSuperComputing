\chapter{Conclusion}\label{chap:conclusion}
Conclusion here.

lkjl:
\begin{itemize}
\item We showed in \cref{sec:results-convergence} that our scheme converges for all considered orders for the Navier-Stokes equations.
\item cfd-scenarios competitive results
\item clouds accurate simulated
\item The results of \cref{sec:results-tts-amr} showed, that our \amr{}-criterion is able to save a significant portion of the computational effort compared to a uniformly fine grid.
  It is able to do this without degrading the quality of the solution, even for a scenario that contains small-scale details.
\item Chemical reactions and limiter work for non-stiff detonations.
\end{itemize}

Future work:
\begin{itemize}
\item Improvement of boundary conditions for muscl
\item Due to the \textsc{cfl}-condition,the timestep of our \aderdg{}-scheme scales quadratically with the number of grid cells.
  Even though our \amr{}-criterion is able to reduce the number of needed cells drastically, we could get another large improvement by utilizing local, instead of global timestepping.
\item Our implementation of the \textsc{cj}-detonation wave cannot handle realistic detonations because we do not have an implementation of stiff source terms.
\end{itemize}
%%% Local Variables:
%%% mode: latex
%%% TeX-master: "../main"
%%% End:

\chapter{Conclusion}\label{chap:conclusion}
We implemented a \muscl{}-limited \aderdg{}-scheme with \amr{} for the reactive Navier Stokes equations in this thesis.
We showed that the scheme works and is capable of simulation different scenarios:\todo{Reference sections of results. Reference theory chapter. Reference implementation.}
\begin{itemize}
\item We showed in \cref{sec:results-convergence} that our scheme converges for all considered orders for the Navier-Stokes equations.
\item We evaluated our method for standard \textsc{cfd} scenarios~(\cref{sec:cfd}) and achieved competitive results for all used scenarios.
  This is true for both two-dimensional scenarios (Taylor-Green and Lid-Driven Cavity) and for the three-dimensional \textsc{abc}-flow.
\item Furthermore, our method allows us to simulate flows in hydrostatic equilibrium correctly (see \cref{sec:results-cloud}).
\item The results of \cref{sec:results-tts-amr} showed that our \amr{}-criterion is able to save a significant portion of the computational effort compared to a uniform fine grid.
  It is able to do this without degrading the quality of the solution, even for a scenario that contains small-scale details.
\item Our results of \cref{sec:results-reactive} show that our implementation is capable of simulating chemical reactions with a limited scheme, at least for a source term that is not stiff.
\end{itemize}

Of course, there are some things that can be improved and which need further work, such as:
\begin{itemize}
\item The current treatment of the boundary conditions is not stable for some scenarios.
  Especially the viscous walls for the \muscl{}-method are not correct, due to limitations of \exahype{}.
\item The\textsc{cfl}-condition~\pcref{eq:cfl-aderdg,eq:cfl-muscl} has the effect that our method needs a large number of timesteps.
  Even though our \amr{}-criterion is able to reduce the number of needed cells drastically, we could get another large improvement by utilizing local timestepping.
\item Our implementation of the \textsc{cj}-detonation wave cannot handle realistic detonations because we do not have an implementation of stiff source terms.
  An implementation of this is straight-forward but includes changes to the space-time predictor and its initial guess, and also to the \muscl{}-scheme.
\end{itemize}

%%% Local Variables:
%%% mode: latex
%%% TeX-master: "../main"
%%% End:

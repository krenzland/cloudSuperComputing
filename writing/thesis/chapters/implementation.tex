\chapter{Implementation}\label{chap:implementation}
This chapter is organized as follows:
We first give a short overview of the general structure of the \exahype{} code and then describe the changes done over the course of this project.
\exahype{} implements an \aderdg{} solver with \muscl{} limiting and \textsc{amr}, as described in \cref{chap:methods}.
It uses \softwareName{Intel® Threading Building Blocks} (\tbb{}) for shared memory parallelism and the \softwareName{Message Passing Interface} (\mpi{}) for distributed memory parallelism. 
In the following, we define a \enquote{user} as someone who implements the logic of a \pde{} but does not change the implementation of \exahype{} itself.
This is in contrast with a \enquote{developer} whose scope is the entirety of the source code.
The goal of our implementation is that all new features should be accessible to users as well.

\section{ExaHyPE}
\exahype{} is based on the space-filling-curves library \peano~\cite{weinzierl2011}.
The workflow from the perspective of a user is:
\begin{itemize}
\item Generate glue-code using a \softwareName{Python}-based Toolkit that inserts parameters into templates.
  The parameters are configured in a setting file.
  The user can configure the number of dimensions, numerical order and type of solver (\aderdg{}, \muscl, or limiting-\aderdg).
  Depending on the solver type, a different implementation is generated.
  In case of a limiting solver, a finite-volume and a \dg\ implementation is generated.
  An implementation consists of
  \begin{itemize}
  \item An \className{AbstractSolver} that is a subclass of \className{FiniteVolumesSolver}, \className{ADERDGSolver} or \className{LimitingADERDGSolver}.
    The solver types all have a shared superclass \className{Solver} which describes a common interface.
  \item A class that inherits from the aforementioned abstract class that implements the \pde{} logic.
  \end{itemize}
  
\item The user then needs to modify the generated \className{Solver} class.
  The \pde{} is specified in the methods \funName{flux} and \funName{eigenvalues}.
  We implement the grid refinement criterion in \funName{refinementCriterion} and the limiting criterion in \funName{isPhysicallyAdmissible}.
  To close the system, the user needs to prescribe initial and boundary conditions with \funName{adjustPointSolution}, \funName{boundaryConditions} and \funName{boundaryValues}.
  The methods \funName{boundaryConditions} first calls \funName{boundaryValues} and then solves the Riemann-problem at the boundary.
  Thus, it suffices in many cases to only override the latter method.
\item Finally, the user can specify more options such as mesh size and spacing, simulation time, parallelization options, plotters, and optimization parameters.
 After compiling the program, it can be run with the setting file as first and only parameter.
\end{itemize}

From the perspective of a developer, the situation is more complicated.
We will thus only give the necessary minimum of information that is needed to understand the modifications conducted by us.
\exahype{} is structured as a \peano{} project and follows the basic structure:
\begin{itemize}
\item A runner iterates over the grid and calls an adapter for so-called events.
  An example for an event is entering a cell.
\item An adapter calls one or multiple mappings.
\item A mapping then performs the actual program logic.
  For our purposes, we only need to consider the mappings (with corresponding adapters):
  \begin{itemize}
  \item \className{FinalizeMeshRefinement}
  \item \className{FusedTimeStep}
  \end{itemize}
\item Often, a mapping calls a method defined in \className{Solver}.
  The implementation of this method can differ between different solvers.
\item The actual numerical logic is outsourced to kernels.
\end{itemize}
Furthermore, the degrees of freedom and some meta-information are stored for each cell in a heap data-structure.
The information for each \dg{}-cell is defined in the file \fileName{ADERDGCellDescrition.def}.
Similar files exist for the finite volume solver but are not relevant for us.
These files are converted to \cpp{} files by the software \softwareName{DaStGen}~\cite{bungartz2008dastgen}.

\section{Implementing numerics}
We needed to modify the existing implementation of the mappings, solvers and kernels to be able to solve problems with diffusive fluxes.
These changes are:
\begin{itemize}
\item Modifying the space-time-predictor (defined in file \fileName{spaceTimePredictorNonlinear.cpph}).
  \begin{itemize}
  \item The function \funName{aderPicardLoopNonlinear} implements the space-time-predictor loop \pcref{eq:space-time-predictor}.
    We pass the gradient of the conserved variables to the \funName{flux} method of the \className{Solver}.
    Because we need the time-average of the gradient later, we return it.
   \item In the function \funName{aderExtrapolatorNonlinear}, which implements the boundary extrapolation, we also compute the extrapolated gradient.
    \item In the function \funName{spaceTimePredictorNonlinear}, we also return the extrapolated gradient.
  \end{itemize}
\item Storing the extrapolated and time-averaged gradient for each cell for use in the boundary conditions and the corrector step.
\item Using the gradient for the computation of the boundary conditions.
  They are defined in the file \fileName{boundaryConditions.cpph}.
\item Adding the diffusive penalty terms for the Riemann solver\\(\fileName{riemannSolverNonlinear.cpph}) and \textsc{cfl}-condition (\fileName{stableTimeStepSize.cpph}).
  This corresponds to \cref{eq:rusanov-flux,eq:cfl-aderdg}.
\item Computing the gradient for the \muscl{} scheme (\fileName{musclhancock.cpph}) and passing it to the flux (\fileName{rusanov.cpph}).
\end{itemize}
Additionally, we needed to change the templates for the glue code and needed to modify some function signatures to accompany the other changes.

\section{Implementing Global Observables}
\newcommand{\ngobs}{N^{\text{gobs}}}
The implementation of the global observables is quite simple.
We use three functions that need to be defined by the user.
They adhere to the interface:
\begin{itemize}
\item The function \funName{mapGlobalObservables} computes the observables for a cell from its degrees of freedoms and its size.
  It returns a vector of size $\ngobs$.
\item The function \funName{reduceGlobalObservables} updates the current partially reduced vector of global observables \varName{reducedGlobalObservables} with another vector of partially reduced observables \varName{curGlobalObservables}.
\item The function \funName{resetGlobalObservables} takes no arguments and returns a vector of size $\ngobs$.
  This vector should be the identity of the reduction, i.e.\ a value $a$ with the property that for all $b$ $\funName{reduce}(a, b) = b$.
\end{itemize}
We then compute the observables by \cref{alg:reduce-gobs}
\begin{algorithm}[htb]
  \begin{algorithmic}
    \Let{observables}{\Call{resetGlobalObservables}{}}
    \For{$\text{cell} \in \text{cells}$}
    \Let{curObservables}{\Call{mapGlobalObservables}{$\Q_\text{cell}, \Delta \bm{x}_\text{cell}$}}
    \Let{observables}{\Call{reduceGlobalObservables}{observables, curObservables}}
    \EndFor{}
    \State\Return{observables}
  \end{algorithmic}
  \caption{\label{alg:reduce-gobs}Reducing global observables}
\end{algorithm}
This results in a simple, yet powerful interface, which can be used to implement all reductions that can be performed in a pair-wise manner.
Examples are computing the minimum, maximum, sum, mean and variance.

For the variance of the total variation, we store the partial mean, sample variance, and count as global observables.
The vector $(-1, -1, 0)$ is then the base-case for the iteration~\pcref{alg:reduce-gobs} and is thus returned by the function \funName{resetGlobalObservables}.
The method \funName{mapGlobalObservables} is defined to return $\left( \tv \left[ f(\Q) \right], 0, 1 \right)$, where $f(\Q)$ returns an arbitrary indicator value per cell.
Finally, \funName{reduceGlobalObservables} is implemented exactly as described in \cref{alg:merge-variance}.

For the actual implementation, we follow \cref{alg:reduce-gobs} but include distributed and shared memory parallelization in the following way:
\begin{itemize}
\item Adding vectors \varName{\_globalObservables} and \varName{\_nextGlobalObservables} to all solvers.
  The former stores the observables of the previous timestep, the latter the ones of the current timestep.
  We can thus use the old observables for mesh refinement and compute a new reduction at the same time.
  Before the first timestep, we reset both current and next observables.
  At the beginning of each timestep, we overwrite the previous data with the current data, and reset \varName{\_nextGlobalObservables} with \funName{resetGlobalObservables}.
  This is done in the methods \funName{startNewTimeStep}, \funName{startNewTimeStepFused}, \funName{updateTimeStepSizesFused} and \funName{updateTimeStepSizes}.
\item Reducing global observables in solver with method \funName{reduceGlobalObservables} that takes a \varName{cellInfo} and a \varName{solverNumber}.
  The solver number is always 0 because we only use one solver at a given time.
  The cell information stores pointers to the degrees of freedoms.
  We implement this method for all solvers.
  This has the advantage that we have a shared interface for all solver types.
\item Reducing global observables per node in the mappings \className{FinaliseMeshRefinement}, \className{FusedTimeStep} and \className{UpdateAndReduce}.
  \begin{itemize}
  \item We initialize the observables in the method \funName{beginIteration} with \funName{resetGlobalObservables}. 
  \item We reduce the observables in \funName{leaveCell} by calling the solver method \funName{reduceGlobalObservables} with the \varName{cellInfo} of the current cell as argument.
    For the mapping \className{UpdateAndReduce}, this is done in the method \funName{enterCell} instead.
  \item If we use \tbb{}, we merge the partially reduced observables in \funName{mergeWithWorkerThread} with \funName{reduceGlobalObservables} using the current reduced observables and the partially reduced observables from the worker.
    This is not necessary for the mapping \className{FinaliseMeshRefinement}.
   \item We update the global observables of the solver in the method \funName{endIteration}, which calls the method \funName{updateGlobalObservables}.
  \end{itemize}
\item Reducing the observables over all ranks, if using \mpi{}.
  \exahype{} already reduces the timestep size.
  We extend these already existing \mpi{} messages to also include the global observables \varName{\_globalObservables}.
  For this, we changed the following methods of the \dg{}-solver:
  \begin{itemize}
  \item \funName{compileMessageForMaster} and \funName{sendDataToMaster} to send data from the workers to the master rank.
  \item \funName{mergeWithWorkerData} to merge the received worker data with the current master data. 
  \item \funName{compileMessageForWorker} and \funName{sendDateToWorker} to send the reduced data from the master to the workers.
  \item \funName{mergeWithMasterData} to overwrite the current data of the workers with the reductions from the previous timestep.
  \end{itemize}
  The solvers update the variable \varName{\_nextGlobalObservables} during the \mpi{}-reduction.
  This again corresponds to \cref{alg:reduce-gobs} together with the aforementioned resetting of the observables, 
\end{itemize}
The global observables interface is defined for all solvers, the \mpi{}-reduction only for the \aderdg{}-solver.

%%% Local Variables:
%%% mode: latex
%%% TeX-master: "../main"
%%% End:

\chapter{Scenarios}\label{sec:scenarios}
Unless otherwise noted, we use the constants
\begin{align}
\begin{split}
  \gamma &= 1.4, \\
  \Pr &= 0.7, \\
  c_v &= 1.0
\end{split}
\end{align}
The other constants follow from the definition of \cref{eq:fluid-constants}.

We describe some scenarios in terms of the primitive variables $(\Qrho, \Qv, \pressure)$, these can be converted to the conservative variables by the definition of momentum and the definition of energy.   
\section{Numerical Convergence Test}
Following~\cite{dumbser2010arbitrary}, the scenario can be described in primitive variables by setting
\begin{align}
\begin{split}
  \pressure(\bm{x}, t) &= \pressure_0 \cos( \bm{k} \bm{x} - \omega t ) + \pressure_b, \\
  \Qrho(\bm{x}, t) &= \Qrho_0 \sin (\bm{k} \bm{x} - \omega t) + \Qrho_b, \\
  \Qv(\bm{x}, t) &= \bm{v_0} \sin(\bm{k} \bm{x} - \omega t).
\end{split}
\end{align}
We set the constants to \( \left(  \pressure_0 = 0.1, \pressure_b = \gamma^{-1}, \Qrho_0 = 0.5, \Qrho_b = 1, 
\bm{v_0} = 0.25 {(1,1)}^\intercal, \bm{k} = \pi/5 {(1,1)}^\intercal \right) \).

We derive a source term by inserting this solution into the \textsc{pde} by using the symbolic math toolkit \textit{Sage}~\cite{sagemath}.
This procedure is called method of manufactured solutions~\cite{salari2000code}.

\section{Classical Scenarios}
In addition to the convergence test, we use standard \textsc{cfd} scenarios to evaluate our simulations.


% \subsection{The first Stokes Problem}
% \todo{Do I actually want to include this? Bit boring scenario}
% \cite{dumbser2010arbitrary}

\subsection{Taylor-Green Vortex}
\begin{align}
  \label{eq:taylor-green}
  \begin{split}
  \Qrho(\bm{x}, t) &= 1\\
  \Qv(\bm{x}, t) &= \exp(-2 \mu t)
  \begin{pmatrix}
    \phantom{-}\sin(x) cos(y) \\
- \cos(x) sin(y) 
    \end{pmatrix} \\
  \pressure(\bm{x}, t) &= \exp(-4 \mu t) \, \frac{1}{4} \left( \cos(2x) + \cos(2y) \right) + C
  \end{split}
\end{align}
\todo{Constants!}
$C = 100/\gamma$ \cite{dumbser2016high}

As initial conditions we use \cref{eq:taylor-green} at time $(t = 0)$ and as boundary conditions we impose the solution and the corresponding gradients.
Note that this solution is not exactly correct, as it assumes an incompressible flow.
Due to the low mach number a compressible flow approaches this solution.

\subsection{Lid-Driven Cavity Flow}
The lid-driven cavity flow is a very simple testing scenario for the Navier-Stokes equations.
We use a constant pressure of (what exactly) and a constant density $(\Qrho = 1.0)$ for the entire domain.
The flow is then driven entirely by the boundary conditions.
They consist in four no-slip walls where the upper wall has a constant velocity $\Qv = (1.0, 0)$.
This scenario is surprisingly difficult for an \textsc{ader-dg} scheme as the boundary conditions lead to a discontinuity at the top left corner.
There, the velocity of the wall is piece-wise constant.
Therefore, this scenario is a good choice to test the stability of the finite volume limiter.

\subsection{ABC-Flow}
We use theArnold–Beltrami–Childress (\textsc{abc}) flow as a simple example of a three-dimensional scenario~\cite{tavelli2016staggered}.
This scenario has, unlike the three-dimensional version of the Taylor-Green vortex, at least for the incompressible case an analytical solution
\begin{align}
  \label{eq:abc-flow}
  \begin{split}
  \Qrho(\bm{x}, t) &= 1\\
  \Qv(\bm{x}, t) &= \phantom{-} \exp(-1\mu t)
  \begin{pmatrix}
    \sin(z) + \cos(y)\\
    \sin(x) + \cos(z)\\
    \sin(y) + \cos(x)
  \end{pmatrix} \\
  \pressure(\bm{x}, t) &= -\exp(-2 \mu t) \, \left(\cos(x)\sin(y) + \sin(x)\cos(z) + \sin(z)\cos(y)\right)
  + C
  \end{split}
\end{align}
Thus, we can simulate this without periodic boundary conditions.
\todo{Analytical solution!}


\subsection{Radialsymmetric CJ Detonation Wave}
We investigate the radialsymmetric CJ detonation wave scenario from~\cite{helzel2000modified}.
This scenario for the reactive Euler-equations is set up in a 2D box of size $[-0.5, 0.5] \times [-0.5, 0.5]$.
Inside a circle of radius $0.3$ we have totally unburnt gas, outside of this circle we have burned gas.
Let $\alpha =  \operatorname{atan2}(y,x)$ be the angle in polar coordinates.
We use the initial conditions
\begin{align}
  \begin{split}
  \Qrho_u &= 0.887565\\
  \Qv_u &= -0.577350
  \begin{pmatrix}
     \cos (\alpha)\\
     \sin (\alpha)
  \end{pmatrix} \\
  \pressure_u &= 0.191709 \\
  \QZZ_u &= 1
  \end{split}
\intertext{for the burnt gas and}
\begin{split}
  \Qrho_b &= 1.4\\
  \Qv_b &= 0\\
  \pressure_b &= 1 \\
  \QZZ_b &= 0
  \end{split}
\end{align}
for the unburnt gas.


\section{Atmospheric Scenarios}
The scenario is described in terms of potential temperature $\potT$
\begin{equation}
  \potT = T \frac{p_0}{p}^{R/c_p},
\end{equation}
where $p_o = \SI{10e5}{\Pa}$ is the reference pressure.
Solving for $T$ leads to
\begin{equation}
  \label{eq:potTToT}
  T = \potT (\frac{p}{p_0})^{R/c_p}.
\end{equation}
To allow for an easier description of the initial conditions we split the $\potT$ into a background state $\backgroundPotT$ and a pertubation $\pertubationPotT$
\begin{equation}
  \label{eq:potT-split}
  \potT = \backgroundPotT + \pertubationPotT.
\end{equation}

The initial conditions for pressure and density can be computed directly from the \textsc{pde} once few assumptions have been given.
We assume that $\backgroundPotT$ is constant over the entire domain.
Inserting the definition of potential temperature, given by \cref{eq:potTToT}, into the equation of state (see \cref{eq:eos}) leads us to
\begin{equation}
  \pressure(z) = \Qrho(z) R {\left( \backgroundPotT \frac{p}{p_0}\right)}^{R/c_p},
\end{equation}
which we solve for $p(z)$.
After some algebra and simplifying, we arrive at
\begin{equation}
 p(z) = p_0 \left( \frac{R \backgroundPotT \rho(z)}{p_0} \right)^{\gamma},
\end{equation}
or equivalently
\begin{equation}
  \label{eq:eos-potT}
\Qrho(z) = \frac{p_{0}^{\frac{R}{c_{p}}} p^{\frac{1}{\gamma}}{\left (z \right )}}{R \backgroundPotT}.
\end{equation}
We want to consider flow in hydrostatic equilibrium, i.e.\ flow where the force originating from the pressure-gradient is exactly balanced by gravitation.
Inserting this assumption in the momentum equation in $z$-direction and using \cref{eq:eos} once again, we arrive at the ordinary differential equation (\textsc{ode})
\begin{align}
  \label{eq:hydrostatic-balance-potT}
  \begin{split}
  \frac{d}{d z} p{\left (z \right )} &= -g \Qrho(z)
                                     = 
        - \frac{g p_{0}^{\frac{R}{c_{p}}} p^{\frac{1}{\gamma}}{\left (z \right )}}{R \backgroundPotT}\\
  p{\left (0 \right )} &= p_0
  \end{split}
\end{align}
where we use the reference pressure $p_0$ as the pressure on ground level.
This \textsc{ode} can be now simply solved by separation of variables.
After simplifying, we arrive at the initial condition for pressure
\begin{align}
  \label{eq:hydrostatic-background-pressure}
p(z) &= \left(\left(1 - \frac{1}{\gamma}\right) \left(C - \frac{g p_{0}^{\frac{R}{c_{p}}} z}{R \backgroundPotT}\right)\right)^{\frac{c_{p}}{R}},\\
\shortintertext{with constant of integration}
 C &= \frac{c_{p} p_{0}^{\frac{R}{c_{p}}}}{R}.
\end{align}

We now compute the pertubated potential temperature $\potT$ and use \cref{eq:potTToT} to convert it to temperature.
We then evaluate the density $\Qrho(\bm{x})$ with \cref{eq:temperature}.
The energy can be computed by inserting the previously computed pressure into the equation of state~(\ref{eq:eos}).

This leaves us with one problem:
The adiabatic no-slip boundary conditions defined by~\vref{eq:no-slip} are no longer valid.
Concretely, we need to impose a viscous heat flux to ensure that the atmosphere stays in hydrostatic balance~\cite{giraldo2008study}
\todo{Flux correct? Fix notation}
\begin{equation}
  \label{eq:atmosphere-bc}
  F^\text{visc} = \kappa \pdv{\overline{T}}{z} =
\frac{R \overline{\theta} \left(\frac{\overline{p}{\left (z \right )}}{p_{0}}\right)^{\frac{R}{c_{p}}} \frac{d}{d z} \overline{p}{\left (z \right )}}{c_{p} \overline{p}{\left (z \right )}}.
\end{equation}
The background pressure can be reconstructed by \cref{eq:hydrostatic-background-pressure}, its derivative is given by \cref{eq:hydrostatic-balance-potT}.

\subsection{Robert's smooth bubble}
\cite{robert1993bubble}
\subsection{Density current}
\cite{straka1993numerical}
%%% Local Variables:
%%% mode: latex
%%% TeX-master: "../main"
%%% End:

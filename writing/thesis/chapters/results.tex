\chapter{Results}\label{chap:results}
\section{Convergence Test}
\begin{table}[htb]
  \centering
\caption{Numerical order of convergence of ADER-DG method}%
\label{tab:convergence-order}
\begin{tabular}{@{}lS[table-format=1.2]S[table-format=1.2]S[table-format=1.2]@{}}
\toprule
{Order} & {$L_1$} & {$L_2$} & {$L_\infty$}\\ \midrule
1 & 2.03 & 2.00 & 1.92\\
2 & 2.56 & 2.55 & 2.56\\
3 & 3.44 & 3.41 & 3.42\\
4 & 4.27 & 4.27 & 4.33\\
5 & 5.36 & 5.32 & 5.26\\
6 & 5.38 & 5.35 & 5.17\\
\bottomrule
\end{tabular}
\end{table}
\newcommand{\error}{\operatorname{Total-Error}}

Before computing the error we first need to define the $L_p$ norms for a $p > 0$ by
% notation: https://en.wikipedia.org/wiki/Lp_space#Lp_spaces
\begin{equation}
  \label{eq:Lp-nrom}
  \Vert f(x) \Vert_p = \left( \int_K \vert f(x) \vert^p d\mu  \right)^{1/p}.
\end{equation}
We are interested in the total error which we define as the norm of the difference between an analytical solution $f(\bm{x}, t)$ and our approximation $\hat{f}(\bm{x}, t)$.
The error at a time $t$ is thus defined as
\begin{equation}
  \label{eq:error}
  \error(t,p) = \Vert f(\bm{x}, t) - \hat{f}(\bm{x}, t) \Vert_p.
\end{equation}

We first observe that for our approximation space $\broken$ we can split up the error as a sum over all cells
\begin{equation}
  \label{eq:lp-norm-broken}
 %\Vert f(\bm{x}) \Vert = \sum_{\cell[] \in \broken} \Vert f_{\cell[]} (\bm{x}) \Vert_p,
  \error(t,p) = \sum_{\cell[] \in \broken} \error_{\cell[]}(t,p)
\end{equation}
where $\error_{\cell[]}$ is the error in the cell $\cell[]$.

\todo{Describe how to integrate the error for cells, and effect of volume}
We evaluate the cell-wise error with Gaussian quadrature using \cref{eq:integration-by-substitution}
 \begin{equation}
   \Vert f_K(x) \Vert_p = \left( V \sum_{\bm{i}} \vert f(\bm{x}_i, t) - \hat{f}(\bm{x}_i, t) \vert^p w_{{\bm{i}}}  \right)^{1/p},
 \end{equation}
where $V = \Delta x \Delta y$ is the volume of each two-dimensional cell.
The sum runs over all quadrature nodes $x_i$ with weights $w_i$.
We use ten nodes.

For the limiting case of $p \to \infty$ we use the maximum point-wise error.


We compute the error for the manufactured-solution scenario (\cref{sec:manufactured-solution}).
Looking at the plot for the $L_2$ error (refer to plot here) shows that our method converges for all tested polynomial orders.
\todo{Preliminary results used here!}

To compute the numerical order of convergence, we perform a linear regression of the logarithm of the error vs.\ the logarithm of the meshsize.
The size of the slope is then the convergence order.
We can see (\cref{tab:convergence-order}) that our numerical convergence rate increases with the polynomial order.
We do not achieve the optimal theoretical order of convergence of $N+1$.
In~\cite{dumbser2010arbitrary}, which use a similar numerical method and the same scenario, but a different grid, \citeauthor{dumbser2010arbitrary} achieves the optimal order for most polynomial orders.
Some \textsc{dg}-methods of odd order only achieve a numerical convergence order of $N+\nicefrac{1}{2}$.

\section{Accurate CFD}
\section{Accurate Clouds}
\section{AMR Efficiency}
\section{Reactive Euler \textit{\&} Navier Stokes}

%%% Local Variables:
%%% mode: latex
%%% TeX-master: "../main"
%%% End:

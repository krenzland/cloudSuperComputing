\chapter{Introduction}\label{chap:introduction}
\sidetitle{Climate modeling}
Simulations are an important part of weather and climate forecasts.
Both types of forecasts are difficult problems that are limited by a lack of computational power.
Especially in a time, where the realization of exascale supercomputers is imminent, scalable numericals model become more and more important.
Current climate and weather models are not ready for the exascale era~\cite{schulthess2018reflecting}.
While~\cite{schulthess2018reflecting} proposes a spatial resolution of $\SI{1}{\kilo\meter}$ as a goal, current state of the art simulations operate on a resolution that is an order of magnitude coarser.

Of course, a finer grid is not useful if the computational effort is too large.
Thus, they also propose a compute rate of one simulated years per wall-clock day.
An example of a recent supercomputing effort in this domain~\cite{fuhrer2018near} achieved a performance of 0.043 simulated years per wall-clock day with a resolution of $\SI{930}{\m}$ for a near-global simulation on the Piz Daint supercomputer.
They used a \textsc{cosmo}-model which is based on a finite-difference approach.

\sidetitle{Discontinuous Galerkin}
The discontinuous Galerkin (\dg{}) methods are an exciting numerical method as they are relatively easy to scale.
In addition, they can be of high order and can perform local grid refinement~\cite{hesthaven2008nodal}.
These advantages are only some reasons why they have been investigated for weather and climate modeling~\cite{muller2010adaptive,giraldo2008study}.
Recently, \ader{} discontinuous Galerkin methods has been proposed, for example in~\cite{dumbser2008unified}.
This method is scalable and of arbitrary order in both time and space.
With an appropiate limiter, it is also stable, even in the vicinity of discontinuities~\cite{dumbser2016simple}.
It is thus a promising candidate for a numerical scheme for the exascale era.

The goal of this thesis is to evaluate the \aderdg{} method for testing scenarios, which have been devised as benchmark examples for meteorological applications~\cite{robert1993bubble,giraldo2008study}.
We implement this with the \exahype{}-engine, which is \enquote{An Exascale Hyperbolic \pde{} Engine}.

We make the following contributions:
\begin{itemize}
\item In \cref{sec:navier-stokes}, we describe a coupling of the compressible Navier-Stokes equations for two and three dimensional problems.
Furthermore, we adapt the equation set for scenarios with gravity.
In addition, we present a coupling with an advection-reaction term.
\item We describe an \aderdg{} (\cref{sec:ader-dg}) and \muscl{}-scheme (\cref{sec:muscl}) for problems with diffusive terms.
  Moreover, we describe a combination of both methods in \cref{sec:limiting}, where we use the \muscl{}-scheme in cases where the \aderdg{}-method is unstable.
\item In \cref{sec:amr}, we introduce an adaptive mesh refinement (\amr{}) method that checks, whether a local cell is an outlier.
  As indicator function we use the total variation of the discrete solution and as outlier detection we use the mean and standard variance of this function.
\item In \cref{chap:implementation} we show how we implemented our proposed method using the \exahype{}-engine.
\item We present seven scenarios that stem from different domains in \cref{chap:scenarios}.
\item We check the performance of our proposed method in~\cref{chap:results}.
  \begin{itemize}
  \item In \cref{sec:results-convergence} we perform a convergence test.
  \item In~\cref{sec:cfd} we investigate the ability of our scheme to simulate classical fluid mechanics (\textsc{cfd}) test scenarios correctly.
  \item We evaluate the simulation quality of atmospheric flows in \cref{sec:results-cloud}.
  \item For these scenarios, we perform a time-to-solution test for our \amr{} strategy (\cref{sec:results-tts-amr}).
  \item Finally, we evaluate the implementation of the reactive terms in \cref{sec:results-reactive}.
\end{itemize}
\end{itemize}

%%% Local Variables:
%%% mode: latex
%%% TeX-master: "../main"
%%% End:
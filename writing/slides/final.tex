% https://tex.stackexchange.com/questions/123106/detect-aspect-ratio-in-beamer?noredirect=1&lq=1
\documentclass[aspectratio=169]{beamer}
%\documentclass[]{beamer}
%\usepackage{eulervm}
\usetheme{metropolis} % Use metropolis theme
\usepackage{appendixnumberbeamer}
\usepackage{lmodern}


\renewcommand{\footnoterule}{%
  \hspace{2cm}
  \kern -3pt
  \hrule width \textwidth height 1pt
  \kern 2pt
}
\usepackage[utf8]{inputenc}
\usepackage[T1]{fontenc}
\usepackage[autostyle, english = british]{csquotes}
\usepackage[british]{babel}
\usepackage{todonotes}
\usepackage[%
  backend=biber,
  doi=false,
  url=false,
  isbn=false,
  eprint=false,
  style=verbose,
  citestyle=verbose,
  hyperref=true,
  maxnames=99,
  minnames=1,
  maxbibnames=99,
  firstinits,
  uniquename=init]{biblatex}
%\DeclareFieldFormat[inproceedings, article]{pages}{}%
%\DeclareFieldFormat[inproceedings]{organization}{}%
  % ignore some field for citations
\DeclareSourcemap{
  \maps[datatype=bibtex, overwrite]{
    \map{
      \step[fieldset=edition, null]
      \step[fieldset=publisher, null]
      \step[fieldset=pages, null]
      \step[fieldset=organization, null]
    }
  }
}

\addbibresource{../bibliography.bib}

\let\oldfootnotesize\footnotesize
\renewcommand*{\footnotesize}{\oldfootnotesize\fontsize{6}{6}}

\usepackage{caption}
\usepackage{xpatch}
\usepackage{xcolor}
\usepackage{amsmath}
\usepackage{mathtools} % for \mathclap
\usepackage{nicefrac}
\usepackage{physics} % for derivatives
\usepackage{varioref}
\usepackage{siunitx}
\usepackage{hyperref}
\usepackage[noabbrev]{cleveref}
\newcommand{\creflastconjunction}{, and\nobreakspace} % use Oxford comma
\usepackage{todonotes}
\usepackage{multimedia}

\graphicspath{{../figures/}}
\let\boundary\undefined%

\usepackage{bm}


% Commands/Macros
% Names of methods or so
\newcommand{\muscl}{\textsc{muscl}-Hancock}
\newcommand{\dg}{\textsc{dg}}
\newcommand{\ader}{\textsc{ader}}
\newcommand{\aderdg}{\textsc{ader-dg}}
\newcommand{\amr}{\textsc{amr}}
\newcommand{\pde}{\textsc{pde}}
\newcommand{\cpp}{C\texttt{++}}
\newcommand{\tbb}{\textsc{tbb}}
\newcommand{\mpi}{\textsc{mpi}}

\newcommand{\softwareName}[1]{\textit{#1}}
\newcommand{\exahype}{\softwareName{ExaHyPE}}
\newcommand{\peano}{\softwareName{Peano}}
\newcommand{\className}[1]{\texttt{#1}}
\newcommand{\funName}[1]{\textsf{#1}}
\newcommand{\fileName}[1]{\textbf{#1}}
\newcommand{\varName}[1]{\enquote{#1}}



% Variables and other equation stuff
\newcommand{\Q}{\bm{Q}}
\newcommand{\gradQ}{\gradient{\Q}}
\newcommand{\Qrho}{\rho}
\newcommand{\Qj}{\rho \bm{v}}
\newcommand{\Qv}{\bm{v}}
\newcommand{\QE}{\rho E}
\newcommand{\QZZ}{Z} % TODO: Name?
\newcommand{\QZ}{\rho \QZZ}
\newcommand{\potT}{\theta}
\newcommand{\backgroundPotT}{\overline{\theta}}
\newcommand{\pertubationPotT}{\theta'}
\newcommand{\stressT}{\bm{\sigma}}
\newcommand{\pressure}{p}
\newcommand{\maxConvEigen}[1][]{
  \vert%
  \lambda_c^{\text{max}}
  \notblank{#1}{\left(#1\right)}{}
  \vert%
}
\newcommand{\maxViscEigen}[1][]{
  \vert%
  \lambda_v^{\text{max}}
  \notblank{#1}{\left(#1\right)}{}
  \vert%
}
\newcommand{\Riemann}{\operatorname{Riemann}}

% Domain
\newcommand{\domain}{\Omega}
\newcommand{\boundary}{\partial \domain}
\newcommand{\broken}{\domain}

% Cells
\newcommand{\cell}[1][]{C_{#1}}
\newcommand{\refCell}[1][]{\hat{C}_{#1}}
\newcommand{\cellb}{\partial{} \cell}
\newcommand{\refCellb}{\partial{} \refCell}

% Basis functions
\newcommand{\lagrange}[1][i]{\varphi_{#1}}
\newcommand{\lagrangeRef}[1][i]{\hat{\varphi}_{#1}}
\NewDocumentCommand{\sbasis}{ O{} O{} }{\phi^{#1}_{#2}}
\NewDocumentCommand{\sbasisRef}{ O{} O{} }{\hat{\phi}^{#1}_{#2}} %{\hat{\sbasis[#1][#2]}}
\NewDocumentCommand{\stbasis}{ O{} O{} }{ \psi^{#1}_{#2}}
\NewDocumentCommand{\stbasisRef}{ O{} O{} }{\hat{\psi}^{#1}_{#2}} %{\hat{\stbasis[#1][#2]}}
%\newcommand{\stestfunction}[1]{\sbasis[#1]}
\NewDocumentCommand{\stestfunction}{ O{} O{} }{\sbasis[#1][#2]}
\NewDocumentCommand{\sttestfunction}{ O{} O{} }{\stbasis[#1][#2]}

% Bold if no index is supplied.
\newcommand{\bmempty}[2]{
\notblank{#2}{#1}{\bm{#1}}
}

% Space-time dofs
\NewDocumentCommand{\stpredictor}{ O{} O{} }{\overline{\bmempty{q}{#2}}^{#1}_{#2}}
\NewDocumentCommand{\stpredictorCoeff}{ O{} O{} }{\underline{\overline{\bmempty{q}{#2}}}^{#1}_{#2}}

\NewDocumentCommand{\stflux}{ O{} O{} }{\overline{\bmempty{F}{#2}}^{#1}_{#2}}
\NewDocumentCommand{\stfluxCoeff}{ O{} O{} }{\underline{\overline{\bmempty{F}{#2}}}^{#1}_{#2}}

% \NewDocumentCommand{\stsol}{ O{} O{} }{\overline{u}^{#1}_{#2}}
% \NewDocumentCommand{\stsol}{ O{} O{} }{\underline{\overline{u}}^{#1}_{#2}}

\NewDocumentCommand{\stsource}{ O{} O{} }{\overline{\bmempty{S}{#2}}^{#1}_{#2}}
\NewDocumentCommand{\stsourceCoeff}{ O{} O{} }{\underline{\overline{\bmempty{S}{#2}}}^{#1}_{#2}}

% Space dofs
\NewDocumentCommand{\spredictor}{ O{} O{} }{\bmempty{q}{#2}^{#1}_{#2}}
\NewDocumentCommand{\spredictorCoeff}{ O{} O{} }{\underline{q}^{#1}_{#2}}

\NewDocumentCommand{\ssource}{ O{} O{} }{\bmempty{S}{#2}^{#1}_{#2}}
\NewDocumentCommand{\ssourceCoeff}{ O{} O{} }{\underline{\bmempty{S}{#2}}^{#1}_{#2}}

\NewDocumentCommand{\ssol}{ O{} O{} }{\bmempty{u}{#2}^{#1}_{#2}}
\NewDocumentCommand{\ssolCoeff}{ O{} O{} }{\underline{\bmempty{u}{#2}}^{#1}_{#2}}

\NewDocumentCommand{\sflux}{ O{} O{} }{\bmempty{F}{#2}^{#1}_{#2}}
\NewDocumentCommand{\sfluxCoeff}{ O{} O{} }{\underline{\bmempty{F}{#2}}^{#1}_{#2}}

% Equation parts
\newcommand{\flux}{\bm{F}}
\newcommand{\viscFlux}{\flux^{v}}
\newcommand{\hyperFlux}{\flux^{h}}
%\newcommand{\source}{\bm{S}}
\newcommand{\source}[1][]{
  \notblank{#1}{
S_{#1}
}{
\bm{S}
}
}

% Integrals
\newcommand{\curTime}{t^k}
\newcommand{\nextTime}{t^{k+1}}
\newcommand{\intdt}[1]{\int_{\curTime}^{\nextTime} #1 \dd{t}}
\newcommand{\intdcell}[1]{\int_{\cell} #1 \dd{\bm{x}}}
\newcommand{\intdrefcell}[1]{\int_{\refCell} #1 \dd{\hat{\bm{x}}}}
\newcommand{\intdcellb}[1]{\int_{\cellb} #1 \dd{S}} % TODO: define boundary of cell
\newcommand{\intdrefcellb}[1]{\int_{\refCellb} #1 \hat{\dd{S}}} % TODO: define boundary of cell
\newcommand{\intddomain}[1]{\int_{\domain} #1 \dd{\bm{x}}}

% Yet to be sorted
\newcommand{\quadWeight}[1][i]{w_{#1}}
\newcommand{\quadNode}[1][i]{\hat{x}_{#1}}
\newcommand{\quadNodeNd}[1][i]{\hat{\bm{x}}_{#1}}
\newcommand{\sbasisSize}{(N+1)^\dimensions}
\newcommand{\stbasisSize}{(N+1)^{\dimensions+1}}
\newcommand{\normal}{\bm{n}}

\newcommand{\pleft}{\bm{L}}
\newcommand{\prightsol}{\bm{r}}
\newcommand{\prightpred}{\bm{w}}

\newcommand{\reactionTimescale}{\tau}
\newcommand{\reactionTemperature}{T_{\text{ign}}}

\newcommand{\tv}{\operatorname{TV}}

\newcommand{\NVar}{N^{\text{var}}}
\newcommand{\dimensions}{d}
\newcommand{\mapping}{\mathcal{M}}
\newcommand{\volume}{V}
\newcommand{\cellCenter}{\operatorname{cell-center}}

% MUSCL
\newcommand{\cellAvg}[1][i,j]{\bm{U}_{#1}}
\newcommand{\extrapolatedCellAvg}[3][i,j]{\cellAvg[#1]^{#3 #2} = \cellAvg #3 \frac{1}{2} \slope{#2}}%
\newcommand{\evolvedCellAvg}[2][i,j]{\hat{\bm{U}}_{#1}^{#2}}
\newcommand{\sign}{\operatorname{sign}}
\newcommand{\minmod}{\operatorname{minmod}}
\newcommand{\slope}[2][i,j]{\bm{s}^{#2}_{#1}}
\newcommand{\gradCellAvg}[1][i,j]{\gradient{\cellAvg[#1]}}
\newcommand{\fluxX}{\flux_x}
\newcommand{\fluxY}{\flux_y}%

% https://tex.stackexchange.com/questions/123106/detect-aspect-ratio-in-beamer?noredirect=1&lq=1

\makeatletter
\newcommand\ifratio[3]{%
\ifnum#1=169%
    \ifdim\beamer@paperwidth=16.00cm\relax%
        \ifdim\beamer@paperheight=9.00cm\relax%
            #2%
        \else%
            #3%
        \fi%
    \else%
        #3%
    \fi%
\else%
    \ifnum#1=43%
        \ifdim\beamer@paperwidth=12.80cm\relax%
            \ifdim\beamer@paperheight=9.60cm\relax%
                #2%
            \else%
                #3%
            \fi%
        \else%
            #3%
        \fi%
    \fi%
\fi%
}
\makeatother


% \newcommand{\Qrho}{\ensuremath{\rho}}
% \newcommand{\Qj}{\ensuremath{\rho \bm{v}}}
% %\newcommand{\Qv}{\ensuremath{\Qrho^{-1} \Qj}}
% \newcommand{\Qv}{\ensuremath{\bm{v}}}
% \newcommand{\QE}{\ensuremath{\rho E}}
% \newcommand{\stressT}{\ensuremath{\bm{\sigma}}}
% \newcommand{\pressure}{\ensuremath{p}}
% \newcommand{\maxConvEigen}{\ensuremath{\vert \lambda_c^{\text{max}} \vert}}
% \newcommand{\maxViscEigen}{\ensuremath{\vert \lambda_v^{\text{max}} \vert}}

\newcommand{\cn}{\footnote{\enquote{TODO: Citation.}, 2017, \textbf{Conference}, \textit{Author 1, Author 2, Author 3, Author 4} }}

\title{Cloud Simulation with the ExaHyPE-Engine\\Master Thesis Final}
\author{Lukas Krenz\\Adviser: Leonhard Rannabauer\\Supervisor: Prof.\ Dr.\ Michael Bader}
\date{January 24, 2019} 
\institute{\textsc{tum}, Chair for Scientific Computing}

\begin{document}
\maketitle
\begin{frame}{Two Bubbles: Hydrostatic Equilibrium\footfullcite{robert1993bubble}}
  \begin{columns}
    \begin{column}[t]{0.5\textwidth}
      \begin{itemize}
      \item 
  Air is in hydrostatic equilibrium:

  Gravitational force and pressure-gradient force are \textbf{exactly balanced}.

  \item Constant potential temperature (temperature normalized by pressure) with larger, warm bubble and small, cold bubble on top.
      
  \end{itemize}
    \end{column}~%
    \begin{column}[t]{0.5\textwidth}
      \begin{figure}[h]
        \ifratio{43}{
          \includegraphics{beamer_43_hydrostatic_equilibrium}
        } {
          \includegraphics{beamer_169_hydrostatic_equilibrium}
        }
\caption{Background pressure in equilibrium}
\end{figure}
    \end{column}
  \end{columns}
\end{frame}

\begin{frame}
  \frametitle{Two Bubbles: Simulation}
   \begin{figure}[h]
    \centering
    \ifratio{43}{
    \movie[width=0.9\textwidth, height=0.6\textwidth, autostart,, loop, poster]{}{two_bubbles_cartesian.ogv}
  } {
    \movie[width=0.7\textwidth, height=0.466666\textwidth, autostart,, loop, poster]{}{two_bubbles_cartesian.ogv}
  }
    \caption{Two Bubbles scenario}
  \end{figure}
\end{frame}

\begin{frame}{The ADER-DG Approach\footfullcite{dumbser2008unified}}
  Solve \textbf{hyperbolic conservation laws} of the form
\begin{equation}
  \label{eq:conservation-law}
 \frac{\partial}{\partial_t}  \Q + \divergence{\flux(\Q)} = \source(\bm{x}, t, \Q)
\end{equation}
with $Q$ vector of conserved variables, $\bm{x}$ position, $t$ time,  $\nabla \cdot F(Q)$ divergence of flux and $S(\bm{x}, t, Q)$ source term.

\textbf{Discontinuous Galerkin} (\textsc{DG}) divides domain into disjoint elements, approximates solutions by piecewise-polynomials.
Elements are connected by solving the Riemann problem.

\textbf{ADER}-Approach uses space-time polynomials for time integration instead of Runge-Kutta procedures.

Implemented in the \textsc{PDE}-Framework \textbf{ExaHyPE}.
\end{frame}

\begin{frame}{The Navier-Stokes Equations (with advection coupling\footfullcite{hidalgo2011ader})}
  Vector of conserved quantities:
\begin{equation}
  \label{eq:conserved-variables}
 \Q = \left( \Qrho, \Qj, \QE, \QZ \right) 
\end{equation}
With $\Qrho$ density of fluid, $\Qj$ velocity density, $\QE$ energy density, $\QZ$ mass fraction of unburnt gas.

Flux:
\begin{equation}
  F(\Q, \textcolor{orange}{\gradQ}) = 
  \begin{pmatrix}
    \Qj \\
    \Qv  \otimes \Qj + \bm{I} \pressure + \textcolor{orange}{\stressT} (\Q, \gradQ)  \\
    \Qv \cdot \left(\bm{I} \QE + \bm{I} \pressure + \textcolor{orange}{\stressT (\Q, \gradQ)} \right) -
    \textcolor{orange}{\kappa \nabla T}\\
    \Qj \QZZ
  \end{pmatrix}
\end{equation}

$T$ temperature, $\pressure(\Q)$ pressure,
$\textcolor{orange}{\stressT}$ stress tensor, $\textcolor{orange}{\kappa \nabla T}$ heat diffusion term
\end{frame}

\begin{frame}{Problem: Not hyperbolic}
  \textsc{DG} solves equations of the form (e.g.\ Euler):
  \begin{equation}
  \label{eq:conservation-law}
 \pdv{\Q}{t} + \divergence{\flux(\Q)} = \source(\bm{x}, t, \Q)
\end{equation}

We have:
\begin{equation}
  \label{eq:conservation-law-gradient}
 \pdv{\Q}{t} + \divergence{\flux(\Q, \textcolor{orange}{\gradQ})} = \source(\bm{x}, t, \Q)
\end{equation}

\textbf{Solution}:
Modify numerical flux (Riemann solver), time step size (\textsc{cfl}-Condition) and boundary conditions to \textbf{allow diffusive terms}\footfullcite{dumbser2010arbitrary}.

\alert{No explicit discretization} of gradient $\gradQ$.
\end{frame}

\begin{frame}{Convergence Tests}
  \begin{figure}[h]
    \centering
    \ifratio{43}{
      \includegraphics[]{{beamer_43_convergence}}
  } {
    \includegraphics[]{{beamer_169_convergence}}
  }

    \caption{Errors. $N$ polynomial order. Dotted lines optimal order $N+1$. }
  \end{figure}
\end{frame}


\begin{frame}{Adaptive Mesh Refinement: Indicator}
  $\bm{f}(\bm{x}): \mathbb{R}^{N_\text{vars}} \to \mathbb{R}$ maps solution to indicator variable.

  In our case: potential temperature.

  Total variation of $f$:
\begin{equation}
  \label{eq:tv}
  \tv \left[ f(\bm{x}) \right] =
  \Vert
\intdcell{ \vert \gradient{f \left( \bm{x} \right)} \vert }
\Vert_1
\end{equation}

\end{frame}  

\begin{frame}{Adaptive Mesh Refinement: Global Criterion}
  \begin{block}{Chebyshev's inequality}
\begin{equation}
  \label{eq:chebychev}
  \mathbb{P}(\vert X - \mu \vert \geq c \sigma) \leq \frac{1}{c^2}
\end{equation}
\end{block}
  
\begin{block}{Criterion}
  
\begin{equation}
  \label{eq:refinement-criterion}
  \operatorname{evaluate-refinement}(\Q, \mu, \sigma) =
  \begin{cases}
    \text{refine} & \text{if } \tv(\Q) \geq \mu + T_\text{refine} \sigma \\
    \text{delete} & \text{if } \tv(\Q) < \mu + T_\text{delete} \sigma \\
    \text{keep} & \text{otherwise}
    \end{cases}
\end{equation}
Computation of $\mu, \sigma$ with stable, pairwise reduction\footfullcite{chan1982updating}.

Global observables implemented in \exahype{} for \textbf{general} pairwise reductions.
\end{block}
\end{frame}  

\begin{frame}
  \frametitle{AMR vs.\ fully refined grid: Settings \textit{\&} time
    to solution}
  \begin{block}{Settings}
    \begin{description}
    \item[AMR] Coarse mesh size with $27 \times 27$ cells, one adaptive level.

      Potentially up to $81 \times 81$ cells.

      $T_\text{refine} = 2.5$ and $T_\text{delete} = -0.5$.
    \item[Reference] $81 \times 81$ cells.
    \end{description}
  \end{block}
  Both polynomial order 4.

  Fully refined grid took \textbf{71\%} longer!
\end{frame}


\begin{frame}
  \frametitle{AMR vs.\ fully refined grid: Grid}
   \begin{figure}[h]
    \centering
    \ifratio{43}{
    \movie[width=0.9\textwidth, height=0.6\textwidth, autostart,, loop, poster]{}{two_bubbles_legendre.ogv}
  }{
    \movie[width=0.7\textwidth, height=0.46666\textwidth, autostart,, loop, poster]{}{two_bubbles_legendre.ogv}
  }
    \caption{Two bubbles scenario, with \amr{}}
  \end{figure}
\end{frame}

\begin{frame}
  \frametitle{AMR vs.\ fully refined grid: Error}
  \begin{figure}[H]
    \centering
    \ifratio{43}{
\includegraphics[width=0.9\textwidth]{thesis_two_bubbles_amr_error}
  } {
\includegraphics[width=0.6\textwidth]{thesis_two_bubbles_amr_error}
  }
\caption{Potential temperature of fully refined solution minus \amr{} grid.}
  \end{figure}
\end{frame}

\begin{frame}{MUSCL-Hancock Scheme~\footfullcite{vanLeer1979towards}}
  Finite Volume scheme: store only \textbf{cell averages}

  \textbf{Reconstruction} of linear function

  \textbf{Second order} in time and space

  \textbf{Stabilized} with (minmod) slope limiter

  Problem in \exahype{}: Boundary conditions \textbf{unstable}, can only prescribe values at boundary.
\end{frame}

\begin{frame}{Finite Volume Limiting\footfullcite{dumbser2016simple}}
  \begin{description}
\item[ADER-DG] high order, but unstable
\item[MUSCL] low order, but stable
\item[Combination] high order \textbf{and} stable
  \end{description}

Recompute if solution:
\begin{itemize}
\item varies by a suspicious amount, using discrete maximum principle (\textsc{dmp}) as heuristic
\item is NaN or $\infty$
\item is unphysical
 \begin{equation}
  \label{eq:limiting-physical}
  \operatorname{is-admissible}(\Q) =
  \begin{cases}
    \text{true} & \Qrho > 0 \land \pressure > 0 \land \QZZ \geq 0 \\
    \text{false} & \text{otherwise}
  \end{cases}
\end{equation}
\end{itemize}
\end{frame}

\begin{frame}\frametitle{Reactive Source\footfullcite{hidalgo2011ader,helzel2000modified}}
Source term for mass fraction $\QZ$:  
 \begin{equation}\label{eq:source-reaction}
  \source[\QZ] = - K(T) \QZ
\end{equation}

Reaction model:
\begin{equation}\label{eq:reaction-model}
  K(T) = 
\begin{cases}
  - \frac{1}{\reactionTimescale} & T > \reactionTemperature\\
  0 & \text{otherwise}
\end{cases}  
\end{equation}
$\reactionTimescale = 0.01$ timescale of reaction

$\reactionTemperature$ activation temperature
\end{frame}

\begin{frame}
  \frametitle{Result Reactive}
   \begin{figure}[h]
    \centering
    \ifratio{43}{
    \movie[width=0.9\textwidth, height=0.6\textwidth, autostart,, loop, poster]{}{detonation.ogv}
  }{
    \movie[width=0.7\textwidth, height=0.46666\textwidth, autostart,, loop, poster]{}{detonation.ogv}
  }
    \caption{Detonation. Height is pressure, color refinement.
    Reactive Euler Equations.}
  \end{figure}
\end{frame}

\begin{frame}{Summary}
  \begin{block}{}
  \begin{itemize}
  \item \aderdg{} \textbf{converges} for Navier Stokes equations
  \item \aderdg{} shows \textbf{good results} for large set of scenarios (many not shown here!)
  \item (Limiting) \aderdg{} works for reactive Euler \textit{\&} Navier-Stokes
  \item \amr{}-criterion is \textbf{faster} and tracks cloud \textbf{correctly}.
  \item All \textbf{extensions} to engine (diffusive fluxes, global observables) useful for \textbf{other users} of \exahype{}.
  \end{itemize}
  \end{block}

  \begin{block}{Future Work}
    \begin{itemize}
    \item Extension to stiff source terms
    \item Boundary conditions for \muscl{}.  
    \end{itemize}

  \end{block}
\end{frame}

\appendix
\begin{frame}
  \frametitle{Reference Solution Two Bubbles\footfullcite{muller2010adaptive}}
  \begin{center}
    \ifratio{43}{
\only<1>{\includegraphics[scale=0.8,trim={0cm 16.3cm 0cm 0cm}, clip]{beamer_two_bubbles_contour_collage_alternative}}
\only<2>{\includegraphics[scale=0.8,trim={0cm 11.0cm 0cm 5.2cm}, clip]{beamer_two_bubbles_contour_collage_alternative}}
\only<3>{\includegraphics[scale=0.8,trim={0cm 5.6cm 0cm 10.6cm}, clip]{beamer_two_bubbles_contour_collage_alternative}} 
\only<4>{\includegraphics[scale=0.8,trim={0cm 0.0cm 0cm 16cm}, clip]{beamer_two_bubbles_contour_collage_alternative}}

} {
\only<1>{\includegraphics[scale=1.0,trim={0cm 16.3cm 0cm 0cm}, clip]{beamer_two_bubbles_contour_collage_alternative}}
\only<2>{\includegraphics[scale=1.0,trim={0cm 11.0cm 0cm 5.2cm}, clip]{beamer_two_bubbles_contour_collage_alternative}}
\only<3>{\includegraphics[scale=1.0,trim={0cm 5.6cm 0cm 10.6cm}, clip]{beamer_two_bubbles_contour_collage_alternative}} 
\only<4>{\includegraphics[scale=1.0,trim={0cm 0.0cm 0cm 16cm}, clip]{beamer_two_bubbles_contour_collage_alternative}}
}
 \end{center}
\end{frame}

\end{document}
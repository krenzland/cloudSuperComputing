\documentclass[runningheads]{llncs}
\title{Erratum: A High-Order Discontinuous Galerkin Solver with Dynamic Adaptive Mesh Refinement to Simulate Cloud Formation Processes }
\titlerunning{A High-Order DG Solver to Simulate Cloud Formation Processes}
\author{Lukas Krenz\orcidID{0000-0001-6378-0778} \and Leonhard Rannabauer \and Michael Bader}
\authorrunning{L.\ Krenz, L.\ Rannabauer, M.\ Bader}
\institute{Department of Informatics, Technical University of Munich\\
  \email{lukas.krenz@in.tum.de}, \email{rannabau@in.tum.de}, \email{bader@in.tum.de}
} 
\usepackage[utf8]{inputenc}
%\usepackage[T1]{fontenc}
\usepackage[american]{babel}
\usepackage[autostyle, english = american]{csquotes}
% \usepackage[%
%   backend=biber,
%   url=false,
%   doi=false,
%   % style=alphabetic,
%   % backref=true,
%   % hyperref=true,
%   maxnames=3,
%   % minnames=3,
%   % maxbibnames=99,
%   firstinits=true,
%   % uniquename=init
%   ]{biblatex}
% \addbibresource{../bibliography.bib}
\usepackage{cite}
\newcommand{\citeauthor}{???}

\usepackage{caption}
\usepackage{subcaption}
\usepackage{xparse} % for NewDocumentCommand
\usepackage{etoolbox} % for notblank (brackets only when argument)
\usepackage{xstring} % for \IfSubStr
\usepackage{xpatch}
\usepackage{xcolor}
\usepackage{amsmath}
\usepackage{amsfonts}
\usepackage{amssymb}
\usepackage{mathtools} % for \mathclap
\usepackage{nicefrac}
\usepackage{physics} % for derivatives
\usepackage{varioref}
\usepackage{nicefrac}
\usepackage{physics} % for derivatives
\usepackage[separate-uncertainty]{siunitx}
\usepackage{hyperref}
\usepackage[capitalise]{cleveref}
\newcommand{\creflastconjunction}{, and\nobreakspace} % use Oxford comma
\usepackage{graphicx}
\usepackage{multimedia}
%\graphicspath{{../figures/}}
\graphicspath{{.}}
\let\boundary\undefined%
\crefformat{equation}{\eqA{}Eq.~\eqB #2#1#3)}
\crefmultiformat{equation}{Eqs.~\eqMultiA#2#1#3\eqMultiB}%
{ and \eqMultiA#2#1#3\eqMultiB}{, \eqMultiA#2#1#3\eqMultiB}{ and~\eqMultiA#2#1#3\eqMultiB}
\newcommand{\eqA}{}
\newcommand{\eqB}{(}
\newcommand{\eqMultiA}{(}
\newcommand{\eqMultiB}{)}
\DeclareRobustCommand{\pcrefSingle}[1]{%
\begingroup%
  \renewcommand{\eqA}{(}\renewcommand{\eqB}{}%
\cref{#1}%
\endgroup%
}
\DeclareRobustCommand{\pcrefMulti}[1]{%
\begingroup%
    \renewcommand{\eqMultiA}{}\renewcommand{\eqMultiB}{}%
    (\cref{#1})%
\endgroup%
}
\DeclareRobustCommand{\pcref}[1]{%
\IfSubStr{#1}{,}{\pcrefMulti{#1}}{\pcrefSingle{#1}}%
}

\usepackage{bm}
% Commands/Macros
\newcommand{\muscl}{\textsc{muscl}-Hancock}
\newcommand{\dg}{\textsc{dg}}
\newcommand{\ader}{\textsc{ader}}
\newcommand{\aderdg}{\textsc{ader-dg}}
\newcommand{\amr}{\textsc{amr}}
\newcommand{\pde}{\textsc{pde}}
\newcommand{\tbb}{\textsc{tbb}}
\newcommand{\mpi}{\textsc{mpi}}

\newcommand{\softwareName}[1]{#1}
\newcommand{\exahype}{\softwareName{ExaHyPE}}
\newcommand{\exahypeengine}{\softwareName{ExaHyPE Engine}}

% Variables and other equation stuff
\newcommand{\Q}{\bm{Q}}
\newcommand{\gradQ}{\gradient{\Q}}
\newcommand{\Qrho}{\rho}
\newcommand{\Qj}{\rho \bm{v}}
\newcommand{\Qv}{\bm{v}}
\newcommand{\QE}{\rho E}
\newcommand{\potT}{\theta}
\newcommand{\backgroundPotT}{\overline{\theta}}
\newcommand{\pertubationPotT}{\theta'}
\newcommand{\stressT}{\bm{\sigma}}
\newcommand{\pressure}{p}

% Cells
\newcommand{\cell}[1][]{C_{#1}}

% Bold if no index is supplied.
\newcommand{\bmempty}[2]{
\notblank{#2}{#1}{\bm{#1}}
}

% Equation parts
\newcommand{\flux}{\bm{F}}
\newcommand{\viscFlux}{\flux^{v}}
\newcommand{\hyperFlux}{\flux^{h}}
%\newcommand{\source}{\bm{S}}
\newcommand{\source}[1][]{
  \notblank{#1}{
S_{#1}
}{
\bm{S}
}
}
\newcommand{\intdcell}[1]{\int_{\cell} #1 \dd{\bm{x}}}
\newcommand{\tv}{\operatorname{TV}}

\begin{document}
\maketitle 
\begin{abstract}

\keywords{\aderdg{}  \and Navier-Stokes \and Adaptive Mesh Refinement}
\end{abstract}

\subsection{Erratum}
This is an erratum for \cite{krenz2019high}.
When trying to reproduce the results, we noticed that we erronously computed the cloud scenarios with a different background pressure and viscosity as stated in the paper.
As the former may result in different physical results, we here evaluated the simulations with the correct background pressure of $\ldots$.
In our paper, while viscosity follows the Navier-Stokes equations, we use it additionally to stabilize the simulation.
Neither classical CFD scenarios (Taylor-Green, Lid-driven cavity, 3D scenario) nor the convergence test are affected by this.
We recomputed the key results with correct background pressure.

The first scenario is the cosine bubble scenario with and without adaptive mesh refinement (AMR).
We ran it as described in the paper but with the correct background pressure and a viscosity of $\mu = \ldots$. 
%TODO Refinement settings.
Figure XXX depicts the results.
Both show an excellent agreement with the original results.
The AMR simulation shows slight differences in the refinement; however, the refinement has the same quality as the original one.

The second scenario is the colliding bubble scenario.
We ran this with the Finite Volume method using a viscosity of $\mu=0$.
The results (Figure xxx) agree with the original results.
We also ran this with the ADER-DG method, with and without AMR (Figure XXX).
Similarly to the cosine bubble, the results show an excellent agreement with the original results (with some small differences in the refinement).

The third scenario is the three-dimensional cosine bubble.
We ran this with a viscosity of $\mu = \ldots$.
The results (Figure XXX) agree with our original simulations.

To summarize, we reran the affected simulations with the correct background pressure.
After adapting the viscosity, the new results agree well with the original results.

\subsection*{Acknowledgments}
This work was funded by the European Union’s Horizon 2020 Research and Innovation Programme under grant agreements 
No~671698 (project ExaHyPE, \url{www.exahype.eu}) and 
No~823844 (ChEESE centre of excellence, \url{www.cheese-coe.eu}).
Computing resources were provided by the Leibniz Supercomputing Centre (project pr83no).
Special thanks go to Dominic E.\ Charrier for his support with the implementation in the \exahypeengine{}.

\bibliographystyle{CP032_spmpsci}
\bibliography{CP032_bibliography}{}
\end{document}
